% Options for packages loaded elsewhere
\PassOptionsToPackage{unicode}{hyperref}
\PassOptionsToPackage{hyphens}{url}
%
\documentclass[
  11pt,
  oneside,
  article]{memoir}
\usepackage{amsmath,amssymb}
\usepackage{iftex}
\ifPDFTeX
  \usepackage[T1]{fontenc}
  \usepackage[utf8]{inputenc}
  \usepackage{textcomp} % provide euro and other symbols
\else % if luatex or xetex
  \usepackage{unicode-math} % this also loads fontspec
  \defaultfontfeatures{Scale=MatchLowercase}
  \defaultfontfeatures[\rmfamily]{Ligatures=TeX,Scale=1}
\fi
\usepackage{lmodern}
\ifPDFTeX\else
  % xetex/luatex font selection
\fi
% Use upquote if available, for straight quotes in verbatim environments
\IfFileExists{upquote.sty}{\usepackage{upquote}}{}
\IfFileExists{microtype.sty}{% use microtype if available
  \usepackage[]{microtype}
  \UseMicrotypeSet[protrusion]{basicmath} % disable protrusion for tt fonts
}{}
\makeatletter
\@ifundefined{KOMAClassName}{% if non-KOMA class
  \IfFileExists{parskip.sty}{%
    \usepackage{parskip}
  }{% else
    \setlength{\parindent}{0pt}
    \setlength{\parskip}{6pt plus 2pt minus 1pt}}
}{% if KOMA class
  \KOMAoptions{parskip=half}}
\makeatother
\usepackage{xcolor}
\usepackage{color}
\usepackage{fancyvrb}
\newcommand{\VerbBar}{|}
\newcommand{\VERB}{\Verb[commandchars=\\\{\}]}
\DefineVerbatimEnvironment{Highlighting}{Verbatim}{commandchars=\\\{\}}
% Add ',fontsize=\small' for more characters per line
\newenvironment{Shaded}{}{}
\newcommand{\AlertTok}[1]{\textcolor[rgb]{1.00,0.00,0.00}{\textbf{#1}}}
\newcommand{\AnnotationTok}[1]{\textcolor[rgb]{0.38,0.63,0.69}{\textbf{\textit{#1}}}}
\newcommand{\AttributeTok}[1]{\textcolor[rgb]{0.49,0.56,0.16}{#1}}
\newcommand{\BaseNTok}[1]{\textcolor[rgb]{0.25,0.63,0.44}{#1}}
\newcommand{\BuiltInTok}[1]{\textcolor[rgb]{0.00,0.50,0.00}{#1}}
\newcommand{\CharTok}[1]{\textcolor[rgb]{0.25,0.44,0.63}{#1}}
\newcommand{\CommentTok}[1]{\textcolor[rgb]{0.38,0.63,0.69}{\textit{#1}}}
\newcommand{\CommentVarTok}[1]{\textcolor[rgb]{0.38,0.63,0.69}{\textbf{\textit{#1}}}}
\newcommand{\ConstantTok}[1]{\textcolor[rgb]{0.53,0.00,0.00}{#1}}
\newcommand{\ControlFlowTok}[1]{\textcolor[rgb]{0.00,0.44,0.13}{\textbf{#1}}}
\newcommand{\DataTypeTok}[1]{\textcolor[rgb]{0.56,0.13,0.00}{#1}}
\newcommand{\DecValTok}[1]{\textcolor[rgb]{0.25,0.63,0.44}{#1}}
\newcommand{\DocumentationTok}[1]{\textcolor[rgb]{0.73,0.13,0.13}{\textit{#1}}}
\newcommand{\ErrorTok}[1]{\textcolor[rgb]{1.00,0.00,0.00}{\textbf{#1}}}
\newcommand{\ExtensionTok}[1]{#1}
\newcommand{\FloatTok}[1]{\textcolor[rgb]{0.25,0.63,0.44}{#1}}
\newcommand{\FunctionTok}[1]{\textcolor[rgb]{0.02,0.16,0.49}{#1}}
\newcommand{\ImportTok}[1]{\textcolor[rgb]{0.00,0.50,0.00}{\textbf{#1}}}
\newcommand{\InformationTok}[1]{\textcolor[rgb]{0.38,0.63,0.69}{\textbf{\textit{#1}}}}
\newcommand{\KeywordTok}[1]{\textcolor[rgb]{0.00,0.44,0.13}{\textbf{#1}}}
\newcommand{\NormalTok}[1]{#1}
\newcommand{\OperatorTok}[1]{\textcolor[rgb]{0.40,0.40,0.40}{#1}}
\newcommand{\OtherTok}[1]{\textcolor[rgb]{0.00,0.44,0.13}{#1}}
\newcommand{\PreprocessorTok}[1]{\textcolor[rgb]{0.74,0.48,0.00}{#1}}
\newcommand{\RegionMarkerTok}[1]{#1}
\newcommand{\SpecialCharTok}[1]{\textcolor[rgb]{0.25,0.44,0.63}{#1}}
\newcommand{\SpecialStringTok}[1]{\textcolor[rgb]{0.73,0.40,0.53}{#1}}
\newcommand{\StringTok}[1]{\textcolor[rgb]{0.25,0.44,0.63}{#1}}
\newcommand{\VariableTok}[1]{\textcolor[rgb]{0.10,0.09,0.49}{#1}}
\newcommand{\VerbatimStringTok}[1]{\textcolor[rgb]{0.25,0.44,0.63}{#1}}
\newcommand{\WarningTok}[1]{\textcolor[rgb]{0.38,0.63,0.69}{\textbf{\textit{#1}}}}
\setlength{\emergencystretch}{3em} % prevent overfull lines
\providecommand{\tightlist}{%
  \setlength{\itemsep}{0pt}\setlength{\parskip}{0pt}}
\setcounter{secnumdepth}{-\maxdimen} % remove section numbering
\settrims{0pt}{0pt} % page and stock same size
\settypeblocksize{*}{34.5pc}{*} % {height}{width}{ratio}
\setlrmargins{*}{*}{1} % {spine}{edge}{ratio}
\setulmarginsandblock{.98in}{.98in}{*} % height of typeblock computed
\setheadfoot{\onelineskip}{2\onelineskip} % {headheight}{footskip}
\setheaderspaces{*}{1.5\onelineskip}{*} % {headdrop}{headsep}{ratio}
\checkandfixthelayout


\usepackage{amsthm}
\usepackage{mathtools}

\usepackage[inline]{enumitem}
\usepackage{ifthen}
\usepackage[utf8]{inputenc} %allows non-ascii in bib file
\usepackage{xcolor}

\usepackage{newunicodechar}
\newunicodechar{λ}{\ensuremath{\mathnormal\lambda}}
\newunicodechar{∀}{\ensuremath{\mathnormal\forall}}
\newunicodechar{≡}{\ensuremath{\mathnormal\equiv}}
\newunicodechar{ℓ}{\ensuremath{\mathnormal\ell}}
\newunicodechar{κ}{\ensuremath{\mathnormal\kappa}}
\newunicodechar{Σ}{\ensuremath{\mathnormal\Sigma}}
\newunicodechar{⊔}{\ensuremath{\mathnormal\sqcup}}
\newunicodechar{♭}{\ensuremath{\mathnormal\flat}}
\newunicodechar{ε}{\ensuremath{\mathnormal\epsilon}}
\newunicodechar{₀}{\ensuremath{\mathnormal{_0}}}
\newunicodechar{⊥}{\ensuremath{\mathnormal\bot}}
\newunicodechar{⊤}{\ensuremath{\mathnormal\top}}
\newunicodechar{α}{\ensuremath{\mathnormal\alpha}}
\newunicodechar{β}{\ensuremath{\mathnormal\beta}}
\newunicodechar{η}{\ensuremath{\mathnormal\eta}}
\newunicodechar{⁻}{\ensuremath{\mathnormal{^-}}}
\newunicodechar{¹}{\ensuremath{\mathnormal{^1}}}
\newunicodechar{ℕ}{\ensuremath{\mathbb{N}}}
\newunicodechar{ω}{\ensuremath{\mathnormal\omega}}
\newunicodechar{∘}{\ensuremath{\mathnormal\circ}}
\newunicodechar{◃}{\ensuremath{\mathnormal\triangleleft}}
\newunicodechar{⊗}{\ensuremath{\mathnormal\otimes}}
\newunicodechar{□}{\ensuremath{\mathnormal\Box}}
\newunicodechar{∥}{\ensuremath{\mathnormal\Vert}}
\newunicodechar{⇆}{\ensuremath{\mathnormal\leftrightarrows}}
\newunicodechar{𝓤}{\ensuremath{\mathnormal\mathcal{U}}}
\newunicodechar{𝔲}{\ensuremath{\mathnormal\mathfrak{u}}}
\newunicodechar{♯}{\ensuremath{\mathnormal\sharp}}
\newunicodechar{σ}{\ensuremath{\mathnormal\sigma}}
\newunicodechar{Π}{\ensuremath{\mathnormal\Pi}}
\newunicodechar{𝕪}{\ensuremath{\mathnormal\mathbb{y}}}
\newunicodechar{≃}{\ensuremath{\mathnormal\simeq}}
\newunicodechar{μ}{\ensuremath{\mathnormal\mu}}
\newunicodechar{ρ}{\ensuremath{\mathnormal\rho}}
\newunicodechar{⇈}{\ensuremath{\mathnormal\upuparrows}}
\newunicodechar{π}{\ensuremath{\mathnormal\pi}}




\usepackage[backend=biber, backref=true, maxbibnames = 10, style = alphabetic]{biblatex}
\usepackage[bookmarks=true, colorlinks=true, linkcolor=blue!50!black,
citecolor=orange!50!black, urlcolor=orange!50!black, pdfencoding=unicode]{hyperref}
\usepackage[capitalize]{cleveref}

\usepackage{tikz}

\usepackage{amssymb}
\usepackage{newpxtext}
\usepackage[varg,bigdelims]{newpxmath}
\usepackage{mathrsfs}
\usepackage{dutchcal}
\usepackage{fontawesome}
\usepackage{ebproof}
\usepackage{stmaryrd}
\usepackage{mathpartir}
\usepackage{quiver}


% cleveref %
  \newcommand{\creflastconjunction}{, and\nobreakspace} % serial comma
  \crefformat{enumi}{\##2#1#3}
  \crefalias{chapter}{section}


% biblatex %
  \addbibresource{Library20211226.bib} 

% hyperref %
  \hypersetup{final}

% enumitem %
  \setlist{nosep}
  \setlistdepth{6}



% tikz %



  \usetikzlibrary{ 
  	cd,
  	math,
  	decorations.markings,
		decorations.pathreplacing,
  	positioning,
  	arrows.meta,
  	shapes,
		shadows,
		shadings,
  	calc,
  	fit,
  	quotes,
  	intersections,
    circuits,
    circuits.ee.IEC
  }
  
  \tikzset{
biml/.tip={Glyph[glyph math command=triangleleft, glyph length=.95ex]},
bimr/.tip={Glyph[glyph math command=triangleright, glyph length=.95ex]},
}

\tikzset{
	tick/.style={postaction={
  	decorate,
    decoration={markings, mark=at position 0.5 with
    	{\draw[-] (0,.4ex) -- (0,-.4ex);}}}
  }
} 
\tikzset{
	slash/.style={postaction={
  	decorate,
    decoration={markings, mark=at position 0.5 with
    	{\draw[-] (.3ex,.3ex) -- (-.3ex,-.3ex);}}}
  }
} 

\newcommand{\upp}{\begin{tikzcd}[row sep=6pt]~\\~\ar[u, bend left=50pt, looseness=1.3, start anchor=east, end anchor=east]\end{tikzcd}}

\newcommand{\bito}[1][]{
	\begin{tikzcd}[ampersand replacement=\&, cramped]\ar[r, biml-bimr, "#1"]\&~\end{tikzcd}  
}
\newcommand{\bifrom}[1][]{
	\begin{tikzcd}[ampersand replacement=\&, cramped]\ar[r, bimr-biml, "{#1}"]\&~\end{tikzcd}  
}
\newcommand{\bifromlong}[2][]{
	\begin{tikzcd}[ampersand replacement=\&, column sep=#2, cramped]\ar[r, bimr-biml, "#1"]\&~\end{tikzcd}  
}

% Adjunctions
\newcommand{\adj}[5][30pt]{%[size] Cat L, Left, Right, Cat R.
\begin{tikzcd}[ampersand replacement=\&, column sep=#1]
  #2\ar[r, shift left=7pt, "#3"]
  \ar[r, phantom, "\scriptstyle\Rightarrow"]\&
  #5\ar[l, shift left=7pt, "#4"]
\end{tikzcd}
}

\newcommand{\adjr}[5][30pt]{%[size] Cat R, Right, Left, Cat L.
\begin{tikzcd}[ampersand replacement=\&, column sep=#1]
  #2\ar[r, shift left=7pt, "#3"]\&
  #5\ar[l, shift left=7pt, "#4"]
  \ar[l, phantom, "\scriptstyle\Leftarrow"]
\end{tikzcd}
}

\newcommand{\xtickar}[1]{\begin{tikzcd}[baseline=-0.5ex,cramped,sep=small,ampersand 
replacement=\&]{}\ar[r,tick, "{#1}"]\&{}\end{tikzcd}}
\newcommand{\xslashar}[1]{\begin{tikzcd}[baseline=-0.5ex,cramped,sep=small,ampersand 
replacement=\&]{}\ar[r,tick, "{#1}"]\&{}\end{tikzcd}}



  
  % amsthm %
\theoremstyle{definition}
\newtheorem{definitionx}{Definition}[chapter]
\newtheorem{examplex}[definitionx]{Example}
\newtheorem{remarkx}[definitionx]{Remark}
\newtheorem{notation}[definitionx]{Notation}


\theoremstyle{plain}

\newtheorem{theorem}[definitionx]{Theorem}
\newtheorem{proposition}[definitionx]{Proposition}
\newtheorem{corollary}[definitionx]{Corollary}
\newtheorem{lemma}[definitionx]{Lemma}
\newtheorem{warning}[definitionx]{Warning}
\newtheorem*{theorem*}{Theorem}
\newtheorem*{proposition*}{Proposition}
\newtheorem*{corollary*}{Corollary}
\newtheorem*{lemma*}{Lemma}
\newtheorem*{warning*}{Warning}
%\theoremstyle{definition}
%\newtheorem{definition}[theorem]{Definition}
%\newtheorem{construction}[theorem]{Construction}

\newenvironment{example}
  {\pushQED{\qed}\renewcommand{\qedsymbol}{$\lozenge$}\examplex}
  {\popQED\endexamplex}
  
 \newenvironment{remark}
  {\pushQED{\qed}\renewcommand{\qedsymbol}{$\lozenge$}\remarkx}
  {\popQED\endremarkx}
  
  \newenvironment{definition}
  {\pushQED{\qed}\renewcommand{\qedsymbol}{$\lozenge$}\definitionx}
  {\popQED\enddefinitionx} 

    
%-------- Single symbols --------%
	
\DeclareSymbolFont{stmry}{U}{stmry}{m}{n}
\DeclareMathSymbol\fatsemi\mathop{stmry}{"23}

\DeclareFontFamily{U}{mathx}{\hyphenchar\font45}
\DeclareFontShape{U}{mathx}{m}{n}{
      <5> <6> <7> <8> <9> <10>
      <10.95> <12> <14.4> <17.28> <20.74> <24.88>
      mathx10
      }{}
\DeclareSymbolFont{mathx}{U}{mathx}{m}{n}
\DeclareFontSubstitution{U}{mathx}{m}{n}
\DeclareMathAccent{\widecheck}{0}{mathx}{"71}


%-------- Renewed commands --------%

\renewcommand{\ss}{\subseteq}

%-------- Other Macros --------%


\DeclarePairedDelimiter{\present}{\langle}{\rangle}
\DeclarePairedDelimiter{\copair}{[}{]}
\DeclarePairedDelimiter{\floor}{\lfloor}{\rfloor}
\DeclarePairedDelimiter{\ceil}{\lceil}{\rceil}
\DeclarePairedDelimiter{\corners}{\ulcorner}{\urcorner}
\DeclarePairedDelimiter{\ihom}{[}{]}

\DeclareMathOperator{\Hom}{Hom}
\DeclareMathOperator{\Mor}{Mor}
\DeclareMathOperator{\dom}{dom}
\DeclareMathOperator{\cod}{cod}
\DeclareMathOperator{\idy}{idy}
\DeclareMathOperator{\comp}{com}
\DeclareMathOperator*{\colim}{colim}
\DeclareMathOperator{\im}{im}
\DeclareMathOperator{\ob}{Ob}
\DeclareMathOperator{\Tr}{Tr}
\DeclareMathOperator{\el}{El}




\newcommand{\const}[1]{\texttt{#1}}%a constant, or named element of a set
\newcommand{\Set}[1]{\mathsf{#1}}%a named set
\newcommand{\ord}[1]{\mathsf{#1}}%an ordinal
\newcommand{\cat}[1]{\mathcal{#1}}%a generic category
\newcommand{\Cat}[1]{\mathbf{#1}}%a named category
\newcommand{\fun}[1]{\mathrm{#1}}%a function
\newcommand{\Fun}[1]{\mathit{#1}}%a named functor




\newcommand{\id}{\mathrm{id}}
\newcommand{\then}{\mathbin{\fatsemi}}

\newcommand{\cocolon}{:\!}


\newcommand{\iso}{\cong}
\newcommand{\too}{\longrightarrow}
\newcommand{\tto}{\rightrightarrows}
\newcommand{\To}[2][]{\xrightarrow[#1]{#2}}
\renewcommand{\Mapsto}[1]{\xmapsto{#1}}
\newcommand{\Tto}[3][13pt]{\begin{tikzcd}[sep=#1, cramped, ampersand replacement=\&, text height=1ex, text depth=.3ex]\ar[r, shift left=2pt, "#2"]\ar[r, shift right=2pt, "#3"']\&{}\end{tikzcd}}
\newcommand{\Too}[1]{\xrightarrow{\;\;#1\;\;}}
\newcommand{\from}{\leftarrow}
\newcommand{\ffrom}{\leftleftarrows}
\newcommand{\From}[1]{\xleftarrow{#1}}
\newcommand{\Fromm}[1]{\xleftarrow{\;\;#1\;\;}}
\newcommand{\surj}{\twoheadrightarrow}
\newcommand{\inj}{\rightarrowtail}
\newcommand{\wavyto}{\rightsquigarrow}
\newcommand{\lollipop}{\multimap}
\newcommand{\imp}{\Rightarrow}
\renewcommand{\iff}{\Leftrightarrow}
\newcommand{\down}{\mathbin{\downarrow}}
\newcommand{\fromto}{\leftrightarrows}
\newcommand{\tickar}{\xtickar{}}
\newcommand{\slashar}{\xslashar{}}



\newcommand{\inv}{^{-1}}
\newcommand{\op}{^\tn{op}}

\newcommand{\tn}[1]{\textnormal{#1}}
\newcommand{\ol}[1]{\overline{#1}}
\newcommand{\ul}[1]{\underline{#1}}
\newcommand{\wt}[1]{\widetilde{#1}}
\newcommand{\wh}[1]{\widehat{#1}}
\newcommand{\wc}[1]{\widecheck{#1}}
\newcommand{\ubar}[1]{\underaccent{\bar}{#1}}



\newcommand{\bb}{\mathbb{B}}
\newcommand{\cc}{\mathbb{C}}
\newcommand{\nn}{\mathbb{N}}
\newcommand{\pp}{\mathbb{P}}
\newcommand{\qq}{\mathbb{Q}}
\newcommand{\zz}{\mathbb{Z}}
\newcommand{\rr}{\mathbb{R}}


\newcommand{\finset}{\Cat{Fin}}
\newcommand{\smset}{\Cat{Set}}
\newcommand{\smcat}{\Cat{Cat}}
\newcommand{\catsharp}{\Cat{Cat}^{\sharp}}
\newcommand{\ppolyfun}{\mathbb{P}\Cat{olyFun}}
\newcommand{\ccatsharp}{\mathbb{C}\Cat{at}^{\sharp}}
\newcommand{\ccatsharpdisc}{\mathbb{C}\Cat{at}^{\sharp}_{\tn{disc}}}
\newcommand{\ccatsharplin}{\mathbb{C}\Cat{at}^{\sharp}_{\tn{lin}}}
\newcommand{\ccatsharpdisccon}{\mathbb{C}\Cat{at}^{\sharp}_{\tn{disc,con}}}
\newcommand{\sspan}{\mathbb{S}\Cat{pan}}
\newcommand{\en}{\Cat{End}}

\newcommand{\List}{\Fun{List}}
\newcommand{\set}{\tn{-}\Cat{Set}}




\newcommand{\yon}{\mathcal{y}}
\newcommand{\poly}{\Cat{Poly}}
\newcommand{\polycart}{\pol\yon^{\Cat{Cart}}}
\newcommand{\ppoly}{\mathbb{P}\Cat{oly}}
\newcommand{\0}{\textsf{0}}
\newcommand{\1}{\tn{\textsf{1}}}
\newcommand{\U}{\tn{\textsf{U}}}
\newcommand{\tri}{\mathbin{\triangleleft}}
\newcommand{\triright}{\mathbin{\triangleright}}
\newcommand{\tripow}[1]{^{\tri #1}}
\newcommand{\indep}{\Fun{Indep}}
\newcommand{\duoid}{\Fun{Duoid}}
\newcommand{\jump}{\pi}
\newcommand{\jumpmap}{\ol{\jump}}
\newcommand{\founds}{\Yleft}


% lenses
\newcommand{\biglens}[2]{
     \begin{bmatrix}{\vphantom{f_f^f}#2} \\ {\vphantom{f_f^f}#1} \end{bmatrix}
}
\newcommand{\littlelens}[2]{
     \begin{bsmallmatrix}{\vphantom{f}#2} \\ {\vphantom{f}#1} \end{bsmallmatrix}
}
\newcommand{\lens}[2]{
  \relax\if@display
     \biglens{#1}{#2}
  \else
     \littlelens{#1}{#2}
  \fi
}



\newcommand{\qand}{\quad\text{and}\quad}
\newcommand{\qqand}{\qquad\text{and}\qquad}


\newcommand{\coto}{\nrightarrow}
\newcommand{\cofun}{{\raisebox{2pt}{\resizebox{2.5pt}{2.5pt}{$\setminus$}}}}

\newcommand{\coalg}{\tn{-}\Cat{Coalg}}

\newcommand{\bic}[2]{{}_{#1}\Cat{Comod}_{#2}}

\newcommand{\dnote}[1]{{\quad \color{blue}$\lozenge$\;David says:}~#1\;{\color{blue}$\lozenge$}\quad}
\newcommand{\cbnote}[1]{{\quad \color{red}$\lozenge$\;C.B.\ says:}~#1\;{\color{red}$\lozenge$}\quad}


% ---- Changeable document parameters ---- %

\linespread{1.1}
\allowdisplaybreaks
\setsecnumdepth{subsection}
\settocdepth{subsection}
\setlength{\parindent}{15pt}
\setcounter{tocdepth}{1}
\usepackage{bookmark}
\IfFileExists{xurl.sty}{\usepackage{xurl}}{} % add URL line breaks if available
\urlstyle{same}
\hypersetup{
  pdftitle={Polynomial Universes and Dependent Types},
  pdfauthor={C.B. Aberlé; David I. Spivak},
  hidelinks,
  pdfcreator={LaTeX via pandoc}}

\title{Polynomial Universes and Dependent Types}
\author{C.B. Aberlé \and David I. Spivak}
\date{}

\begin{document}
\makeatletter
\@ifclassloaded{memoir}{\ifartopt\else\frontmatter\fi}{\frontmatter}
\makeatother
\maketitle

\makeatletter
\@ifclassloaded{memoir}{\ifartopt\else\mainmatter\fi}{\mainmatter}
\makeatother
\begin{abstract}

Awodey, later with Newstead, showed how polynomial pseudomonads $(u,\1,\Sigma)$ with extra structure (termed "natural models" by Awodey) hold within them the syntax and rules for dependent type theory. Their work presented these ideas clearly but ultimately led them outside of the category of polynomial functors in order to explain all of the structure possessed by such models of type theory.

This paper builds off that work---explicating the syntax and rules for dependent type theory by axiomatizing them \emph{entirely} in the language of polynomial functors. In order to handle the higher-categorical coherences required for such an explanation, we work with polynomial functors internally in the language of Homotopy Type Theory, which allows for higher-dimensional structures such as pseudomonads, etc. to be expressed purely in terms of the structure of a suitably-chosen $\infty$-category of polynomial functors. The move from set theory to Homotopy Type Theory thus has a twofold effect of enabling a simpler exposition of natural models, which is at the same time amenable to formalization in a proof assistant, such as Agda.

Moreover, the choice to remain firmly within the setting of polynomial functors reveals many additional structures of natural models that were otherwise left implicit or not considered by Awodey \& Newstead. Chief among these, we highlight the fact that every polynomial pseudomonad $(u,\1,\Sigma)$ as above that is also equipped with structure to interpret dependent product types gives rise to a self-distributive law $u\tri u\to u\tri u$, which witnesses the usual distributive law of dependent products over dependent sums.
\end{abstract}

\chapter{Introduction}\label{introduction}

Dependent type theory \cite{martin-lof1975intuitionistic} was founded by
Per Martin-Löf in 1975 (among others) to formalize constructive
mathematics. The basic idea is that the \emph{order of events} is
fundamental to the mathematical story arc: when playing out any specific
example story in that arc, the beginning of the story affects not only
the later events, but even the very terms with which the later events
will be described. For example, in the story arc of conditional
probability, one may say ``now if the set \(P\) that we are asked to
condition on happens to have measure zero, we must stop; but assuming
that's not the case then the result will be a new probability
measure.'' Here the story teller is saying that no terms will describe
what happens if \(P\) has measure zero, whereas otherwise the terms of
standard probability will apply.

Dependent types form a logical system with syntax, rules of computation,
and methods of deduction. In
\cite{awodey2014natural,awodey2018polynomial}, Awodey and later Newstead
show that there is a strong connection between dependent type theory and
polynomial functors, via their concept of \emph{natural models}, which
cleanly solve the problem of \emph{strictififying} certain identities
that typically hold only up to isomorphism in arbitrary categories, but
must hold \emph{strictly} in order for these to soundly model dependent
type theory. The solution to this problem offered by Awodey and Newstead
makes use of the type-theoretic concept of a \emph{universe}. Such
universes then turn out to naturally be regarded as polynomial functors
on a suitably-chosen category of presheaves, satisfying a certain
\emph{representability} condition.

Although the elementary structure of natural models is thus
straightforwardly described by considering them as objects in the
category of polynomial functors, Awodey and Newstead were ultimately led
outside of this category in order to fully explicate those parts of
natural models that require identities to hold only \emph{up to
isomorphism}, rather than strictly. There is thus an evident tension
between \emph{strict} and \emph{weak} identities that has not yet been
fully resolved in the story of natural models. In the present work, we
build on Awodey and Newstead's work to fully resolve this impasse by
showing how type universes can be fully axiomatized in terms of
polynomial functors, by working with polynomial functors internally in
the language of \emph{Homotopy Type Theory} (HoTT). We thus come full
circle from Awodey's original motivation to develop natural models
\emph{of} Homotopy Type Theory, to describing natural models \emph{in}
Homotopy Type Theory.

The ability for us to tell the story of natural models as set entirely 
in the category of polynomial functors has a great simplifying effect
upon the resultant theory, and reveals many additional structures, both
of polynomial universes, and of the category of polynomial functors as a
whole. As an illustration of this, we show how every polynomial universe
\(u\), regarded as a polynomial pseudomonad with additional structure,
gives rise to self-distributive law \(u\tri u\to u\tri u\), which in
particular witnesses the usual distributive law of dependent products
over dependent sums.

Moreover, the move from set theory to HoTT as a setting in which to tell
this story enables new tools to be applied for its telling. In
particular, the account of polynomial universes we develop is
well-suited to formalization in a proof assistant, and we present such a
formalization in Agda. This paper is thus itself a literate Agda
document in which all results have been fully formalized and checked for
validity.

\begin{Shaded}
\begin{Highlighting}[]
\PreprocessorTok{\{{-}\# OPTIONS {-}{-}without{-}K {-}{-}rewriting \#{-}\}}
\KeywordTok{module}\NormalTok{ poly{-}universes }\KeywordTok{where}

\KeywordTok{open} \KeywordTok{import}\NormalTok{ Agda}\OtherTok{.}\NormalTok{Primitive}
\KeywordTok{open} \KeywordTok{import}\NormalTok{ Agda}\OtherTok{.}\NormalTok{Builtin}\OtherTok{.}\NormalTok{Sigma}
\KeywordTok{open} \KeywordTok{import}\NormalTok{ Agda}\OtherTok{.}\NormalTok{Builtin}\OtherTok{.}\NormalTok{Unit}
\end{Highlighting}
\end{Shaded}

The structure of this paper is as follows:
\dnote{Fill}

\chapter{Background on Type Theory, Natural Models \&
HoTT}\label{background-on-type-theory-natural-models-hott}

We begin with a recap of natural models, dependent type theory, and
HoTT, taking this also as an opportunity to introduce the basics of our
Agda formalization.

\section{Dependent Types and their Categorical
Semantics}\label{dependent-types-and-their-categorical-semantics}

The question ``what is a type'' is as deep as it is philosophically
fraught. For present purposes, however, we need not concern ourselves so
much directly with what (dependent) type \emph{are}, as with what they
can \emph{do}, and how best to mathematically model this behavior.
Suffice it to say, then, that a type specifies rules for both
constructing and using the \emph{inhabitants} of that type in arbitrary
contexts of usage. Following standard conventions, we use the notation
\texttt{a\ :\ A} to mean that \texttt{a} is an inhabitant of type
\texttt{A}.

In Agda, one example of such a type is the \emph{unit type} \texttt{⊤},
which is defined to have a single inhabitant \texttt{tt\ :\ ⊤}, such
that for any other inhabitant \texttt{x\ :\ ⊤} we must have
\texttt{x\ =\ tt}.

Another type (or rather, family of types) of particular importance is
the \emph{universe} of types \texttt{Type}, whose inhabitants themsleves
represent types.\footnote{For consistency with the usage of the term
  ``set'' in HoTT (whereby sets are types satisfying a certain
  \emph{truncation} condition, to be explained shortly,) we relabel
  Agda's universes of types as \texttt{Type}, rather than the default
  \texttt{Set}.} So e.g.~to say that \texttt{⊤}, as defined above, is a
type, we may simply write \texttt{⊤\ :\ Type}.\dnote{The agda below involves levels and sets, but they aren't discussed above. Say something about levels?}

\begin{Shaded}
\begin{Highlighting}[]
\NormalTok{Type }\OtherTok{:} \OtherTok{(}\NormalTok{ℓ }\OtherTok{:}\NormalTok{ Level}\OtherTok{)} \OtherTok{→} \DataTypeTok{Set} \OtherTok{(}\NormalTok{lsuc ℓ}\OtherTok{)}
\NormalTok{Type ℓ }\OtherTok{=} \DataTypeTok{Set}\NormalTok{ ℓ}
\end{Highlighting}
\end{Shaded}

Given a type \texttt{A}, one may in turn consider families of types
\texttt{B\ x} indexed by, or \emph{dependent} upon aribtrary inhabitants
\texttt{x\ :\ A}. In agda, we represent such a type family \texttt{B} as
a function \texttt{A\ →\ Type}.

Given a type \texttt{A\ :\ Type} and a family of types
\texttt{B\ :\ A\ →\ Type} as above, two key examples of types we may
construct are:

\begin{itemize}
\tightlist
\item
  The \emph{dependent function type} \texttt{(x\ :\ A)\ →\ B\ x}, whose
  inhabitants are functions \texttt{λ\ x\ →\ f\ x} such that, for all
  \texttt{a\ :\ A}, we have \texttt{f\ a\ :\ B\ a}.
\item
  The \emph{dependent pair type} \texttt{Σ\ A\ B}, whose inhabitants are
  of the form \texttt{(a\ ,\ b)} for \texttt{a\ :\ A} and
  \texttt{b\ :\ B\ a}, such that there are functions
  \texttt{fst\ :\ Σ\ A\ B\ →\ A} and
  \texttt{snd\ :\ (p\ :\ Σ\ A\ B)\ →\ B\ (fst\ p)}.
\end{itemize}

Note that in the case where \texttt{B} does not depend upon
\texttt{x\ :\ A} (i.e.~the variable \texttt{x} does not appear in the
expression for \texttt{B}), these correspond to the more familiar
function type \texttt{A\ →\ B} and pair type \texttt{A\ ×\ B},
respectively. E.g. we can define the Cartesian product of two types
\texttt{A} and \texttt{B} as follows:

\begin{Shaded}
\begin{Highlighting}[]
\OtherTok{\_}\NormalTok{×}\OtherTok{\_} \OtherTok{:} \OtherTok{∀} \OtherTok{\{}\NormalTok{ℓ κ}\OtherTok{\}} \OtherTok{(}\NormalTok{A }\OtherTok{:}\NormalTok{ Type ℓ}\OtherTok{)} \OtherTok{(}\NormalTok{B }\OtherTok{:}\NormalTok{ Type κ}\OtherTok{)} \OtherTok{→}\NormalTok{ Type }\OtherTok{(}\NormalTok{ℓ ⊔ κ}\OtherTok{)}
\NormalTok{A × B }\OtherTok{=}\NormalTok{ Σ A }\OtherTok{(λ} \OtherTok{\_} \OtherTok{→}\NormalTok{ B}\OtherTok{)}
\end{Highlighting}
\end{Shaded}

In more traditional type-theoretic notation, one might see the rules for
these types written as follows: \[ 
\inferrule{}{\Gamma \vdash \top : \mathsf{Type}} \qquad \inferrule{}{\Gamma \vdash \mathsf{tt} : \top} \qquad \inferrule{\Gamma \vdash x : \top}{\Gamma \vdash x = \mathsf{tt}}
\] \[
\inferrule{\Gamma \vdash A : \mathsf{Type}\\ \Gamma, x : A \vdash B[x] : \mathsf{Type}}{\Gamma \vdash \Pi x : A . B[x] : \mathsf{Type}} \qquad \inferrule{\Gamma \vdash A : \mathsf{Type}\\ \Gamma, x : A \vdash B[x] : \mathsf{Type}}{\Gamma \vdash \Sigma x : A . B[x] : \mathsf{Type}}
\] \[
\inferrule{\Gamma, x : A \vdash f[x] : B[x]}{\Gamma \vdash \lambda x . f[x] : \Pi x : A . B[x]} \qquad \inferrule{\Gamma \vdash a : A\\ \Gamma \vdash b : B[a]}{\Gamma \vdash (a , b) : \Sigma x : A . B[x]}
\] \[
\inferrule{\Gamma \vdash f : \Pi x : A . B[x]\\ \Gamma \vdash a : A}{\Gamma \vdash f a : B[a]} \qquad \inferrule{\Gamma \vdash p : \Sigma x : A . B[x]}{\Gamma \vdash \pi_1(p) : A} \quad \inferrule{\Gamma \vdash p : \Sigma x : A . B[x]}{\Gamma \vdash \pi_2(p) : B[\pi_1(p)]}
\] \[
(\lambda x . f[x]) a = f[a] \qquad \pi_1(a , b) = a \quad \pi_2(a , b) = b
\] \[
f = \lambda x . fx \qquad p = (\pi_1(p) , \pi_2(p))
\]

The constructors \(\lambda\) and \((- , -)\) are called the
\emph{introduction} forms of \(\Pi x : A . B[x]\) and
\(\Sigma x : A . B[x]\), while \(f a\) and \(\pi_1(p), ~ \pi_2(p)\) are
called the \emph{elimination} forms of these types, respectively. One
may wonder why all typing judgments in the above rules have been
decorated with annotations of the form \(\Gamma \vdash\), for some
\(\Gamma\). In these cases, \(\Gamma\) is the \emph{context} of the
corresponding judgment, used to keep track of the types of variables
that may appear in that judgment.

Although contexts may seem rather trivial from a syntactic perspective,
they are key to understanding the categorical semantics of dependent
type theory. In particular, when modelling a dependent type theory as a
category, it is the \emph{contexts} which form the objects of this
category, with morphisms between contexts being \emph{substitutions} of
terms in the domain context for the variables of the codomain context. A
type \(A\) dependent upon variables in a context \(\Gamma\) is then
interpreted as a morphism (i.e.~substitution)
\(\Gamma, x : A \to \Gamma\), whose domain represents the context
\(\Gamma\) extended with a variable of type \(A\). We then interpret a
term \(a\) of type \(A\) in context \(\Gamma\) as a \emph{section} of
the display map representing \(A\), i.e.~\[
\begin{tikzcd}
    \Gamma & {\Gamma, x : A} \\
    & \Gamma
    \arrow["a", from=1-1, to=1-2]
    \arrow[Rightarrow, no head, from=1-1, to=2-2]
    \arrow["A", from=1-2, to=2-2]
\end{tikzcd}
\] Hence for each context \(\Gamma\), there is a category
\(\mathbf{Ty}[\Gamma]\), which is the full subcategory of the slice
category \(\mathcal{C}/\Gamma\) consisting of all display maps, wherein
objects correspond to types in context \(\Gamma\), and morphisms
correspond to terms.

In typical categorical semantics, given a substitution
\(f : \Gamma \to \Delta\), and a type \(A : \Delta, x : A \to \Delta\),
we then interpret the action of \(f\) on \(A\) as a pullback: \[
\begin{tikzcd}
    {\Gamma, x : A[f]} & {\Delta, x : A} \\
    \Gamma & \Delta
    \arrow[from=1-1, to=1-2]
    \arrow["{A[f]}"', from=1-1, to=2-1]
    \arrow["\lrcorner"{anchor=center, pos=0.125}, draw=none, from=1-1, to=2-2]
    \arrow["A", from=1-2, to=2-2]
    \arrow["f"', from=2-1, to=2-2]
\end{tikzcd}
\] In particular, then, any display map \(A : \Gamma, x : A \to \Gamma\)
induces a functor \(\mathbf{Ty}[\Gamma] \to \mathbf{Ty}[\Gamma, x : A]\)
by substitution along \(A\). The left and right adjoints to this functor
(if they exist) then correspond to dependent pair and dependent function
types, respectively.

So far, we have told a pleasingly straightforward story of how to
interpret the syntax of dependent type theory categorically.
Unfortunately, this story is a fantasy, and the interpretation of
type-theoretic syntax into categorical semantics sketched above is
unsound, as it stands. The problem in essentials is that, in the syntax
of type theory, substitution is strictly associative -- i.e.~given
substitutions \(f : \Gamma \to \Delta\) and \(g : \Delta \to \Theta\)
and a type \texttt{A}, we have \(A[g][f] = A[g[f]]\); however, in the
above categorical semantics, such iterated substitution is interpreted
via successively taking pullbacks, which is in general only associative
up to isomorphism. It seems, then, that something more is needed to
account for this kind of \emph{strictness} in the semantics of dependent
type theory.%
\footnote{Something something Beck-Chevalley\ldots{}} 
It is precisely this problem which natural models exist to solve.

\section{Natural Models}\label{natural-models}

The key insight of Awodey in formulating the notion of a natural model
is that the problem of strictness in the semantics of type theory has,
in a sense, already been solved by the notion of \emph{type universes},
such as \texttt{Type} as introduced above. Given a universe of types
\(\mathcal{U}\), rather than representing dependent types as display
maps, and substitution as pullback, we can simply represent a family of
types \(B[x]\) dependent upon a type \(A\) as a function
\(A \to \mathcal{U}\), with substitution then given by precomposition,
which is automatically strictly associative.

To interpret the syntax of dependent type theory in a category
\(\mathcal{C}\) of contexts and substitutions, it therefore suffices to
\emph{embed} \(\mathcal{C}\) into a category whose type-theoretic
internal language posesses such a universe whose types correspond to
those of \(\mathcal{C}\). For this purpose, we work in the category of
\emph{prehseaves} \(\smset^{\mathcal{C}\op}\), with the
embedding
\(\mathcal{C} \hookrightarrow \smset^{\mathcal{C}\op}\) being
nothing other than the Yoneda embedding.

The universe \(\mathcal{U}\) is then given by an object of
\(\smset^{\mathcal{C}\op}\), i.e.~an assignment, to each context
\(\Gamma:\ob\cat{C}\), of a set \(\mathsf{Ty}[\Gamma]\) of types in context
\(\Gamma\), with functions
\(\mathsf{Ty}[\Delta] \to \mathsf{Ty}[\Gamma]\) for each substitution
\(f : \Gamma \to \Delta\) that compose associatively, together with a
\(\mathcal{U}\)-indexed family of objects
\(u \in \smset^{\mathcal{C}\op}\)\!\(/\mathcal{U}\), i.e.~a natural
transformation \(u : \mathcal{U}_\bullet \Rightarrow \mathcal{U}\),
where for each context \(\Gamma\) and type
\(A \in \mathsf{Ty}[\Gamma]\), the fibre of \(u_\Gamma\) over \(A\) is
the set \(\mathsf{Tm}[\Gamma,A]\) of inhabitants of \(A\) in context
\(\Gamma\).

The condition that all types in \(\mathcal{U}\) ``belong to
\(\mathcal{C}\)'', in an appropriate sense, can then be expressed by
requiring \(u\) to be \emph{representable}, i.e.~for any representable
\(\gamma \in \smset^{\mathcal{C}\op}\) with a natural
transformation \(\alpha : \gamma \Rightarrow \mathcal{U}\), the pullback
\[
\begin{tikzcd}
    {\mathcal{\gamma} \times_{\alpha, u} \mathcal{U}_\bullet} & {\mathcal{U}_\bullet} \\
    \gamma & {\mathcal{U}}
    \arrow[Rightarrow, from=1-1, to=1-2]
    \arrow["{u[\alpha]}"', Rightarrow, from=1-1, to=2-1]
    \arrow["\lrcorner"{anchor=center, pos=0.125}, draw=none, from=1-1, to=2-2]
    \arrow["u", Rightarrow, from=1-2, to=2-2]
    \arrow["\alpha"', Rightarrow, from=2-1, to=2-2]
\end{tikzcd}
\] of \(u\) along \(\alpha\) is representable.\dnote{I've always been confused by this. If $\mathcal{C}=1$, then there is only one representable functor, the terminal set. So doesn't this condition say that $\mathcal{U}_\bullet=\mathcal{U}$? But this is not the universe for set-polynomials. Replace with something explaining why List is a universe on sets.}

The question, then, is how to express that \(\mathcal{C}\) has dependent
pair types, dependent function types, etc., in terms of the structure of
\(u\). A further insight of Awodey, toward answering this question, is
that \(u\) gives rise to a functor (indeed, a \emph{polynomial functor})
\(\overline{u} : \smset^{\mathcal{C}\op} \to \smset^{\mathcal{C}\op}\),
defined on $P:\mathcal{C}\op\to\smset$ as follows \[
\overline{u}(P)(\Gamma) = \sum_{A : \mathsf{Ty}[\Gamma]} P(\Gamma)^{\mathsf{Tm}[\Gamma, A]}
\] and much of the type-theoretic structure of \(u\) can be accounted
for in terms of this functor. For instance (for reasons to be explained
shortly), dependent pair types are given by a natural transformation
\(\sigma : \overline{u} \circ \overline{u} \Rightarrow \overline{u}\),
that is \emph{Cartesian} in that, for every
\(\alpha : P \Rightarrow Q\), the following naturality square is a
pullback \[
\begin{tikzcd}
    {\overline{u}(\overline{u}(P))} & {\overline{u}(\overline{u}(Q))} \\
    {\overline{u}(P)} & {\overline{u}(Q)}
    \arrow["{\overline{u}(\overline{u}(\alpha))}", Rightarrow, from=1-1, to=1-2]
    \arrow["{\sigma_P}"', Rightarrow, from=1-1, to=2-1]
    \arrow["\lrcorner"{anchor=center, pos=0.125}, draw=none, from=1-1, to=2-2]
    \arrow["{\sigma_Q}", Rightarrow, from=1-2, to=2-2]
    \arrow["{\overline{u}(\alpha)}"', Rightarrow, from=2-1, to=2-2]
\end{tikzcd}
\]
\dnote{Say somewhere that a map of polynomials is cartesian iff for each position, the map on directions is an iso? That's certainly how I think about this. \cite[Proposition 5.59]{Niu-Spivak-"Polybook"}}

A question that arises, then, is what structure such a natural
transformation interpreting dependent pair types must possess. It is
natural to think that \(\sigma\), along with a suitably-chosen natural
transformation \(Id \Rightarrow \overline{u}\), ought to give
\(\overline{u}\) the structure of a monad. However, this turns out to be
too strong a requirement, as it amounts to asking that
\(\Sigma x : A . (\Sigma y : B[x] . C[x,y]) = \Sigma (x,y) : (\Sigma x : A . B[x]) . C[x,y]\),
when in general this identity only holds up to isomorphism. Hence we
seem to have crossed over from Scylla of our semantics for dependent
type theory not being strict enough to interpret those identities we
expect to hold strictly, to the Charybdis of them now being too strict
to interpret the identities we expect to hold only up to isomorphism. It
was for this reason that Awodey \& Newstead were forced to ultimately go
beyond Polynomial functors in their accounts of natural models.

However, another possibility exists to solve this dilemma -- to use the
language of HoTT itself to reason about such equality-up-to-isomorphism
in natural models. For this purpose, rather than taking natural models
to be certain (representable) morphisms in
\(\smset^{\mathcal{C}\op}\), we can instead expand the
mathematical universe in which these models live to
\(\mathbf{\infty Grpd}^{\mathcal{C}\op}\), which, as an
\(\infty\)-topos, has HoTT as its internal language. Taking advantage of
this fact, we can use HoTT itself as a language for studying the
semantics of type theory, by postulating an abstract type
\(\mathcal{U}\) together with a type family
\(u : \mathcal{U} \to \mathsf{Type}\), corresponding to a representable
natural transformation
\(u : \mathcal{U}_\bullet \Rightarrow \mathcal{U}\) as above.

What remains, then, is to show how the various type-theoretic properties
of such natural models can be expressed in terms of polynomial functors
in the language of HoTT, and the complex identities to which these give
rise. For this purpose, we begin with a recap of the basics of HoTT,
before launching into a development of the theory of polynomial functors
within HoTT, with an eye toward the latter's use in the study of natural
models.

\section{Homotopy Type Theory}\label{homotopy-type-theory}

\subsection{The Identity Type}\label{the-identity-type}

Given elements \texttt{a,b\ :\ A} for some type \texttt{A}, the identity
type \texttt{a\ ≡\ b} is inductively generated from the single
constructor \texttt{refl\ :\ \{x\ :\ A\}\ →\ x\ ≡\ x}, witnessing
reflexivity of equality.

\begin{Shaded}
\begin{Highlighting}[]
\KeywordTok{open} \KeywordTok{import}\NormalTok{ Agda}\OtherTok{.}\NormalTok{Builtin}\OtherTok{.}\NormalTok{Equality}
\KeywordTok{open} \KeywordTok{import}\NormalTok{ Agda}\OtherTok{.}\NormalTok{Builtin}\OtherTok{.}\NormalTok{Equality}\OtherTok{.}\NormalTok{Rewrite}
\end{Highlighting}
\end{Shaded}

The core insight of Homotopy Type Theory is that the presence of
(intensional) identity types in a system of dependent type theory endows
each type with the structure of an \(\infty\)-groupoid, and endows each
function between types with the structure of a functor between
\(\infty\)-groupoids, etc. This allows a wealth of higher-categorical
properties and structures to be defined and studied \emph{internally} in
the language of dependent type theory.

Since an invocation of reflexivity typically occurs at the end of an
equality proof, we introduce the notation \texttt{□} as a shorthand for
\texttt{refl} as follows:

\begin{Shaded}
\begin{Highlighting}[]
\OtherTok{\_}\NormalTok{□ }\OtherTok{:} \OtherTok{∀} \OtherTok{\{}\NormalTok{ℓ}\OtherTok{\}} \OtherTok{\{}\NormalTok{A }\OtherTok{:}\NormalTok{ Type ℓ}\OtherTok{\}} \OtherTok{(}\NormalTok{a }\OtherTok{:}\NormalTok{ A}\OtherTok{)} \OtherTok{→}\NormalTok{ a ≡ a}
\NormalTok{a □ }\OtherTok{=}\NormalTok{ refl}
\end{Highlighting}
\end{Shaded}

The inductive generation of \texttt{a\ ≡\ b} from \texttt{refl} then
gives rise to the operation of \emph{transport} that allows an
inhabitant of the type \texttt{B\ a} to be converted to an inhabitant of
\texttt{B\ b} for any type family \texttt{B\ :\ (x\ :\ A)\ →\ Type}.

\begin{Shaded}
\begin{Highlighting}[]
\NormalTok{transp }\OtherTok{:} \OtherTok{∀} \OtherTok{\{}\NormalTok{ℓ κ}\OtherTok{\}} \OtherTok{\{}\NormalTok{A }\OtherTok{:}\NormalTok{ Type ℓ}\OtherTok{\}} \OtherTok{(}\NormalTok{B }\OtherTok{:}\NormalTok{ A }\OtherTok{→}\NormalTok{ Type κ}\OtherTok{)} \OtherTok{\{}\NormalTok{a a\textquotesingle{} }\OtherTok{:}\NormalTok{ A}\OtherTok{\}} 
         \OtherTok{→} \OtherTok{(}\NormalTok{e }\OtherTok{:}\NormalTok{ a ≡ a\textquotesingle{}}\OtherTok{)} \OtherTok{→}\NormalTok{ B a }\OtherTok{→}\NormalTok{ B a\textquotesingle{}}
\NormalTok{transp B refl b }\OtherTok{=}\NormalTok{ b}
\end{Highlighting}
\end{Shaded}

Transitivity of equality then follows in the usual way. Note, however,
that we take advantage of Agda's support for mixfix notation to present
transitivity in such a way as to streamline both the reading and writing
of equality proofs:

\begin{Shaded}
\begin{Highlighting}[]
\OtherTok{\_}\NormalTok{≡〈}\OtherTok{\_}\NormalTok{〉}\OtherTok{\_} \OtherTok{:} \OtherTok{∀} \OtherTok{\{}\NormalTok{ℓ}\OtherTok{\}} \OtherTok{\{}\NormalTok{A }\OtherTok{:}\NormalTok{ Type ℓ}\OtherTok{\}} \OtherTok{(}\NormalTok{a }\OtherTok{:}\NormalTok{ A}\OtherTok{)} \OtherTok{\{}\NormalTok{b c }\OtherTok{:}\NormalTok{ A}\OtherTok{\}} 
          \OtherTok{→}\NormalTok{ a ≡ b }\OtherTok{→}\NormalTok{ b ≡ c }\OtherTok{→}\NormalTok{ a ≡ c}
\NormalTok{a ≡〈 e 〉 refl }\OtherTok{=}\NormalTok{ e}
\end{Highlighting}
\end{Shaded}

Symmetry of equality follows similarly:

\begin{Shaded}
\begin{Highlighting}[]
\NormalTok{sym }\OtherTok{:} \OtherTok{∀} \OtherTok{\{}\NormalTok{ℓ}\OtherTok{\}} \OtherTok{\{}\NormalTok{A }\OtherTok{:}\NormalTok{ Type ℓ}\OtherTok{\}} \OtherTok{\{}\NormalTok{a a\textquotesingle{} }\OtherTok{:}\NormalTok{ A}\OtherTok{\}} \OtherTok{→}\NormalTok{ a ≡ a\textquotesingle{} }\OtherTok{→}\NormalTok{ a\textquotesingle{} ≡ a}
\NormalTok{sym refl }\OtherTok{=}\NormalTok{ refl}
\end{Highlighting}
\end{Shaded}

As mentioned above, each function \texttt{f\ :\ A\ →\ B} in HoTT is
canonically endowed with the structure of a functor between
\(\infty\)-groupoids, where the action of such a function \texttt{f} on
paths (i.e.~elements of the identity type) is as follows:

\begin{Shaded}
\begin{Highlighting}[]
\NormalTok{ap }\OtherTok{:} \OtherTok{∀} \OtherTok{\{}\NormalTok{ℓ κ}\OtherTok{\}} \OtherTok{\{}\NormalTok{A }\OtherTok{:}\NormalTok{ Type ℓ}\OtherTok{\}} \OtherTok{\{}\NormalTok{B }\OtherTok{:}\NormalTok{ Type κ}\OtherTok{\}} \OtherTok{\{}\NormalTok{a a\textquotesingle{} }\OtherTok{:}\NormalTok{ A}\OtherTok{\}}
     \OtherTok{→} \OtherTok{(}\NormalTok{f }\OtherTok{:}\NormalTok{ A }\OtherTok{→}\NormalTok{ B}\OtherTok{)} \OtherTok{→}\NormalTok{ a ≡ a\textquotesingle{} }\OtherTok{→} \OtherTok{(}\NormalTok{f a}\OtherTok{)}\NormalTok{ ≡ }\OtherTok{(}\NormalTok{f a\textquotesingle{}}\OtherTok{)}
\NormalTok{ap f refl }\OtherTok{=}\NormalTok{ refl}
\end{Highlighting}
\end{Shaded}

By the same token, given a proof \texttt{f\ ≡\ g} for two functions
\texttt{f,g\ :\ (x\ :\ A)\ →\ B\ x}, it follows that for any
\texttt{a\ :\ A} we have \texttt{f\ a\ ≡\ g\ a}.

\begin{Shaded}
\begin{Highlighting}[]
\NormalTok{coAp }\OtherTok{:} \OtherTok{∀} \OtherTok{\{}\NormalTok{ℓ κ}\OtherTok{\}} \OtherTok{\{}\NormalTok{A }\OtherTok{:}\NormalTok{ Type ℓ}\OtherTok{\}} \OtherTok{\{}\NormalTok{B }\OtherTok{:}\NormalTok{ Type κ}\OtherTok{\}} \OtherTok{\{}\NormalTok{f g }\OtherTok{:}\NormalTok{ A }\OtherTok{→}\NormalTok{ B}\OtherTok{\}}
       \OtherTok{→}\NormalTok{ f ≡ g }\OtherTok{→} \OtherTok{(}\NormalTok{x }\OtherTok{:}\NormalTok{ A}\OtherTok{)} \OtherTok{→}\NormalTok{ f x ≡ g x}
\NormalTok{coAp refl x }\OtherTok{=}\NormalTok{ refl}
\end{Highlighting}
\end{Shaded}

Moreover, to show that two pairs \texttt{(a\ ,\ b)} and
\texttt{(a\textquotesingle{}\ ,\ b\textquotesingle{})} are equal, it
suffices to show that there is an identification
\texttt{e\ :\ a\ ≡\ a\textquotesingle{}} together with
\texttt{e\textquotesingle{}\ :\ transp\ B\ e\ b\ ≡\ b\textquotesingle{}}.

\begin{Shaded}
\begin{Highlighting}[]
\NormalTok{pairEq }\OtherTok{:} \OtherTok{∀} \OtherTok{\{}\NormalTok{ℓ κ}\OtherTok{\}} \OtherTok{\{}\NormalTok{A }\OtherTok{:}\NormalTok{ Type ℓ}\OtherTok{\}} \OtherTok{\{}\NormalTok{B }\OtherTok{:}\NormalTok{ A }\OtherTok{→}\NormalTok{ Type κ}\OtherTok{\}} 
         \OtherTok{→} \OtherTok{\{}\NormalTok{a a\textquotesingle{} }\OtherTok{:}\NormalTok{ A}\OtherTok{\}} \OtherTok{\{}\NormalTok{b }\OtherTok{:}\NormalTok{ B a}\OtherTok{\}} \OtherTok{\{}\NormalTok{b\textquotesingle{} }\OtherTok{:}\NormalTok{ B a\textquotesingle{}}\OtherTok{\}}
         \OtherTok{→} \OtherTok{(}\NormalTok{e }\OtherTok{:}\NormalTok{ a ≡ a\textquotesingle{}}\OtherTok{)} \OtherTok{(}\NormalTok{e\textquotesingle{} }\OtherTok{:}\NormalTok{ transp B e b ≡ b\textquotesingle{}}\OtherTok{)} 
         \OtherTok{→} \OtherTok{(}\NormalTok{a , b}\OtherTok{)}\NormalTok{ ≡ }\OtherTok{(}\NormalTok{a\textquotesingle{} , b\textquotesingle{}}\OtherTok{)}
\NormalTok{pairEq refl refl }\OtherTok{=}\NormalTok{ refl}
\end{Highlighting}
\end{Shaded}

\subsection{Truncation, Bracket Types \&
Factorization}\label{truncation-bracket-types-factorization}

We say that a type \texttt{A} is:

\begin{enumerate}
\def\labelenumi{\arabic{enumi}.}
\tightlist
\item
  \emph{contractible} (aka (-2)-truncated) if \texttt{A} is uniquely
  inhabited
\item
  a (mere) \emph{proposition} (aka (-1)-truncated) if any two elements
  of \texttt{A} are identical
\item
  a \emph{set} (aka 0-truncated) if for any \texttt{a,b\ :\ A}, the type
  \texttt{a\ ≡\ b} is a proposition.
\end{enumerate}

\begin{Shaded}
\begin{Highlighting}[]
\NormalTok{isContr }\OtherTok{:} \OtherTok{∀} \OtherTok{\{}\NormalTok{ℓ}\OtherTok{\}} \OtherTok{→}\NormalTok{ Type ℓ }\OtherTok{→}\NormalTok{ Type ℓ}
\NormalTok{isContr A }\OtherTok{=}\NormalTok{ Σ A }\OtherTok{(λ}\NormalTok{ a }\OtherTok{→} \OtherTok{(}\NormalTok{b }\OtherTok{:}\NormalTok{ A}\OtherTok{)} \OtherTok{→}\NormalTok{ a ≡ b}\OtherTok{)}

\NormalTok{isProp }\OtherTok{:} \OtherTok{∀} \OtherTok{\{}\NormalTok{ℓ}\OtherTok{\}} \OtherTok{→}\NormalTok{ Type ℓ }\OtherTok{→}\NormalTok{ Type ℓ}
\NormalTok{isProp A }\OtherTok{=} \OtherTok{\{}\NormalTok{a b }\OtherTok{:}\NormalTok{ A}\OtherTok{\}} \OtherTok{→}\NormalTok{ a ≡ b}

\NormalTok{isSet }\OtherTok{:} \OtherTok{∀} \OtherTok{\{}\NormalTok{ℓ}\OtherTok{\}} \OtherTok{→}\NormalTok{ Type ℓ }\OtherTok{→}\NormalTok{ Type ℓ}
\NormalTok{isSet A }\OtherTok{=} \OtherTok{(}\NormalTok{a b }\OtherTok{:}\NormalTok{ A}\OtherTok{)} \OtherTok{→}\NormalTok{ isProp }\OtherTok{(}\NormalTok{a ≡ b}\OtherTok{)}
\end{Highlighting}
\end{Shaded}

We additionally postulate the existence of a \emph{propositional
truncation,} or \emph{bracket type} operation, that takes a type
\texttt{A} to the least proposition (with respect to entailment) entailed by
inhabitation of \texttt{A}.

\begin{Shaded}
\begin{Highlighting}[]
\KeywordTok{postulate}
\NormalTok{    ∥}\OtherTok{\_}\NormalTok{∥ }\OtherTok{:} \OtherTok{∀} \OtherTok{\{}\NormalTok{ℓ}\OtherTok{\}} \OtherTok{(}\NormalTok{A }\OtherTok{:}\NormalTok{ Type ℓ}\OtherTok{)} \OtherTok{→}\NormalTok{ Type lzero}
\NormalTok{    in∥{-}∥ }\OtherTok{:} \OtherTok{∀} \OtherTok{\{}\NormalTok{ℓ}\OtherTok{\}} \OtherTok{\{}\NormalTok{A }\OtherTok{:}\NormalTok{ Type ℓ}\OtherTok{\}} \OtherTok{→}\NormalTok{ A }\OtherTok{→}\NormalTok{ ∥ A ∥}
\NormalTok{    ∥{-}∥IsProp }\OtherTok{:} \OtherTok{∀} \OtherTok{\{}\NormalTok{ℓ}\OtherTok{\}} \OtherTok{\{}\NormalTok{A }\OtherTok{:}\NormalTok{ Type ℓ}\OtherTok{\}} \OtherTok{→}\NormalTok{ isProp }\OtherTok{(}\NormalTok{∥ A ∥}\OtherTok{)}
\NormalTok{    ∥{-}∥Rec }\OtherTok{:} \OtherTok{∀} \OtherTok{\{}\NormalTok{ℓ κ}\OtherTok{\}} \OtherTok{\{}\NormalTok{A }\OtherTok{:}\NormalTok{ Type ℓ}\OtherTok{\}} \OtherTok{\{}\NormalTok{B }\OtherTok{:}\NormalTok{ Type κ}\OtherTok{\}}
            \OtherTok{→}\NormalTok{ isProp B }\OtherTok{→} \OtherTok{(}\NormalTok{A }\OtherTok{→}\NormalTok{ B}\OtherTok{)} \OtherTok{→}\NormalTok{ ∥ A ∥ }\OtherTok{→}\NormalTok{ B}
\end{Highlighting}
\end{Shaded}
When the type \texttt{∥\ A\ ∥} is inhabited, we say that \texttt{A} is
\emph{merely} inhabited. From this operation on types, we straightforwardly obtain the higher
analogue of the usual epi-mono factorization system on functions between
sets, as follows:

Given a function \texttt{f\ :\ A\ →\ B} and an element \texttt{b\ :\ B},
the \emph{fibre} of \texttt{f} at \texttt{b} is the type of elements of
\texttt{a} equipped with a proof of \texttt{f\ a\ ≡\ b}:

\begin{Shaded}
\begin{Highlighting}[]
\NormalTok{Fibre }\OtherTok{:} \OtherTok{∀} \OtherTok{\{}\NormalTok{ℓ κ}\OtherTok{\}} \OtherTok{\{}\NormalTok{A }\OtherTok{:}\NormalTok{ Type ℓ}\OtherTok{\}} \OtherTok{\{}\NormalTok{B }\OtherTok{:}\NormalTok{ Type κ}\OtherTok{\}} \OtherTok{→} \OtherTok{(}\NormalTok{A }\OtherTok{→}\NormalTok{ B}\OtherTok{)} \OtherTok{→}\NormalTok{ B }\OtherTok{→}\NormalTok{ Type }\OtherTok{(}\NormalTok{ℓ ⊔ κ}\OtherTok{)}
\NormalTok{Fibre }\OtherTok{\{}\NormalTok{A }\OtherTok{=}\NormalTok{ A}\OtherTok{\}}\NormalTok{ f b }\OtherTok{=}\NormalTok{ Σ A }\OtherTok{(λ}\NormalTok{ a }\OtherTok{→}\NormalTok{ f a ≡ b}\OtherTok{)}
\end{Highlighting}
\end{Shaded}

By the same token, given such an \texttt{f}, its \emph{image} is the
type of elements of \texttt{B} whose fibres are merely inhabited.

\begin{Shaded}
\begin{Highlighting}[]
\NormalTok{Im }\OtherTok{:} \OtherTok{∀} \OtherTok{\{}\NormalTok{ℓ κ}\OtherTok{\}} \OtherTok{\{}\NormalTok{A }\OtherTok{:}\NormalTok{ Type ℓ}\OtherTok{\}} \OtherTok{\{}\NormalTok{B }\OtherTok{:}\NormalTok{ Type κ}\OtherTok{\}} \OtherTok{→} \OtherTok{(}\NormalTok{A }\OtherTok{→}\NormalTok{ B}\OtherTok{)} \OtherTok{→}\NormalTok{ Type κ}
\NormalTok{Im }\OtherTok{\{}\NormalTok{B }\OtherTok{=}\NormalTok{ B}\OtherTok{\}}\NormalTok{ f }\OtherTok{=}\NormalTok{ Σ B }\OtherTok{(λ}\NormalTok{ b }\OtherTok{→}\NormalTok{ ∥ Fibre f b ∥}\OtherTok{)}
\end{Highlighting}
\end{Shaded}

We say that \texttt{f} is \emph{(-1)-truncated} (i.e.~a monomorphism),
if for each \texttt{b\ :\ B}, the fibre of \texttt{f} at \texttt{b} is a
proposition.

\begin{Shaded}
\begin{Highlighting}[]
\NormalTok{isMono }\OtherTok{:} \OtherTok{∀} \OtherTok{\{}\NormalTok{ℓ κ}\OtherTok{\}} \OtherTok{\{}\NormalTok{A }\OtherTok{:}\NormalTok{ Type ℓ}\OtherTok{\}} \OtherTok{\{}\NormalTok{B }\OtherTok{:}\NormalTok{ Type κ}\OtherTok{\}} \OtherTok{→} \OtherTok{(}\NormalTok{A }\OtherTok{→}\NormalTok{ B}\OtherTok{)} \OtherTok{→}\NormalTok{ Type }\OtherTok{(}\NormalTok{ℓ ⊔ κ}\OtherTok{)}
\NormalTok{isMono }\OtherTok{\{}\NormalTok{B }\OtherTok{=}\NormalTok{ B}\OtherTok{\}}\NormalTok{ f }\OtherTok{=} \OtherTok{(}\NormalTok{b }\OtherTok{:}\NormalTok{ B}\OtherTok{)} \OtherTok{→}\NormalTok{ isProp }\OtherTok{(}\NormalTok{Fibre f b}\OtherTok{)}
\end{Highlighting}
\end{Shaded}

Likewise, we say that \texttt{f} is \emph{(-1)-connected} (i.e.~an
epimorphism), if all of its fibres are merely inhabited.

\begin{Shaded}
\begin{Highlighting}[]
\NormalTok{isEpi }\OtherTok{:} \OtherTok{∀} \OtherTok{\{}\NormalTok{ℓ κ}\OtherTok{\}} \OtherTok{\{}\NormalTok{A }\OtherTok{:}\NormalTok{ Type ℓ}\OtherTok{\}} \OtherTok{\{}\NormalTok{B }\OtherTok{:}\NormalTok{ Type κ}\OtherTok{\}} \OtherTok{→} \OtherTok{(}\NormalTok{A }\OtherTok{→}\NormalTok{ B}\OtherTok{)} \OtherTok{→}\NormalTok{ Type κ}
\NormalTok{isEpi }\OtherTok{\{}\NormalTok{B }\OtherTok{=}\NormalTok{ B}\OtherTok{\}}\NormalTok{ f }\OtherTok{=} \OtherTok{(}\NormalTok{b }\OtherTok{:}\NormalTok{ B}\OtherTok{)} \OtherTok{→}\NormalTok{ ∥ Fibre f b ∥}
\end{Highlighting}
\end{Shaded}

Every function \texttt{f} can then be factored into a (-1)-connected map
onto its image followed by a (-1)-truncated map onto its codomain:

\begin{Shaded}
\begin{Highlighting}[]
\NormalTok{factor1 }\OtherTok{:} \OtherTok{∀} \OtherTok{\{}\NormalTok{ℓ κ}\OtherTok{\}} \OtherTok{\{}\NormalTok{A }\OtherTok{:}\NormalTok{ Type ℓ}\OtherTok{\}} \OtherTok{\{}\NormalTok{B }\OtherTok{:}\NormalTok{ Type κ}\OtherTok{\}} \OtherTok{→} \OtherTok{(}\NormalTok{f }\OtherTok{:}\NormalTok{ A }\OtherTok{→}\NormalTok{ B}\OtherTok{)} \OtherTok{→}\NormalTok{ A }\OtherTok{→}\NormalTok{ Im f}
\NormalTok{factor1 f a }\OtherTok{=} \OtherTok{(}\NormalTok{f a}\OtherTok{)}\NormalTok{ , in∥{-}∥ }\OtherTok{(}\NormalTok{a , refl}\OtherTok{)}

\NormalTok{factor2 }\OtherTok{:} \OtherTok{∀} \OtherTok{\{}\NormalTok{ℓ κ}\OtherTok{\}} \OtherTok{\{}\NormalTok{A }\OtherTok{:}\NormalTok{ Type ℓ}\OtherTok{\}} \OtherTok{\{}\NormalTok{B }\OtherTok{:}\NormalTok{ Type κ}\OtherTok{\}} \OtherTok{→} \OtherTok{(}\NormalTok{f }\OtherTok{:}\NormalTok{ A }\OtherTok{→}\NormalTok{ B}\OtherTok{)} \OtherTok{→}\NormalTok{ Im f }\OtherTok{→}\NormalTok{ B}
\NormalTok{factor2 f }\OtherTok{(}\NormalTok{b , }\OtherTok{\_)} \OtherTok{=}\NormalTok{ b}

\NormalTok{factor≡ }\OtherTok{:} \OtherTok{∀} \OtherTok{\{}\NormalTok{ℓ κ}\OtherTok{\}} \OtherTok{\{}\NormalTok{A }\OtherTok{:}\NormalTok{ Type ℓ}\OtherTok{\}} \OtherTok{\{}\NormalTok{B }\OtherTok{:}\NormalTok{ Type κ}\OtherTok{\}} 
          \OtherTok{→} \OtherTok{(}\NormalTok{f }\OtherTok{:}\NormalTok{ A }\OtherTok{→}\NormalTok{ B}\OtherTok{)} \OtherTok{(}\NormalTok{a }\OtherTok{:}\NormalTok{ A}\OtherTok{)} \OtherTok{→}\NormalTok{ factor2 f }\OtherTok{(}\NormalTok{factor1 f a}\OtherTok{)}\NormalTok{ ≡ f a}
\NormalTok{factor≡ f a }\OtherTok{=}\NormalTok{ refl}

\NormalTok{factor1IsEpi }\OtherTok{:} \OtherTok{∀} \OtherTok{\{}\NormalTok{ℓ κ}\OtherTok{\}} \OtherTok{\{}\NormalTok{A }\OtherTok{:}\NormalTok{ Type ℓ}\OtherTok{\}} \OtherTok{\{}\NormalTok{B }\OtherTok{:}\NormalTok{ Type κ}\OtherTok{\}}
               \OtherTok{→} \OtherTok{(}\NormalTok{f }\OtherTok{:}\NormalTok{ A }\OtherTok{→}\NormalTok{ B}\OtherTok{)} \OtherTok{→}\NormalTok{ isEpi }\OtherTok{(}\NormalTok{factor1 f}\OtherTok{)}
\NormalTok{factor1IsEpi f }\OtherTok{(}\NormalTok{b , x}\OtherTok{)} \OtherTok{=} 
\NormalTok{    ∥{-}∥Rec ∥{-}∥IsProp }
          \OtherTok{(λ} \OtherTok{\{(}\NormalTok{a , refl}\OtherTok{)} \OtherTok{→}\NormalTok{ in∥{-}∥ }\OtherTok{(}\NormalTok{a , pairEq refl ∥{-}∥IsProp}\OtherTok{)\})} 
\NormalTok{          x}

\NormalTok{factor2IsMono }\OtherTok{:} \OtherTok{∀} \OtherTok{\{}\NormalTok{ℓ κ}\OtherTok{\}} \OtherTok{\{}\NormalTok{A }\OtherTok{:}\NormalTok{ Type ℓ}\OtherTok{\}} \OtherTok{\{}\NormalTok{B }\OtherTok{:}\NormalTok{ Type κ}\OtherTok{\}}
                \OtherTok{→} \OtherTok{(}\NormalTok{f }\OtherTok{:}\NormalTok{ A }\OtherTok{→}\NormalTok{ B}\OtherTok{)} \OtherTok{→}\NormalTok{ isMono }\OtherTok{(}\NormalTok{factor2 f}\OtherTok{)}
\NormalTok{factor2IsMono f b }\OtherTok{\{}\NormalTok{a }\OtherTok{=} \OtherTok{((}\NormalTok{b\textquotesingle{} , x}\OtherTok{)}\NormalTok{ , refl}\OtherTok{)\}} \OtherTok{\{}\NormalTok{b }\OtherTok{=} \OtherTok{((}\NormalTok{b\textquotesingle{}\textquotesingle{} , x\textquotesingle{}}\OtherTok{)}\NormalTok{ , refl}\OtherTok{)\}} \OtherTok{=} 
\NormalTok{    pairEq }\OtherTok{(}\NormalTok{pairEq refl ∥{-}∥IsProp}\OtherTok{)} \OtherTok{(}\NormalTok{lemma ∥{-}∥IsProp}\OtherTok{)}
    \KeywordTok{where}\NormalTok{ lemma }\OtherTok{:} \OtherTok{(}\NormalTok{e }\OtherTok{:}\NormalTok{ x ≡ x\textquotesingle{}}\OtherTok{)} \OtherTok{→}\NormalTok{ transp }\OtherTok{(λ}\NormalTok{ a }\OtherTok{→}\NormalTok{ factor2 f a ≡ b}\OtherTok{)} 
                                        \OtherTok{(}\NormalTok{pairEq refl e}\OtherTok{)}\NormalTok{ refl ≡ refl}
\NormalTok{          lemma refl }\OtherTok{=}\NormalTok{ refl}
\end{Highlighting}
\end{Shaded}

\subsection{Equivalences}\label{equivalences}

A pivotal notion, both for HoTT in general, and for the content of this
paper, is that of a function \texttt{f\ :\ A\ →\ B} being an
\emph{equivalence} of types. The reader familiar with HoTT will know
that there are several definitions -- all equivalent -- of this concept
appearing in the HoTT literature. For present purposes, we make use of
the \emph{bi-invertible maps} notion of equivalence. Hence we say that a
function \texttt{f\ :\ A\ →\ B} is an equivalence if it has both a left
inverse and a right inverse:

\begin{Shaded}
\begin{Highlighting}[]
\NormalTok{isEquiv }\OtherTok{:} \OtherTok{∀} \OtherTok{\{}\NormalTok{ℓ κ}\OtherTok{\}} \OtherTok{\{}\NormalTok{A }\OtherTok{:}\NormalTok{ Type ℓ}\OtherTok{\}} \OtherTok{\{}\NormalTok{B }\OtherTok{:}\NormalTok{ Type κ}\OtherTok{\}} \OtherTok{→} \OtherTok{(}\NormalTok{A }\OtherTok{→}\NormalTok{ B}\OtherTok{)} \OtherTok{→}\NormalTok{ Type }\OtherTok{(}\NormalTok{ℓ ⊔ κ}\OtherTok{)}
\NormalTok{isEquiv }\OtherTok{\{}\NormalTok{A }\OtherTok{=}\NormalTok{ A}\OtherTok{\}} \OtherTok{\{}\NormalTok{B }\OtherTok{=}\NormalTok{ B}\OtherTok{\}}\NormalTok{ f }\OtherTok{=}
      \OtherTok{(}\NormalTok{Σ }\OtherTok{(}\NormalTok{B }\OtherTok{→}\NormalTok{ A}\OtherTok{)} \OtherTok{(λ}\NormalTok{ g }\OtherTok{→} \OtherTok{(}\NormalTok{a }\OtherTok{:}\NormalTok{ A}\OtherTok{)} \OtherTok{→}\NormalTok{ g }\OtherTok{(}\NormalTok{f a}\OtherTok{)}\NormalTok{ ≡ a}\OtherTok{))} 
\NormalTok{    × }\OtherTok{(}\NormalTok{Σ }\OtherTok{(}\NormalTok{B }\OtherTok{→}\NormalTok{ A}\OtherTok{)} \OtherTok{(λ}\NormalTok{ h }\OtherTok{→} \OtherTok{(}\NormalTok{b }\OtherTok{:}\NormalTok{ B}\OtherTok{)} \OtherTok{→}\NormalTok{ f }\OtherTok{(}\NormalTok{h b}\OtherTok{)}\NormalTok{ ≡ b}\OtherTok{))}
\end{Highlighting}
\end{Shaded}

A closely-related notion is that of a function \texttt{f} being an
\emph{isomorphism}, i.e.~having a single two-sided inverse:

\begin{Shaded}
\begin{Highlighting}[]
\NormalTok{Iso }\OtherTok{:} \OtherTok{∀} \OtherTok{\{}\NormalTok{ℓ κ}\OtherTok{\}} \OtherTok{\{}\NormalTok{A }\OtherTok{:}\NormalTok{ Type ℓ}\OtherTok{\}} \OtherTok{\{}\NormalTok{B }\OtherTok{:}\NormalTok{ Type κ}\OtherTok{\}} \OtherTok{→} \OtherTok{(}\NormalTok{A }\OtherTok{→}\NormalTok{ B}\OtherTok{)} \OtherTok{→}\NormalTok{ Type }\OtherTok{(}\NormalTok{ℓ ⊔ κ}\OtherTok{)}
\NormalTok{Iso }\OtherTok{\{}\NormalTok{A }\OtherTok{=}\NormalTok{ A}\OtherTok{\}} \OtherTok{\{}\NormalTok{B }\OtherTok{=}\NormalTok{ B}\OtherTok{\}}\NormalTok{ f }\OtherTok{=}
    \OtherTok{(}\NormalTok{Σ }\OtherTok{(}\NormalTok{B }\OtherTok{→}\NormalTok{ A}\OtherTok{)} \OtherTok{(λ}\NormalTok{ g }\OtherTok{→} \OtherTok{((}\NormalTok{a }\OtherTok{:}\NormalTok{ A}\OtherTok{)} \OtherTok{→}\NormalTok{ g }\OtherTok{(}\NormalTok{f a}\OtherTok{)}\NormalTok{ ≡ a}\OtherTok{)} 
\NormalTok{                    × }\OtherTok{((}\NormalTok{b }\OtherTok{:}\NormalTok{ B}\OtherTok{)} \OtherTok{→}\NormalTok{ f }\OtherTok{(}\NormalTok{g b}\OtherTok{)}\NormalTok{ ≡ b}\OtherTok{)))}
\end{Highlighting}
\end{Shaded}

One might be inclined to wonder, then, why we bother to define
equivalence via the seemingly more complicated notion of having both a
left and a right inverse when the familiar notion of isomorphism can
just as well be defined, as above. The full reasons for this are beyond
the scope of this paper, though see \cite{hottbook} for further
discussion. Suffice it to say that, for subtle reasons due to the
higher-categorical structure of types in HoTT, the plain notion of
isomorphism given above is not a \emph{good} notion of equivalence,
whereas that of bi-invertible maps is. In particular, the type
\texttt{Iso\ f} is not necessarily a proposition for arbitrary
\texttt{f}, whereas \texttt{isEquiv\ f} is.

We may nonetheless move more-or-less freely back and forth between the
notions of equivalence and isomorphism given above, thanks to the
following functions, which allow us to convert isomorphisms to
equivalences and vice versa:

\begin{Shaded}
\begin{Highlighting}[]
\NormalTok{Iso→isEquiv }\OtherTok{:} \OtherTok{∀} \OtherTok{\{}\NormalTok{ℓ κ}\OtherTok{\}} \OtherTok{\{}\NormalTok{A }\OtherTok{:}\NormalTok{ Type ℓ}\OtherTok{\}} \OtherTok{\{}\NormalTok{B }\OtherTok{:}\NormalTok{ Type κ}\OtherTok{\}} \OtherTok{\{}\NormalTok{f }\OtherTok{:}\NormalTok{ A }\OtherTok{→}\NormalTok{ B}\OtherTok{\}} 
              \OtherTok{→}\NormalTok{ Iso f }\OtherTok{→}\NormalTok{ isEquiv f}
\NormalTok{Iso→isEquiv }\OtherTok{(}\NormalTok{g , l , r}\OtherTok{)} \OtherTok{=} \OtherTok{((}\NormalTok{g , l}\OtherTok{)}\NormalTok{ , }\OtherTok{(}\NormalTok{g , r}\OtherTok{))}

\NormalTok{isEquiv→Iso }\OtherTok{:} \OtherTok{∀} \OtherTok{\{}\NormalTok{ℓ κ}\OtherTok{\}} \OtherTok{\{}\NormalTok{A }\OtherTok{:}\NormalTok{ Type ℓ}\OtherTok{\}} \OtherTok{\{}\NormalTok{B }\OtherTok{:}\NormalTok{ Type κ}\OtherTok{\}} \OtherTok{\{}\NormalTok{f }\OtherTok{:}\NormalTok{ A }\OtherTok{→}\NormalTok{ B}\OtherTok{\}} 
              \OtherTok{→}\NormalTok{ isEquiv f }\OtherTok{→}\NormalTok{ Iso f}
\NormalTok{isEquiv→Iso }\OtherTok{\{}\NormalTok{f }\OtherTok{=}\NormalTok{ f}\OtherTok{\}} \OtherTok{((}\NormalTok{g , l}\OtherTok{)}\NormalTok{ , }\OtherTok{(}\NormalTok{h , r}\OtherTok{))} \OtherTok{=} 
\NormalTok{    h , }\OtherTok{(λ}\NormalTok{ x }\OtherTok{→} \OtherTok{(}\NormalTok{h }\OtherTok{(}\NormalTok{f x}\OtherTok{))}\NormalTok{        ≡〈 sym }\OtherTok{(}\NormalTok{l }\OtherTok{(}\NormalTok{h }\OtherTok{(}\NormalTok{f x}\OtherTok{)))}\NormalTok{ 〉 }
               \OtherTok{(}\NormalTok{g }\OtherTok{(}\NormalTok{f }\OtherTok{(}\NormalTok{h }\OtherTok{(}\NormalTok{f x}\OtherTok{)))}\NormalTok{ ≡〈 ap g }\OtherTok{(}\NormalTok{r }\OtherTok{(}\NormalTok{f x}\OtherTok{))}\NormalTok{ 〉}
               \OtherTok{((}\NormalTok{g }\OtherTok{(}\NormalTok{f x}\OtherTok{))}\NormalTok{       ≡〈 l x 〉 }
               \OtherTok{(}\NormalTok{x □}\OtherTok{))))}\NormalTok{ , r}
\end{Highlighting}
\end{Shaded}

Straightforwardly, the identity function at each type is an equivalence,
and equivalences are closed under composition:

\begin{Shaded}
\begin{Highlighting}[]
\NormalTok{idIsEquiv }\OtherTok{:} \OtherTok{∀} \OtherTok{\{}\NormalTok{ℓ}\OtherTok{\}} \OtherTok{\{}\NormalTok{A }\OtherTok{:}\NormalTok{ Type ℓ}\OtherTok{\}} \OtherTok{→}\NormalTok{ isEquiv }\OtherTok{\{}\NormalTok{A }\OtherTok{=}\NormalTok{ A}\OtherTok{\}} \OtherTok{(λ}\NormalTok{ x }\OtherTok{→}\NormalTok{ x}\OtherTok{)}
\NormalTok{idIsEquiv }\OtherTok{=} \OtherTok{((λ}\NormalTok{ x }\OtherTok{→}\NormalTok{ x}\OtherTok{)}\NormalTok{ , }\OtherTok{(λ}\NormalTok{ x }\OtherTok{→}\NormalTok{ refl}\OtherTok{))}\NormalTok{ , }\OtherTok{((λ}\NormalTok{ x }\OtherTok{→}\NormalTok{ x}\OtherTok{)}\NormalTok{ , }\OtherTok{(λ}\NormalTok{ x }\OtherTok{→}\NormalTok{ refl}\OtherTok{))}

\NormalTok{compIsEquiv }\OtherTok{:} \OtherTok{∀} \OtherTok{\{}\NormalTok{ℓ ℓ\textquotesingle{} ℓ\textquotesingle{}\textquotesingle{}}\OtherTok{\}} \OtherTok{\{}\NormalTok{A }\OtherTok{:}\NormalTok{ Type ℓ}\OtherTok{\}} \OtherTok{\{}\NormalTok{B }\OtherTok{:}\NormalTok{ Type ℓ\textquotesingle{}}\OtherTok{\}} \OtherTok{\{}\NormalTok{C }\OtherTok{:}\NormalTok{ Type ℓ\textquotesingle{}\textquotesingle{}}\OtherTok{\}}
              \OtherTok{→} \OtherTok{(}\NormalTok{g }\OtherTok{:}\NormalTok{ B }\OtherTok{→}\NormalTok{ C}\OtherTok{)} \OtherTok{(}\NormalTok{f }\OtherTok{:}\NormalTok{ A }\OtherTok{→}\NormalTok{ B}\OtherTok{)} \OtherTok{→}\NormalTok{ isEquiv g }\OtherTok{→}\NormalTok{ isEquiv f}
              \OtherTok{→}\NormalTok{ isEquiv }\OtherTok{(λ}\NormalTok{ a }\OtherTok{→}\NormalTok{ g }\OtherTok{(}\NormalTok{f a}\OtherTok{))}
\NormalTok{compIsEquiv g f }\OtherTok{((}\NormalTok{g\textquotesingle{} , lg}\OtherTok{)}\NormalTok{ , }\OtherTok{(}\NormalTok{g\textquotesingle{}\textquotesingle{} , rg}\OtherTok{))} \OtherTok{((}\NormalTok{f\textquotesingle{} , lf}\OtherTok{)}\NormalTok{ , }\OtherTok{(}\NormalTok{f\textquotesingle{}\textquotesingle{} , rf}\OtherTok{))} \OtherTok{=} 
      \OtherTok{(} \OtherTok{(λ}\NormalTok{ c }\OtherTok{→}\NormalTok{ f\textquotesingle{} }\OtherTok{(}\NormalTok{g\textquotesingle{} c}\OtherTok{))}   
\NormalTok{      , }\OtherTok{λ}\NormalTok{ a }\OtherTok{→} \OtherTok{(}\NormalTok{f\textquotesingle{} }\OtherTok{(}\NormalTok{g\textquotesingle{} }\OtherTok{(}\NormalTok{g }\OtherTok{(}\NormalTok{f a}\OtherTok{))))}\NormalTok{   ≡〈 ap f\textquotesingle{} }\OtherTok{(}\NormalTok{lg }\OtherTok{(}\NormalTok{f a}\OtherTok{))}\NormalTok{ 〉 }
              \OtherTok{(}\NormalTok{f\textquotesingle{} }\OtherTok{(}\NormalTok{f a}\OtherTok{)}\NormalTok{             ≡〈 lf a 〉 }
              \OtherTok{(}\NormalTok{a                    □}\OtherTok{)))} 
\NormalTok{    , }\OtherTok{((λ}\NormalTok{ c }\OtherTok{→}\NormalTok{ f\textquotesingle{}\textquotesingle{} }\OtherTok{(}\NormalTok{g\textquotesingle{}\textquotesingle{} c}\OtherTok{))} 
\NormalTok{      , }\OtherTok{λ}\NormalTok{ c }\OtherTok{→} \OtherTok{(}\NormalTok{g }\OtherTok{(}\NormalTok{f }\OtherTok{(}\NormalTok{f\textquotesingle{}\textquotesingle{} }\OtherTok{(}\NormalTok{g\textquotesingle{}\textquotesingle{} c}\OtherTok{))))}\NormalTok{ ≡〈 ap g  }\OtherTok{(}\NormalTok{rf }\OtherTok{(}\NormalTok{g\textquotesingle{}\textquotesingle{} c}\OtherTok{))}\NormalTok{ 〉 }
              \OtherTok{(}\NormalTok{g }\OtherTok{(}\NormalTok{g\textquotesingle{}\textquotesingle{} c}\OtherTok{)}\NormalTok{            ≡〈 rg c 〉}
              \OtherTok{(}\NormalTok{c                    □}\OtherTok{)))}
\end{Highlighting}
\end{Shaded}

And by the above translation between equivalences and isomorphisms, each
equivalence has a corresponding inverse map in the opposite direction,
which is itself an equivalence:

\begin{Shaded}
\begin{Highlighting}[]
\NormalTok{inv }\OtherTok{:} \OtherTok{∀} \OtherTok{\{}\NormalTok{ℓ κ}\OtherTok{\}} \OtherTok{\{}\NormalTok{A }\OtherTok{:}\NormalTok{ Type ℓ}\OtherTok{\}} \OtherTok{\{}\NormalTok{B }\OtherTok{:}\NormalTok{ Type κ}\OtherTok{\}} \OtherTok{\{}\NormalTok{f }\OtherTok{:}\NormalTok{ A }\OtherTok{→}\NormalTok{ B}\OtherTok{\}} \OtherTok{→}\NormalTok{ isEquiv f }\OtherTok{→}\NormalTok{ B }\OtherTok{→}\NormalTok{ A}
\NormalTok{inv }\OtherTok{(\_}\NormalTok{ , }\OtherTok{(}\NormalTok{h , }\OtherTok{\_))} \OtherTok{=}\NormalTok{ h}

\NormalTok{isoInv }\OtherTok{:} \OtherTok{∀} \OtherTok{\{}\NormalTok{ℓ κ}\OtherTok{\}} \OtherTok{\{}\NormalTok{A }\OtherTok{:}\NormalTok{ Type ℓ}\OtherTok{\}} \OtherTok{\{}\NormalTok{B }\OtherTok{:}\NormalTok{ Type κ}\OtherTok{\}} \OtherTok{\{}\NormalTok{f }\OtherTok{:}\NormalTok{ A }\OtherTok{→}\NormalTok{ B}\OtherTok{\}}
         \OtherTok{→} \OtherTok{(}\NormalTok{isof }\OtherTok{:}\NormalTok{ Iso f}\OtherTok{)} \OtherTok{→}\NormalTok{ Iso }\OtherTok{(}\NormalTok{fst isof}\OtherTok{)}
\NormalTok{isoInv }\OtherTok{\{}\NormalTok{f }\OtherTok{=}\NormalTok{ f}\OtherTok{\}} \OtherTok{(}\NormalTok{g , l , r}\OtherTok{)} \OtherTok{=} \OtherTok{(}\NormalTok{f , r , l}\OtherTok{)}

\NormalTok{invIsEquiv }\OtherTok{:} \OtherTok{∀} \OtherTok{\{}\NormalTok{ℓ κ}\OtherTok{\}} \OtherTok{\{}\NormalTok{A }\OtherTok{:}\NormalTok{ Type ℓ}\OtherTok{\}} \OtherTok{\{}\NormalTok{B }\OtherTok{:}\NormalTok{ Type κ}\OtherTok{\}} \OtherTok{\{}\NormalTok{f }\OtherTok{:}\NormalTok{ A }\OtherTok{→}\NormalTok{ B}\OtherTok{\}}
             \OtherTok{→} \OtherTok{(}\NormalTok{ef }\OtherTok{:}\NormalTok{ isEquiv f}\OtherTok{)} \OtherTok{→}\NormalTok{ isEquiv }\OtherTok{(}\NormalTok{inv ef}\OtherTok{)}
\NormalTok{invIsEquiv ef }\OtherTok{=}\NormalTok{ Iso→isEquiv }\OtherTok{(}\NormalTok{isoInv }\OtherTok{(}\NormalTok{isEquiv→Iso ef}\OtherTok{))}
\end{Highlighting}
\end{Shaded}

We close this section by noting that, for each type family
\texttt{B\ :\ A\ →\ Type}, the map
\texttt{B\ a\ →\ B\ a\textquotesingle{}} induced by transport along
\texttt{e\ :\ a\ ≡\ a\textquotesingle{}} for any
\texttt{a,\ a\textquotesingle{}\ :\ A} is an equivalence with inverse
given by transport along \texttt{sym\ e}, as follows:

\begin{Shaded}
\begin{Highlighting}[]
\NormalTok{syml }\OtherTok{:} \OtherTok{∀} \OtherTok{\{}\NormalTok{ℓ κ}\OtherTok{\}} \OtherTok{\{}\NormalTok{A }\OtherTok{:}\NormalTok{ Type ℓ}\OtherTok{\}} \OtherTok{\{}\NormalTok{B }\OtherTok{:}\NormalTok{ A }\OtherTok{→}\NormalTok{ Type κ}\OtherTok{\}} \OtherTok{\{}\NormalTok{a b }\OtherTok{:}\NormalTok{ A}\OtherTok{\}}
       \OtherTok{→} \OtherTok{(}\NormalTok{e }\OtherTok{:}\NormalTok{ a ≡ b}\OtherTok{)} \OtherTok{(}\NormalTok{x }\OtherTok{:}\NormalTok{ B a}\OtherTok{)} \OtherTok{→}\NormalTok{ transp B }\OtherTok{(}\NormalTok{sym e}\OtherTok{)} \OtherTok{(}\NormalTok{transp B e x}\OtherTok{)}\NormalTok{ ≡ x}
\NormalTok{syml refl x }\OtherTok{=}\NormalTok{ refl}

\NormalTok{symr }\OtherTok{:} \OtherTok{∀} \OtherTok{\{}\NormalTok{ℓ κ}\OtherTok{\}} \OtherTok{\{}\NormalTok{A }\OtherTok{:}\NormalTok{ Type ℓ}\OtherTok{\}} \OtherTok{\{}\NormalTok{B }\OtherTok{:}\NormalTok{ A }\OtherTok{→}\NormalTok{ Type κ}\OtherTok{\}} \OtherTok{\{}\NormalTok{a b }\OtherTok{:}\NormalTok{ A}\OtherTok{\}}
       \OtherTok{→} \OtherTok{(}\NormalTok{e }\OtherTok{:}\NormalTok{ a ≡ b}\OtherTok{)} \OtherTok{(}\NormalTok{y }\OtherTok{:}\NormalTok{ B b}\OtherTok{)} \OtherTok{→}\NormalTok{ transp B e }\OtherTok{(}\NormalTok{transp B }\OtherTok{(}\NormalTok{sym e}\OtherTok{)}\NormalTok{ y}\OtherTok{)}\NormalTok{ ≡ y}
\NormalTok{symr refl x }\OtherTok{=}\NormalTok{ refl}
\end{Highlighting}
\end{Shaded}

And\ldots{}\dnote{You probably plan to say more here? It's a lot of stuff...}

\begin{Shaded}
\begin{Highlighting}[]
\NormalTok{transpAp }\OtherTok{:} \OtherTok{∀} \OtherTok{\{}\NormalTok{ℓ ℓ\textquotesingle{} κ}\OtherTok{\}} \OtherTok{\{}\NormalTok{A }\OtherTok{:}\NormalTok{ Type ℓ}\OtherTok{\}} \OtherTok{\{}\NormalTok{A\textquotesingle{} }\OtherTok{:}\NormalTok{ Type ℓ\textquotesingle{}}\OtherTok{\}} \OtherTok{\{}\NormalTok{a b }\OtherTok{:}\NormalTok{ A}\OtherTok{\}}
           \OtherTok{→} \OtherTok{(}\NormalTok{B }\OtherTok{:}\NormalTok{ A\textquotesingle{} }\OtherTok{→}\NormalTok{ Type κ}\OtherTok{)} \OtherTok{(}\NormalTok{f }\OtherTok{:}\NormalTok{ A }\OtherTok{→}\NormalTok{ A\textquotesingle{}}\OtherTok{)} \OtherTok{(}\NormalTok{e }\OtherTok{:}\NormalTok{ a ≡ b}\OtherTok{)} \OtherTok{(}\NormalTok{x }\OtherTok{:}\NormalTok{ B }\OtherTok{(}\NormalTok{f a}\OtherTok{))}
           \OtherTok{→}\NormalTok{ transp }\OtherTok{(λ}\NormalTok{ x }\OtherTok{→}\NormalTok{ B }\OtherTok{(}\NormalTok{f x}\OtherTok{))}\NormalTok{ e x ≡ transp B }\OtherTok{(}\NormalTok{ap f e}\OtherTok{)}\NormalTok{ x}
\NormalTok{transpAp B f refl x }\OtherTok{=}\NormalTok{ refl}

\NormalTok{≡siml }\OtherTok{:} \OtherTok{∀} \OtherTok{\{}\NormalTok{ℓ}\OtherTok{\}} \OtherTok{\{}\NormalTok{A }\OtherTok{:}\NormalTok{ Type ℓ}\OtherTok{\}} \OtherTok{\{}\NormalTok{a b }\OtherTok{:}\NormalTok{ A}\OtherTok{\}}
        \OtherTok{→} \OtherTok{(}\NormalTok{e }\OtherTok{:}\NormalTok{ a ≡ b}\OtherTok{)} \OtherTok{→}\NormalTok{ refl ≡ }\OtherTok{(}\NormalTok{b ≡〈 sym e 〉 e}\OtherTok{)}
\NormalTok{≡siml refl }\OtherTok{=}\NormalTok{ refl}

\NormalTok{≡idr }\OtherTok{:} \OtherTok{∀} \OtherTok{\{}\NormalTok{ℓ}\OtherTok{\}} \OtherTok{\{}\NormalTok{A }\OtherTok{:}\NormalTok{ Type ℓ}\OtherTok{\}} \OtherTok{\{}\NormalTok{a b }\OtherTok{:}\NormalTok{ A}\OtherTok{\}}
       \OtherTok{→} \OtherTok{(}\NormalTok{e }\OtherTok{:}\NormalTok{ a ≡ b}\OtherTok{)} \OtherTok{→}\NormalTok{ e ≡ }\OtherTok{(}\NormalTok{a ≡〈 refl 〉 e}\OtherTok{)}
\NormalTok{≡idr refl }\OtherTok{=}\NormalTok{ refl}

\NormalTok{conj }\OtherTok{:} \OtherTok{∀} \OtherTok{\{}\NormalTok{ℓ}\OtherTok{\}} \OtherTok{\{}\NormalTok{A }\OtherTok{:}\NormalTok{ Type ℓ}\OtherTok{\}} \OtherTok{\{}\NormalTok{a b c d }\OtherTok{:}\NormalTok{ A}\OtherTok{\}}
       \OtherTok{→} \OtherTok{(}\NormalTok{e1 }\OtherTok{:}\NormalTok{ a ≡ b}\OtherTok{)} \OtherTok{(}\NormalTok{e2 }\OtherTok{:}\NormalTok{ a ≡ c}\OtherTok{)} \OtherTok{(}\NormalTok{e3 }\OtherTok{:}\NormalTok{ b ≡ d}\OtherTok{)} \OtherTok{(}\NormalTok{e4 }\OtherTok{:}\NormalTok{ c ≡ d}\OtherTok{)}
       \OtherTok{→} \OtherTok{(}\NormalTok{a ≡〈 e1 〉 e3}\OtherTok{)}\NormalTok{ ≡ }\OtherTok{(}\NormalTok{a ≡〈 e2 〉 e4}\OtherTok{)}
       \OtherTok{→}\NormalTok{ e3 ≡ }\OtherTok{(}\NormalTok{b ≡〈 sym e1 〉}\OtherTok{(}\NormalTok{a ≡〈 e2 〉 e4}\OtherTok{))}
\NormalTok{conj e1 e2 refl refl refl }\OtherTok{=}\NormalTok{ ≡siml e1}

\NormalTok{nat }\OtherTok{:} \OtherTok{∀} \OtherTok{\{}\NormalTok{ℓ κ}\OtherTok{\}} \OtherTok{\{}\NormalTok{A }\OtherTok{:}\NormalTok{ Type ℓ}\OtherTok{\}} \OtherTok{\{}\NormalTok{B }\OtherTok{:}\NormalTok{ Type κ}\OtherTok{\}} \OtherTok{\{}\NormalTok{f g }\OtherTok{:}\NormalTok{ A }\OtherTok{→}\NormalTok{ B}\OtherTok{\}} \OtherTok{\{}\NormalTok{a b }\OtherTok{:}\NormalTok{ A}\OtherTok{\}}
      \OtherTok{→} \OtherTok{(}\NormalTok{α }\OtherTok{:} \OtherTok{(}\NormalTok{x }\OtherTok{:}\NormalTok{ A}\OtherTok{)} \OtherTok{→}\NormalTok{ f x ≡ g x}\OtherTok{)} \OtherTok{(}\NormalTok{e }\OtherTok{:}\NormalTok{ a ≡ b}\OtherTok{)}
      \OtherTok{→} \OtherTok{((}\NormalTok{f a}\OtherTok{)}\NormalTok{ ≡〈 α a 〉 }\OtherTok{(}\NormalTok{ap g e}\OtherTok{))}\NormalTok{ ≡ }\OtherTok{((}\NormalTok{f a}\OtherTok{)}\NormalTok{ ≡〈 ap f e 〉 }\OtherTok{(}\NormalTok{α b}\OtherTok{))}
\NormalTok{nat }\OtherTok{\{}\NormalTok{a }\OtherTok{=}\NormalTok{ a}\OtherTok{\}}\NormalTok{ α refl }\OtherTok{=}\NormalTok{ ≡idr }\OtherTok{(}\NormalTok{α a}\OtherTok{)}

\NormalTok{cancel }\OtherTok{:} \OtherTok{∀} \OtherTok{\{}\NormalTok{ℓ}\OtherTok{\}} \OtherTok{\{}\NormalTok{A }\OtherTok{:}\NormalTok{ Type ℓ}\OtherTok{\}} \OtherTok{\{}\NormalTok{a b c }\OtherTok{:}\NormalTok{ A}\OtherTok{\}}
         \OtherTok{→} \OtherTok{(}\NormalTok{e1 e2 }\OtherTok{:}\NormalTok{ a ≡ b}\OtherTok{)} \OtherTok{(}\NormalTok{e3 }\OtherTok{:}\NormalTok{ b ≡ c}\OtherTok{)}
         \OtherTok{→} \OtherTok{(}\NormalTok{a ≡〈 e1 〉 e3}\OtherTok{)}\NormalTok{ ≡ }\OtherTok{(}\NormalTok{a ≡〈 e2 〉 e3}\OtherTok{)}
         \OtherTok{→}\NormalTok{ e1 ≡ e2}
\NormalTok{cancel e1 e2 refl refl }\OtherTok{=}\NormalTok{ refl}

\NormalTok{apId }\OtherTok{:} \OtherTok{∀} \OtherTok{\{}\NormalTok{ℓ}\OtherTok{\}} \OtherTok{\{}\NormalTok{A }\OtherTok{:}\NormalTok{ Type ℓ}\OtherTok{\}} \OtherTok{\{}\NormalTok{a b }\OtherTok{:}\NormalTok{ A}\OtherTok{\}}
       \OtherTok{→} \OtherTok{(}\NormalTok{e }\OtherTok{:}\NormalTok{ a ≡ b}\OtherTok{)} \OtherTok{→}\NormalTok{ ap }\OtherTok{(λ}\NormalTok{ x }\OtherTok{→}\NormalTok{ x}\OtherTok{)}\NormalTok{ e ≡ e}
\NormalTok{apId refl }\OtherTok{=}\NormalTok{ refl}

\NormalTok{apComp }\OtherTok{:} \OtherTok{∀} \OtherTok{\{}\NormalTok{ℓ ℓ\textquotesingle{} ℓ\textquotesingle{}\textquotesingle{}}\OtherTok{\}} \OtherTok{\{}\NormalTok{A }\OtherTok{:}\NormalTok{ Type ℓ}\OtherTok{\}} \OtherTok{\{}\NormalTok{B }\OtherTok{:}\NormalTok{ Type ℓ\textquotesingle{}}\OtherTok{\}} \OtherTok{\{}\NormalTok{C }\OtherTok{:}\NormalTok{ Type ℓ\textquotesingle{}\textquotesingle{}}\OtherTok{\}} \OtherTok{\{}\NormalTok{a b }\OtherTok{:}\NormalTok{ A}\OtherTok{\}}
         \OtherTok{→} \OtherTok{(}\NormalTok{f }\OtherTok{:}\NormalTok{ A }\OtherTok{→}\NormalTok{ B}\OtherTok{)} \OtherTok{(}\NormalTok{g }\OtherTok{:}\NormalTok{ B }\OtherTok{→}\NormalTok{ C}\OtherTok{)} \OtherTok{(}\NormalTok{e }\OtherTok{:}\NormalTok{ a ≡ b}\OtherTok{)}
         \OtherTok{→}\NormalTok{ ap }\OtherTok{(λ}\NormalTok{ x }\OtherTok{→}\NormalTok{ g }\OtherTok{(}\NormalTok{f x}\OtherTok{))}\NormalTok{ e ≡ ap g }\OtherTok{(}\NormalTok{ap f e}\OtherTok{)}
\NormalTok{apComp f g refl }\OtherTok{=}\NormalTok{ refl}

\NormalTok{apHtpy }\OtherTok{:} \OtherTok{∀} \OtherTok{\{}\NormalTok{ℓ}\OtherTok{\}} \OtherTok{\{}\NormalTok{A }\OtherTok{:}\NormalTok{ Type ℓ}\OtherTok{\}} \OtherTok{\{}\NormalTok{a }\OtherTok{:}\NormalTok{ A}\OtherTok{\}}
         \OtherTok{→} \OtherTok{(}\NormalTok{i }\OtherTok{:}\NormalTok{ A }\OtherTok{→}\NormalTok{ A}\OtherTok{)} \OtherTok{(}\NormalTok{α }\OtherTok{:} \OtherTok{(}\NormalTok{x }\OtherTok{:}\NormalTok{ A}\OtherTok{)} \OtherTok{→}\NormalTok{ i x ≡ x}\OtherTok{)}
         \OtherTok{→}\NormalTok{ ap i }\OtherTok{(}\NormalTok{α a}\OtherTok{)}\NormalTok{ ≡ α }\OtherTok{(}\NormalTok{i a}\OtherTok{)}
\NormalTok{apHtpy }\OtherTok{\{}\NormalTok{a }\OtherTok{=}\NormalTok{ a}\OtherTok{\}}\NormalTok{ i α }\OtherTok{=} 
\NormalTok{    cancel }\OtherTok{(}\NormalTok{ap i }\OtherTok{(}\NormalTok{α a}\OtherTok{))} \OtherTok{(}\NormalTok{α }\OtherTok{(}\NormalTok{i a}\OtherTok{))} \OtherTok{(}\NormalTok{α a}\OtherTok{)} 
           \OtherTok{((}\NormalTok{i }\OtherTok{(}\NormalTok{i a}\OtherTok{)}\NormalTok{ ≡〈 ap i }\OtherTok{(}\NormalTok{α a}\OtherTok{)}\NormalTok{ 〉 α a}\OtherTok{)} 
\NormalTok{           ≡〈 sym }\OtherTok{(}\NormalTok{nat α }\OtherTok{(}\NormalTok{α a}\OtherTok{))}\NormalTok{ 〉 }
           \OtherTok{((}\NormalTok{i }\OtherTok{(}\NormalTok{i a}\OtherTok{)}\NormalTok{ ≡〈 α }\OtherTok{(}\NormalTok{i a}\OtherTok{)}\NormalTok{ 〉 ap }\OtherTok{(λ}\NormalTok{ z }\OtherTok{→}\NormalTok{ z}\OtherTok{)} \OtherTok{(}\NormalTok{α a}\OtherTok{))} 
\NormalTok{           ≡〈 ap }\OtherTok{(λ}\NormalTok{ e }\OtherTok{→}\NormalTok{ i }\OtherTok{(}\NormalTok{i a}\OtherTok{)}\NormalTok{ ≡〈 α }\OtherTok{(}\NormalTok{i a}\OtherTok{)}\NormalTok{ 〉 e}\OtherTok{)} \OtherTok{(}\NormalTok{apId }\OtherTok{(}\NormalTok{α a}\OtherTok{))}\NormalTok{ 〉 }
           \OtherTok{((}\NormalTok{i }\OtherTok{(}\NormalTok{i a}\OtherTok{)}\NormalTok{ ≡〈 α }\OtherTok{(}\NormalTok{i a}\OtherTok{)}\NormalTok{ 〉 α a}\OtherTok{)}\NormalTok{ □}\OtherTok{)))}

\NormalTok{HAdj }\OtherTok{:} \OtherTok{∀} \OtherTok{\{}\NormalTok{ℓ κ}\OtherTok{\}} \OtherTok{\{}\NormalTok{A }\OtherTok{:}\NormalTok{ Type ℓ}\OtherTok{\}} \OtherTok{\{}\NormalTok{B }\OtherTok{:}\NormalTok{ Type κ}\OtherTok{\}}
       \OtherTok{→} \OtherTok{(}\NormalTok{A }\OtherTok{→}\NormalTok{ B}\OtherTok{)} \OtherTok{→} \DataTypeTok{Set} \OtherTok{(}\NormalTok{ℓ ⊔ κ}\OtherTok{)}
\NormalTok{HAdj }\OtherTok{\{}\NormalTok{A }\OtherTok{=}\NormalTok{ A}\OtherTok{\}} \OtherTok{\{}\NormalTok{B }\OtherTok{=}\NormalTok{ B}\OtherTok{\}}\NormalTok{ f }\OtherTok{=}
\NormalTok{    Σ }\OtherTok{(}\NormalTok{B }\OtherTok{→}\NormalTok{ A}\OtherTok{)} \OtherTok{(λ}\NormalTok{ g }\OtherTok{→} 
\NormalTok{      Σ }\OtherTok{((}\NormalTok{x }\OtherTok{:}\NormalTok{ A}\OtherTok{)} \OtherTok{→}\NormalTok{ g }\OtherTok{(}\NormalTok{f x}\OtherTok{)}\NormalTok{ ≡ x}\OtherTok{)} \OtherTok{(λ}\NormalTok{ η }\OtherTok{→} 
\NormalTok{        Σ }\OtherTok{((}\NormalTok{y }\OtherTok{:}\NormalTok{ B}\OtherTok{)} \OtherTok{→}\NormalTok{ f }\OtherTok{(}\NormalTok{g y}\OtherTok{)}\NormalTok{ ≡ y}\OtherTok{)} \OtherTok{(λ}\NormalTok{ ε }\OtherTok{→} 
          \OtherTok{(}\NormalTok{x }\OtherTok{:}\NormalTok{ A}\OtherTok{)} \OtherTok{→}\NormalTok{ ap f }\OtherTok{(}\NormalTok{η x}\OtherTok{)}\NormalTok{ ≡ ε }\OtherTok{(}\NormalTok{f x}\OtherTok{))))}

\NormalTok{Iso→HAdj }\OtherTok{:} \OtherTok{∀} \OtherTok{\{}\NormalTok{ℓ κ}\OtherTok{\}} \OtherTok{\{}\NormalTok{A }\OtherTok{:}\NormalTok{ Type ℓ}\OtherTok{\}} \OtherTok{\{}\NormalTok{B }\OtherTok{:}\NormalTok{ Type κ}\OtherTok{\}} \OtherTok{\{}\NormalTok{f }\OtherTok{:}\NormalTok{ A }\OtherTok{→}\NormalTok{ B}\OtherTok{\}}
           \OtherTok{→}\NormalTok{ Iso f }\OtherTok{→}\NormalTok{ HAdj f}
\NormalTok{Iso→HAdj }\OtherTok{\{}\NormalTok{f }\OtherTok{=}\NormalTok{ f}\OtherTok{\}} \OtherTok{(}\NormalTok{g , η , ε}\OtherTok{)} \OtherTok{=}
\NormalTok{    g , }\OtherTok{(}\NormalTok{η }
\NormalTok{    , }\OtherTok{(} \OtherTok{(λ}\NormalTok{ y }\OtherTok{→} 
\NormalTok{           f }\OtherTok{(}\NormalTok{g y}\OtherTok{)}\NormalTok{         ≡〈 sym }\OtherTok{(}\NormalTok{ε }\OtherTok{(}\NormalTok{f }\OtherTok{(}\NormalTok{g y}\OtherTok{)))}\NormalTok{ 〉 }
          \OtherTok{(}\NormalTok{f }\OtherTok{(}\NormalTok{g }\OtherTok{(}\NormalTok{f }\OtherTok{(}\NormalTok{g y}\OtherTok{)))}\NormalTok{ ≡〈 ap f }\OtherTok{(}\NormalTok{η }\OtherTok{(}\NormalTok{g y}\OtherTok{))}\NormalTok{ 〉 }
          \OtherTok{(}\NormalTok{f }\OtherTok{(}\NormalTok{g y}\OtherTok{)}\NormalTok{         ≡〈 ε y 〉 }
          \OtherTok{(}\NormalTok{y               □}\OtherTok{))))} 
\NormalTok{      , }\OtherTok{λ}\NormalTok{ x }\OtherTok{→}\NormalTok{ conj }\OtherTok{(}\NormalTok{ε }\OtherTok{(}\NormalTok{f }\OtherTok{(}\NormalTok{g }\OtherTok{(}\NormalTok{f x}\OtherTok{))))} \OtherTok{(}\NormalTok{ap f }\OtherTok{(}\NormalTok{η }\OtherTok{(}\NormalTok{g }\OtherTok{(}\NormalTok{f x}\OtherTok{))))} 
                   \OtherTok{(}\NormalTok{ap f }\OtherTok{(}\NormalTok{η x}\OtherTok{))} \OtherTok{(}\NormalTok{ε }\OtherTok{(}\NormalTok{f x}\OtherTok{))} 
                   \OtherTok{(((}\NormalTok{f }\OtherTok{(}\NormalTok{g }\OtherTok{(}\NormalTok{f }\OtherTok{(}\NormalTok{g }\OtherTok{(}\NormalTok{f x}\OtherTok{))))}\NormalTok{ ≡〈 ε }\OtherTok{(}\NormalTok{f }\OtherTok{(}\NormalTok{g }\OtherTok{(}\NormalTok{f x}\OtherTok{)))}\NormalTok{ 〉 ap f }\OtherTok{(}\NormalTok{η x}\OtherTok{)))} 
\NormalTok{                    ≡〈 nat }\OtherTok{(λ}\NormalTok{ z }\OtherTok{→}\NormalTok{ ε }\OtherTok{(}\NormalTok{f z}\OtherTok{))} \OtherTok{(}\NormalTok{η x}\OtherTok{)}\NormalTok{ 〉 }
                    \OtherTok{(((}\NormalTok{f }\OtherTok{(}\NormalTok{g }\OtherTok{(}\NormalTok{f }\OtherTok{(}\NormalTok{g }\OtherTok{(}\NormalTok{f x}\OtherTok{))))}\NormalTok{ ≡〈 ap }\OtherTok{(λ}\NormalTok{ z }\OtherTok{→}\NormalTok{ f }\OtherTok{(}\NormalTok{g }\OtherTok{(}\NormalTok{f z}\OtherTok{)))} \OtherTok{(}\NormalTok{η x}\OtherTok{)}\NormalTok{ 〉 ε }\OtherTok{(}\NormalTok{f x}\OtherTok{)))} 
\NormalTok{                    ≡〈 ap }\OtherTok{(λ}\NormalTok{ e }\OtherTok{→} \OtherTok{(}\NormalTok{f }\OtherTok{(}\NormalTok{g }\OtherTok{(}\NormalTok{f }\OtherTok{(}\NormalTok{g }\OtherTok{(}\NormalTok{f x}\OtherTok{))))}\NormalTok{ ≡〈 e 〉 ε }\OtherTok{(}\NormalTok{f x}\OtherTok{)))} 
                          \OtherTok{((}\NormalTok{ap }\OtherTok{(λ}\NormalTok{ z }\OtherTok{→}\NormalTok{ f }\OtherTok{(}\NormalTok{g }\OtherTok{(}\NormalTok{f z}\OtherTok{)))} \OtherTok{(}\NormalTok{η x}\OtherTok{))} 
\NormalTok{                           ≡〈 apComp }\OtherTok{(λ}\NormalTok{ z }\OtherTok{→}\NormalTok{ g }\OtherTok{(}\NormalTok{f z}\OtherTok{))}\NormalTok{ f }\OtherTok{(}\NormalTok{η x}\OtherTok{)}\NormalTok{ 〉 }
                           \OtherTok{((}\NormalTok{ap f }\OtherTok{(}\NormalTok{ap }\OtherTok{(λ}\NormalTok{ z }\OtherTok{→}\NormalTok{ g }\OtherTok{(}\NormalTok{f z}\OtherTok{))} \OtherTok{(}\NormalTok{η x}\OtherTok{)))} 
\NormalTok{                           ≡〈 ap }\OtherTok{(}\NormalTok{ap f}\OtherTok{)} \OtherTok{(}\NormalTok{apHtpy }\OtherTok{(λ}\NormalTok{ z }\OtherTok{→}\NormalTok{ g }\OtherTok{(}\NormalTok{f z}\OtherTok{))}\NormalTok{ η}\OtherTok{)}\NormalTok{ 〉 }
                           \OtherTok{(}\NormalTok{ap f }\OtherTok{(}\NormalTok{η }\OtherTok{(}\NormalTok{g }\OtherTok{(}\NormalTok{f x}\OtherTok{)))}\NormalTok{ □}\OtherTok{)))}\NormalTok{ 〉 }
                    \OtherTok{(((}\NormalTok{f }\OtherTok{(}\NormalTok{g }\OtherTok{(}\NormalTok{f }\OtherTok{(}\NormalTok{g }\OtherTok{(}\NormalTok{f x}\OtherTok{))))}\NormalTok{ ≡〈 ap f }\OtherTok{(}\NormalTok{η }\OtherTok{(}\NormalTok{g }\OtherTok{(}\NormalTok{f x}\OtherTok{)))}\NormalTok{ 〉 ε }\OtherTok{(}\NormalTok{f x}\OtherTok{)))}\NormalTok{ □}\OtherTok{)))))}

\NormalTok{pairEquiv1 }\OtherTok{:} \OtherTok{∀} \OtherTok{\{}\NormalTok{ℓ ℓ\textquotesingle{} κ}\OtherTok{\}} \OtherTok{\{}\NormalTok{A }\OtherTok{:}\NormalTok{ Type ℓ}\OtherTok{\}} \OtherTok{\{}\NormalTok{A\textquotesingle{} }\OtherTok{:}\NormalTok{ Type ℓ\textquotesingle{}}\OtherTok{\}} \OtherTok{\{}\NormalTok{B }\OtherTok{:}\NormalTok{ A\textquotesingle{} }\OtherTok{→}\NormalTok{ Type κ}\OtherTok{\}}
             \OtherTok{→} \OtherTok{(}\NormalTok{f }\OtherTok{:}\NormalTok{ A }\OtherTok{→}\NormalTok{ A\textquotesingle{}}\OtherTok{)} \OtherTok{→}\NormalTok{ isEquiv f}
             \OtherTok{→}\NormalTok{ isEquiv }\OtherTok{\{}\NormalTok{A }\OtherTok{=}\NormalTok{ Σ A }\OtherTok{(λ}\NormalTok{ x }\OtherTok{→}\NormalTok{ B }\OtherTok{(}\NormalTok{f x}\OtherTok{))\}} \OtherTok{\{}\NormalTok{B }\OtherTok{=}\NormalTok{ Σ A\textquotesingle{} B}\OtherTok{\}} 
                       \OtherTok{(λ} \OtherTok{(}\NormalTok{x , y}\OtherTok{)} \OtherTok{→} \OtherTok{(}\NormalTok{f x , y}\OtherTok{))}
\NormalTok{pairEquiv1 }\OtherTok{\{}\NormalTok{A }\OtherTok{=}\NormalTok{ A}\OtherTok{\}} \OtherTok{\{}\NormalTok{A\textquotesingle{} }\OtherTok{=}\NormalTok{ A\textquotesingle{}}\OtherTok{\}} \OtherTok{\{}\NormalTok{B }\OtherTok{=}\NormalTok{ B}\OtherTok{\}}\NormalTok{ f ef }\OtherTok{=}
\NormalTok{  Iso→isEquiv}
    \OtherTok{(} \OtherTok{(λ} \OtherTok{(}\NormalTok{x , y}\OtherTok{)} \OtherTok{→} \OtherTok{(}\NormalTok{g x , transp B }\OtherTok{(}\NormalTok{sym }\OtherTok{(}\NormalTok{ε x}\OtherTok{))}\NormalTok{ y}\OtherTok{))}
\NormalTok{    , }\OtherTok{(} \OtherTok{(λ} \OtherTok{(}\NormalTok{x , y}\OtherTok{)} \OtherTok{→}\NormalTok{ pairEq }\OtherTok{(}\NormalTok{η x}\OtherTok{)} \OtherTok{(}\NormalTok{lemma x y}\OtherTok{))} 
\NormalTok{      , }\OtherTok{λ} \OtherTok{(}\NormalTok{x , y}\OtherTok{)} \OtherTok{→}\NormalTok{ pairEq }\OtherTok{(}\NormalTok{ε x}\OtherTok{)} \OtherTok{(}\NormalTok{symr }\OtherTok{(}\NormalTok{ε x}\OtherTok{)}\NormalTok{ y}\OtherTok{)} \OtherTok{)} \OtherTok{)}
  \KeywordTok{where}
\NormalTok{    g }\OtherTok{:}\NormalTok{ A\textquotesingle{} }\OtherTok{→}\NormalTok{ A}
\NormalTok{    g }\OtherTok{=}\NormalTok{ fst }\OtherTok{(}\NormalTok{Iso→HAdj }\OtherTok{(}\NormalTok{isEquiv→Iso ef}\OtherTok{))}
\NormalTok{    η }\OtherTok{:} \OtherTok{(}\NormalTok{x }\OtherTok{:}\NormalTok{ A}\OtherTok{)} \OtherTok{→}\NormalTok{ g }\OtherTok{(}\NormalTok{f x}\OtherTok{)}\NormalTok{ ≡ x}
\NormalTok{    η }\OtherTok{=}\NormalTok{ fst }\OtherTok{(}\NormalTok{snd }\OtherTok{(}\NormalTok{Iso→HAdj }\OtherTok{(}\NormalTok{isEquiv→Iso ef}\OtherTok{)))}
\NormalTok{    ε }\OtherTok{:} \OtherTok{(}\NormalTok{y }\OtherTok{:}\NormalTok{ A\textquotesingle{}}\OtherTok{)} \OtherTok{→}\NormalTok{ f }\OtherTok{(}\NormalTok{g y}\OtherTok{)}\NormalTok{ ≡ y}
\NormalTok{    ε }\OtherTok{=}\NormalTok{ fst }\OtherTok{(}\NormalTok{snd }\OtherTok{(}\NormalTok{snd }\OtherTok{(}\NormalTok{Iso→HAdj }\OtherTok{(}\NormalTok{isEquiv→Iso ef}\OtherTok{))))}
\NormalTok{    ρ }\OtherTok{:} \OtherTok{(}\NormalTok{x }\OtherTok{:}\NormalTok{ A}\OtherTok{)} \OtherTok{→}\NormalTok{ ap f }\OtherTok{(}\NormalTok{η x}\OtherTok{)}\NormalTok{ ≡ ε }\OtherTok{(}\NormalTok{f x}\OtherTok{)}
\NormalTok{    ρ }\OtherTok{=}\NormalTok{ snd }\OtherTok{(}\NormalTok{snd }\OtherTok{(}\NormalTok{snd }\OtherTok{(}\NormalTok{Iso→HAdj }\OtherTok{(}\NormalTok{isEquiv→Iso ef}\OtherTok{))))}
\NormalTok{    lemma }\OtherTok{:} \OtherTok{(}\NormalTok{x }\OtherTok{:}\NormalTok{ A}\OtherTok{)} \OtherTok{(}\NormalTok{y }\OtherTok{:}\NormalTok{ B }\OtherTok{(}\NormalTok{f x}\OtherTok{))}
            \OtherTok{→}\NormalTok{ transp }\OtherTok{(λ}\NormalTok{ z }\OtherTok{→}\NormalTok{ B }\OtherTok{(}\NormalTok{f z}\OtherTok{))} \OtherTok{(}\NormalTok{η x}\OtherTok{)}
                     \OtherTok{(}\NormalTok{transp B }\OtherTok{(}\NormalTok{sym }\OtherTok{(}\NormalTok{ε }\OtherTok{(}\NormalTok{f x}\OtherTok{)))}\NormalTok{ y}\OtherTok{)}
\NormalTok{              ≡ y}
\NormalTok{    lemma x y }\OtherTok{=} \OtherTok{(}\NormalTok{transp }\OtherTok{(λ}\NormalTok{ z }\OtherTok{→}\NormalTok{ B }\OtherTok{(}\NormalTok{f z}\OtherTok{))} \OtherTok{(}\NormalTok{η x}\OtherTok{)} 
                        \OtherTok{(}\NormalTok{transp B }\OtherTok{(}\NormalTok{sym }\OtherTok{(}\NormalTok{ε }\OtherTok{(}\NormalTok{f x}\OtherTok{)))}\NormalTok{ y}\OtherTok{))} 
\NormalTok{                ≡〈 transpAp B f }\OtherTok{(}\NormalTok{η x}\OtherTok{)} 
                            \OtherTok{(}\NormalTok{transp B }\OtherTok{(}\NormalTok{sym }\OtherTok{(}\NormalTok{ε }\OtherTok{(}\NormalTok{f x}\OtherTok{)))}\NormalTok{ y}\OtherTok{)}\NormalTok{ 〉 }
                \OtherTok{(}\NormalTok{ transp B }\OtherTok{(}\NormalTok{ap f }\OtherTok{(}\NormalTok{η x}\OtherTok{))} 
                           \OtherTok{(}\NormalTok{transp B }\OtherTok{(}\NormalTok{sym }\OtherTok{(}\NormalTok{ε }\OtherTok{(}\NormalTok{f x}\OtherTok{)))}\NormalTok{ y}\OtherTok{)} 
\NormalTok{                ≡〈 ap }\OtherTok{(λ}\NormalTok{ e }\OtherTok{→}\NormalTok{ transp B e }
                                \OtherTok{(}\NormalTok{transp B }\OtherTok{(}\NormalTok{sym }\OtherTok{(}\NormalTok{ε }\OtherTok{(}\NormalTok{f x}\OtherTok{)))}\NormalTok{ y}\OtherTok{))} 
                      \OtherTok{(}\NormalTok{ρ x}\OtherTok{)}\NormalTok{ 〉 }
                \OtherTok{(} \OtherTok{(}\NormalTok{transp B }\OtherTok{(}\NormalTok{ε }\OtherTok{(}\NormalTok{f x}\OtherTok{))} 
                          \OtherTok{(}\NormalTok{transp B }\OtherTok{(}\NormalTok{sym }\OtherTok{(}\NormalTok{ε }\OtherTok{(}\NormalTok{f x}\OtherTok{)))}\NormalTok{ y}\OtherTok{))} 
\NormalTok{                ≡〈 }\OtherTok{(}\NormalTok{symr }\OtherTok{(}\NormalTok{ε }\OtherTok{(}\NormalTok{f x}\OtherTok{))}\NormalTok{ y}\OtherTok{)}\NormalTok{ 〉 }
                \OtherTok{(}\NormalTok{y □}\OtherTok{)))}

\NormalTok{pairEquiv2 }\OtherTok{:} \OtherTok{∀} \OtherTok{\{}\NormalTok{ℓ κ κ\textquotesingle{}}\OtherTok{\}} \OtherTok{\{}\NormalTok{A }\OtherTok{:}\NormalTok{ Type ℓ}\OtherTok{\}} \OtherTok{\{}\NormalTok{B }\OtherTok{:}\NormalTok{ A }\OtherTok{→}\NormalTok{ Type κ}\OtherTok{\}} \OtherTok{\{}\NormalTok{B\textquotesingle{} }\OtherTok{:}\NormalTok{ A }\OtherTok{→}\NormalTok{ Type κ\textquotesingle{}}\OtherTok{\}}
             \OtherTok{→} \OtherTok{(}\NormalTok{g }\OtherTok{:} \OtherTok{(}\NormalTok{x }\OtherTok{:}\NormalTok{ A}\OtherTok{)} \OtherTok{→}\NormalTok{ B x }\OtherTok{→}\NormalTok{ B\textquotesingle{} x}\OtherTok{)} \OtherTok{→} \OtherTok{((}\NormalTok{x }\OtherTok{:}\NormalTok{ A}\OtherTok{)} \OtherTok{→}\NormalTok{ isEquiv }\OtherTok{(}\NormalTok{g x}\OtherTok{))}
             \OtherTok{→}\NormalTok{ isEquiv }\OtherTok{\{}\NormalTok{A }\OtherTok{=}\NormalTok{ Σ A B}\OtherTok{\}} \OtherTok{\{}\NormalTok{B }\OtherTok{=}\NormalTok{ Σ A B\textquotesingle{}}\OtherTok{\}}
                       \OtherTok{(λ} \OtherTok{(}\NormalTok{x , y}\OtherTok{)} \OtherTok{→} \OtherTok{(}\NormalTok{x , g x y}\OtherTok{))}
\NormalTok{pairEquiv2 g eg }\OtherTok{=}
    \KeywordTok{let}\NormalTok{ isog }\OtherTok{=} \OtherTok{(λ}\NormalTok{ x }\OtherTok{→}\NormalTok{ isEquiv→Iso }\OtherTok{(}\NormalTok{eg x}\OtherTok{))} 
    \KeywordTok{in}\NormalTok{ Iso→isEquiv }\OtherTok{(} \OtherTok{(λ} \OtherTok{(}\NormalTok{x , y}\OtherTok{)} \OtherTok{→} \OtherTok{(}\NormalTok{x , fst }\OtherTok{(}\NormalTok{isog x}\OtherTok{)}\NormalTok{ y}\OtherTok{))} 
\NormalTok{                   , }\OtherTok{(} \OtherTok{(λ} \OtherTok{(}\NormalTok{x , y}\OtherTok{)} \OtherTok{→} 
\NormalTok{                          pairEq refl }\OtherTok{(}\NormalTok{fst }\OtherTok{(}\NormalTok{snd }\OtherTok{(}\NormalTok{isog x}\OtherTok{))}\NormalTok{ y}\OtherTok{))} 
\NormalTok{                     , }\OtherTok{λ} \OtherTok{(}\NormalTok{x , y}\OtherTok{)} \OtherTok{→} 
\NormalTok{                         pairEq refl }\OtherTok{(}\NormalTok{snd }\OtherTok{(}\NormalTok{snd }\OtherTok{(}\NormalTok{isog x}\OtherTok{))}\NormalTok{ y}\OtherTok{)))}

\NormalTok{pairEquiv }\OtherTok{:} \OtherTok{∀} \OtherTok{\{}\NormalTok{ℓ ℓ\textquotesingle{} κ κ\textquotesingle{}}\OtherTok{\}} \OtherTok{\{}\NormalTok{A }\OtherTok{:}\NormalTok{ Type ℓ}\OtherTok{\}} \OtherTok{\{}\NormalTok{A\textquotesingle{} }\OtherTok{:}\NormalTok{ Type ℓ\textquotesingle{}}\OtherTok{\}}
            \OtherTok{→} \OtherTok{\{}\NormalTok{B }\OtherTok{:}\NormalTok{ A }\OtherTok{→}\NormalTok{ Type κ}\OtherTok{\}} \OtherTok{\{}\NormalTok{B\textquotesingle{} }\OtherTok{:}\NormalTok{ A\textquotesingle{} }\OtherTok{→}\NormalTok{ Type κ\textquotesingle{}}\OtherTok{\}}
            \OtherTok{→} \OtherTok{(}\NormalTok{f }\OtherTok{:}\NormalTok{ A }\OtherTok{→}\NormalTok{ A\textquotesingle{}}\OtherTok{)} \OtherTok{(}\NormalTok{g }\OtherTok{:} \OtherTok{(}\NormalTok{x }\OtherTok{:}\NormalTok{ A}\OtherTok{)} \OtherTok{→}\NormalTok{ B x }\OtherTok{→}\NormalTok{ B\textquotesingle{} }\OtherTok{(}\NormalTok{f x}\OtherTok{))}
            \OtherTok{→}\NormalTok{ isEquiv f }\OtherTok{→} \OtherTok{((}\NormalTok{x }\OtherTok{:}\NormalTok{ A}\OtherTok{)} \OtherTok{→}\NormalTok{ isEquiv }\OtherTok{(}\NormalTok{g x}\OtherTok{))}
            \OtherTok{→}\NormalTok{ isEquiv }\OtherTok{\{}\NormalTok{A }\OtherTok{=}\NormalTok{ Σ A B}\OtherTok{\}} \OtherTok{\{}\NormalTok{B }\OtherTok{=}\NormalTok{ Σ A\textquotesingle{} B\textquotesingle{}}\OtherTok{\}}
                      \OtherTok{(λ} \OtherTok{(}\NormalTok{x , y}\OtherTok{)} \OtherTok{→} \OtherTok{(}\NormalTok{f x , g x y}\OtherTok{))}
\NormalTok{pairEquiv f g ef eg }\OtherTok{=} 
\NormalTok{    compIsEquiv }\OtherTok{(λ} \OtherTok{(}\NormalTok{x , y}\OtherTok{)} \OtherTok{→} \OtherTok{(}\NormalTok{f x , y}\OtherTok{))} 
                \OtherTok{(λ} \OtherTok{(}\NormalTok{x , y}\OtherTok{)} \OtherTok{→} \OtherTok{(}\NormalTok{x , g x y}\OtherTok{))} 
                \OtherTok{(}\NormalTok{pairEquiv1 f ef}\OtherTok{)} 
                \OtherTok{(}\NormalTok{pairEquiv2 g eg}\OtherTok{)}
\end{Highlighting}
\end{Shaded}

\dnote{Part of me does think most of this should be in the main paper, rather than the appendix. But this very last part just feels long... Maybe with more text it could be included?} 

\chapter{Polynomials in HoTT}\label{polynomials-in-hott}

\dnote{Introduce the section}

\begin{remark}
Since essentially all of the
categorical structures treated in this paper will be
infinite-dimensional, we shall generally omit the prefix ``\((\infty,1)\)-''
from our descriptions these structures. Hence hereafter ``category''
generally means \((\infty,1)\)-category, ``functor'' means
\((\infty,1)\)-functor, etc.
\end{remark}

\section{Basics}\label{basics}

Let \(\mathbf{Type}\) be the category of types and functions between
them. Given a type \texttt{A}, let \(\yon^A\) denote the corresponding
representable functor \(\mathbf{Type} \to \mathbf{Type}\).

A \emph{polynomial functor} is a coproduct of representable functors
\(\mathbf{Type} \to \mathbf{Type}\), i.e.~an endofunctor on
\(\mathbf{Type}\) of the form \[
\sum_{a : A} \yon^{B(a)}
\] for some type \texttt{A} and family of types
\texttt{B\ :\ A\ →\ Type}. The data of a polynomial functor is thus
uniquely determined by the choice of \texttt{A} and \texttt{B}. Hence we
may represent such functors in Agda simply as pairs (A , B) of this
form:

\begin{Shaded}
\begin{Highlighting}[]
\NormalTok{Poly }\OtherTok{:} \OtherTok{(}\NormalTok{ℓ κ }\OtherTok{:}\NormalTok{ Level}\OtherTok{)} \OtherTok{→}\NormalTok{ Type }\OtherTok{((}\NormalTok{lsuc ℓ}\OtherTok{)}\NormalTok{ ⊔ }\OtherTok{(}\NormalTok{lsuc κ}\OtherTok{))}
\NormalTok{Poly ℓ κ }\OtherTok{=}\NormalTok{ Σ }\OtherTok{(}\DataTypeTok{Set}\NormalTok{ ℓ}\OtherTok{)} \OtherTok{(λ}\NormalTok{ A }\OtherTok{→}\NormalTok{ A }\OtherTok{→} \DataTypeTok{Set}\NormalTok{ κ}\OtherTok{)}
\end{Highlighting}
\end{Shaded}\dnote{I was surprised to see $\DataTypeTok{Set}$ show up, especially in the base. I thought we were allowing an arbitrary type here?}

A basic example of such a polynomial functor is the identity functor
\texttt{𝕪} consisting of a single term of unit arity -- hence
represented by the pair \texttt{(⊤\ ,\ λ\ \_\ →\ ⊤)}.

\begin{Shaded}
\begin{Highlighting}[]
\NormalTok{𝕪 }\OtherTok{:}\NormalTok{ Poly lzero lzero}
\NormalTok{𝕪 }\OtherTok{=} \OtherTok{(}\NormalTok{⊤ , }\OtherTok{λ} \OtherTok{\_} \OtherTok{→}\NormalTok{ ⊤}\OtherTok{)}
\end{Highlighting}
\end{Shaded}

The observant reader may note the striking similarity of the above-given
formula for a polynomial functor and the endofunctor on
\(\smset^{\mathcal{C}\op}\) defined in the previous section on
natural models. \dnote{Either here or there, let's say that $\texttt{Type}$ means small object in $\smset^{\cat{C}\op}$, or whatever is correct along those lines.}Indeed, this is no accident, given a type \texttt{𝓤} and
a function \texttt{u\ :\ 𝓤\ →\ Type} corresponding to a natural model as
described previously, we obtain the corresponding polynomial
\texttt{𝔲\ :\ Poly} as the pair \texttt{(𝓤\ ,\ u)}. Hence we can study
the structure of \texttt{𝓤} and \texttt{u} in terms of \texttt{𝔲}, and
this, as we shall see, allows for some significant simplifications in
the theory of natural models.

Given polynomial functors \(p = \sum_{a : A} \yon^{B(a)}\) and
\(q = \sum_{c : C} \yon^{D(c)}\), a natural transformation
\(p \Rightarrow q\) is equivalently given by the data of a
\emph{forward} map \texttt{f\ :\ A\ →\ B} and a \emph{backward} map
\texttt{g\ :\ (a\ :\ A)\ →\ D\ (f\ a)\ →\ B\ a}, as can be seen from the
following argument via Yoneda: \[
\begin{array}{rl}
& \int_{y \in \mathbf{Type}} \left( \sum_{a : A} \yon^{B(a)}  \right) \to \sum_{c : C} \yon^{D(c)}\\
\simeq & \prod_{a : A} \int_{y \in \mathbf{Type}} \yon^{B(a)} \to \sum_{c : C} \yon^{D(c)}\\
\simeq & \prod_{a : A} \sum_{c : C} B(a)^{D(c)}\\
\simeq & \sum_{f : A \to C} \prod_{a : A} B(a)^{D(f(c))}
\end{array}
\] We use the notation \(p \leftrightarrows q\) to denote the type of
natural transformations from \(p\) to \(q\) (aka \emph{lenses} from
\(p\) to \(q\)), which may be written in Agda as follows:

\begin{Shaded}
\begin{Highlighting}[]
\OtherTok{\_}\NormalTok{⇆}\OtherTok{\_} \OtherTok{:} \OtherTok{∀} \OtherTok{\{}\NormalTok{ℓ ℓ\textquotesingle{} κ κ\textquotesingle{}}\OtherTok{\}} \OtherTok{→}\NormalTok{ Poly ℓ κ }\OtherTok{→}\NormalTok{ Poly ℓ\textquotesingle{} κ\textquotesingle{} }\OtherTok{→}\NormalTok{ Type }\OtherTok{(}\NormalTok{ℓ ⊔ ℓ\textquotesingle{} ⊔ κ ⊔ κ\textquotesingle{}}\OtherTok{)}
\OtherTok{(}\NormalTok{A , B}\OtherTok{)}\NormalTok{ ⇆ }\OtherTok{(}\NormalTok{C , D}\OtherTok{)} \OtherTok{=}\NormalTok{ Σ }\OtherTok{(}\NormalTok{A }\OtherTok{→}\NormalTok{ C}\OtherTok{)} \OtherTok{(λ}\NormalTok{ f }\OtherTok{→} \OtherTok{(}\NormalTok{a }\OtherTok{:}\NormalTok{ A}\OtherTok{)} \OtherTok{→}\NormalTok{ D }\OtherTok{(}\NormalTok{f a}\OtherTok{)} \OtherTok{→}\NormalTok{ B a}\OtherTok{)}
\end{Highlighting}
\end{Shaded}

By application of function extensionality, we derive the following type
for proofs of equality between lenses:

\begin{Shaded}
\begin{Highlighting}[]
\NormalTok{EqLens }\OtherTok{:} \OtherTok{∀} \OtherTok{\{}\NormalTok{ℓ ℓ\textquotesingle{} κ κ\textquotesingle{}}\OtherTok{\}} \OtherTok{(}\NormalTok{p }\OtherTok{:}\NormalTok{ Poly ℓ κ}\OtherTok{)} \OtherTok{(}\NormalTok{q }\OtherTok{:}\NormalTok{ Poly ℓ\textquotesingle{} κ\textquotesingle{}}\OtherTok{)}
         \OtherTok{→} \OtherTok{(}\NormalTok{p ⇆ q}\OtherTok{)} \OtherTok{→} \OtherTok{(}\NormalTok{p ⇆ q}\OtherTok{)} \OtherTok{→} \DataTypeTok{Set} \OtherTok{(}\NormalTok{ℓ ⊔ ℓ\textquotesingle{} ⊔ κ ⊔ κ\textquotesingle{}}\OtherTok{)}
\NormalTok{EqLens }\OtherTok{(}\NormalTok{A , B}\OtherTok{)} \OtherTok{(}\NormalTok{C , D}\OtherTok{)} \OtherTok{(}\NormalTok{f , g}\OtherTok{)} \OtherTok{(}\NormalTok{f\textquotesingle{} , g\textquotesingle{}}\OtherTok{)} \OtherTok{=} 
  \OtherTok{(}\NormalTok{a }\OtherTok{:}\NormalTok{ A}\OtherTok{)} \OtherTok{→}\NormalTok{ Σ }\OtherTok{(}\NormalTok{f a ≡ f\textquotesingle{} a}\OtherTok{)} 
              \OtherTok{(λ}\NormalTok{ e }\OtherTok{→} \OtherTok{(}\NormalTok{b }\OtherTok{:}\NormalTok{ D }\OtherTok{(}\NormalTok{f a}\OtherTok{))} \OtherTok{→}\NormalTok{ g a b ≡ g\textquotesingle{} a }\OtherTok{(}\NormalTok{transp D e b}\OtherTok{))}
\end{Highlighting}
\end{Shaded}

For each polynomial \(p\), the correspnding \emph{identity} lens is
given by the following data:

\begin{Shaded}
\begin{Highlighting}[]
\NormalTok{id }\OtherTok{:} \OtherTok{∀} \OtherTok{\{}\NormalTok{ℓ κ}\OtherTok{\}} \OtherTok{(}\NormalTok{p }\OtherTok{:}\NormalTok{ Poly ℓ κ}\OtherTok{)} \OtherTok{→}\NormalTok{ p ⇆ p}
\NormalTok{id p }\OtherTok{=} \OtherTok{(} \OtherTok{(λ}\NormalTok{ a }\OtherTok{→}\NormalTok{ a}\OtherTok{)}\NormalTok{ , }\OtherTok{λ}\NormalTok{ a b }\OtherTok{→}\NormalTok{ b }\OtherTok{)}
\end{Highlighting}
\end{Shaded}

And given lenses \(p \leftrightarrows q\) and \(q \leftrightarrows r\),
their composition may be computed as follows:

\begin{Shaded}
\begin{Highlighting}[]
\NormalTok{comp }\OtherTok{:} \OtherTok{∀} \OtherTok{\{}\NormalTok{ℓ ℓ\textquotesingle{} ℓ\textquotesingle{}\textquotesingle{} κ κ\textquotesingle{} κ\textquotesingle{}\textquotesingle{}}\OtherTok{\}} 
       \OtherTok{→} \OtherTok{(}\NormalTok{p }\OtherTok{:}\NormalTok{ Poly ℓ κ}\OtherTok{)} \OtherTok{(}\NormalTok{q }\OtherTok{:}\NormalTok{ Poly ℓ\textquotesingle{} κ\textquotesingle{}}\OtherTok{)} \OtherTok{(}\NormalTok{r }\OtherTok{:}\NormalTok{ Poly ℓ\textquotesingle{}\textquotesingle{} κ\textquotesingle{}\textquotesingle{}}\OtherTok{)}
       \OtherTok{→}\NormalTok{ p ⇆ q }\OtherTok{→}\NormalTok{ q ⇆ r }\OtherTok{→}\NormalTok{ p ⇆ r }
\NormalTok{comp p q r }\OtherTok{(}\NormalTok{f , g}\OtherTok{)} \OtherTok{(}\NormalTok{h , k}\OtherTok{)} \OtherTok{=} 
    \OtherTok{(} \OtherTok{(λ}\NormalTok{ a }\OtherTok{→}\NormalTok{ h }\OtherTok{(}\NormalTok{f a}\OtherTok{))}\NormalTok{ , }\OtherTok{λ}\NormalTok{ a z }\OtherTok{→}\NormalTok{ g a }\OtherTok{(}\NormalTok{k }\OtherTok{(}\NormalTok{f a}\OtherTok{)}\NormalTok{ z}\OtherTok{)} \OtherTok{)}
\end{Highlighting}
\end{Shaded}\dnote{Use $(f, f^\sharp)$ and $(g,g^\sharp)$ here, as below?}

Hence we have a category \(\mathbf{Poly}\) of polynomial functors and
lenses between them. Our goal, then, is to show how the type-theoretic
structure of a natural model naturally arises from the structure of this
category. In fact, \(\mathbf{Poly}\) is replete with categorical
structures of all kinds, of which we now mention but a few.

\section{\texorpdfstring{The Vertical-Cartesian Factorization System on
\(\mathbf{Poly}\)}{The Vertical-Cartesian Factorization System on \textbackslash mathbf\{Poly\}}}\label{the-vertical-cartesian-factorization-system-on-mathbfpoly}

We say that a lens \texttt{(f\ ,\ f♯)\ :\ (A\ ,\ B)\ ⇆\ (C\ ,\ D)} is
\emph{vertical} if \texttt{f\ :\ A\ →\ C} is an equivalence, and
Cartesian if for every \texttt{a\ :\ A}, the map
\texttt{f♯\ a\ :\ D{[}f\ a{]}\ →\ B\ a} is an equivalence.

\begin{Shaded}
\begin{Highlighting}[]
\NormalTok{isVertical }\OtherTok{:} \OtherTok{∀} \OtherTok{\{}\NormalTok{ℓ ℓ\textquotesingle{} κ κ\textquotesingle{}}\OtherTok{\}} \OtherTok{(}\NormalTok{p }\OtherTok{:}\NormalTok{ Poly ℓ κ}\OtherTok{)} \OtherTok{(}\NormalTok{q }\OtherTok{:}\NormalTok{ Poly ℓ\textquotesingle{} κ\textquotesingle{}}\OtherTok{)}
             \OtherTok{→}\NormalTok{ p ⇆ q }\OtherTok{→} \DataTypeTok{Set} \OtherTok{(}\NormalTok{ℓ ⊔ ℓ\textquotesingle{}}\OtherTok{)}
\NormalTok{isVertical p q }\OtherTok{(}\NormalTok{f , f♯}\OtherTok{)} \OtherTok{=}\NormalTok{ isEquiv f}

\NormalTok{isCartesian }\OtherTok{:} \OtherTok{∀} \OtherTok{\{}\NormalTok{ℓ ℓ\textquotesingle{} κ κ\textquotesingle{}}\OtherTok{\}} \OtherTok{(}\NormalTok{p }\OtherTok{:}\NormalTok{ Poly ℓ κ}\OtherTok{)} \OtherTok{(}\NormalTok{q }\OtherTok{:}\NormalTok{ Poly ℓ\textquotesingle{} κ\textquotesingle{}}\OtherTok{)}
             \OtherTok{→}\NormalTok{ p ⇆ q }\OtherTok{→} \DataTypeTok{Set} \OtherTok{(}\NormalTok{ℓ ⊔ κ ⊔ κ\textquotesingle{}}\OtherTok{)}
\NormalTok{isCartesian }\OtherTok{(}\NormalTok{A , B}\OtherTok{)}\NormalTok{ q }\OtherTok{(}\NormalTok{f , f♯}\OtherTok{)} \OtherTok{=} \OtherTok{(}\NormalTok{a }\OtherTok{:}\NormalTok{ A}\OtherTok{)} \OtherTok{→}\NormalTok{ isEquiv }\OtherTok{(}\NormalTok{f♯ a}\OtherTok{)}
\end{Highlighting}
\end{Shaded}

Every lens \texttt{(A\ ,\ B)\ ⇆\ (C\ ,\ D)} can then be factored as a
vertical lens followed by a Cartesian lens:

\begin{Shaded}
\begin{Highlighting}[]
\NormalTok{vertfactor }\OtherTok{:} \OtherTok{∀} \OtherTok{\{}\NormalTok{ℓ ℓ\textquotesingle{} κ κ\textquotesingle{}}\OtherTok{\}} \OtherTok{(}\NormalTok{p }\OtherTok{:}\NormalTok{ Poly ℓ κ}\OtherTok{)} \OtherTok{(}\NormalTok{q }\OtherTok{:}\NormalTok{ Poly ℓ\textquotesingle{} κ\textquotesingle{}}\OtherTok{)}
             \OtherTok{→} \OtherTok{(}\NormalTok{f }\OtherTok{:}\NormalTok{ p ⇆ q}\OtherTok{)} \OtherTok{→}\NormalTok{ p ⇆ }\OtherTok{(}\NormalTok{fst p , }\OtherTok{λ}\NormalTok{ x }\OtherTok{→}\NormalTok{ snd q }\OtherTok{(}\NormalTok{fst f x}\OtherTok{))}
\NormalTok{vertfactor p q }\OtherTok{(}\NormalTok{f , f♯}\OtherTok{)} \OtherTok{=} \OtherTok{(λ}\NormalTok{ x }\OtherTok{→}\NormalTok{ x}\OtherTok{)}\NormalTok{ , }\OtherTok{(λ}\NormalTok{ a x }\OtherTok{→}\NormalTok{ f♯ a x}\OtherTok{)}

\NormalTok{vertfactorIsVert }\OtherTok{:} \OtherTok{∀} \OtherTok{\{}\NormalTok{ℓ ℓ\textquotesingle{} κ κ\textquotesingle{}}\OtherTok{\}} \OtherTok{(}\NormalTok{p }\OtherTok{:}\NormalTok{ Poly ℓ κ}\OtherTok{)} 
                   \OtherTok{→} \OtherTok{(}\NormalTok{q }\OtherTok{:}\NormalTok{ Poly ℓ\textquotesingle{} κ\textquotesingle{}}\OtherTok{)} \OtherTok{(}\NormalTok{f }\OtherTok{:}\NormalTok{ p ⇆ q}\OtherTok{)} 
                   \OtherTok{→}\NormalTok{ isVertical p }\OtherTok{(}\NormalTok{fst p , }\OtherTok{λ}\NormalTok{ x }\OtherTok{→}\NormalTok{ snd q }\OtherTok{(}\NormalTok{fst f x}\OtherTok{))}
                                \OtherTok{(}\NormalTok{vertfactor p q f}\OtherTok{)}
\NormalTok{vertfactorIsVert p q f }\OtherTok{=}\NormalTok{ idIsEquiv}

\NormalTok{cartfactor }\OtherTok{:} \OtherTok{∀} \OtherTok{\{}\NormalTok{ℓ ℓ\textquotesingle{} κ κ\textquotesingle{}}\OtherTok{\}} \OtherTok{(}\NormalTok{p }\OtherTok{:}\NormalTok{ Poly ℓ κ}\OtherTok{)} \OtherTok{(}\NormalTok{q }\OtherTok{:}\NormalTok{ Poly ℓ\textquotesingle{} κ\textquotesingle{}}\OtherTok{)}
             \OtherTok{→} \OtherTok{(}\NormalTok{f }\OtherTok{:}\NormalTok{ p ⇆ q}\OtherTok{)} \OtherTok{→} \OtherTok{(}\NormalTok{fst p , }\OtherTok{λ}\NormalTok{ x }\OtherTok{→}\NormalTok{ snd q }\OtherTok{(}\NormalTok{fst f x}\OtherTok{))}\NormalTok{ ⇆ q}
\NormalTok{cartfactor p q }\OtherTok{(}\NormalTok{f , f♯}\OtherTok{)} \OtherTok{=}\NormalTok{ f , }\OtherTok{λ}\NormalTok{ a x }\OtherTok{→}\NormalTok{ x}

\NormalTok{cartfactorIsCart }\OtherTok{:} \OtherTok{∀} \OtherTok{\{}\NormalTok{ℓ ℓ\textquotesingle{} κ κ\textquotesingle{}}\OtherTok{\}} \OtherTok{(}\NormalTok{p }\OtherTok{:}\NormalTok{ Poly ℓ κ}\OtherTok{)} 
                   \OtherTok{→} \OtherTok{(}\NormalTok{q }\OtherTok{:}\NormalTok{ Poly ℓ\textquotesingle{} κ\textquotesingle{}}\OtherTok{)} \OtherTok{(}\NormalTok{f }\OtherTok{:}\NormalTok{ p ⇆ q}\OtherTok{)} 
                   \OtherTok{→}\NormalTok{ isCartesian }\OtherTok{(}\NormalTok{fst p , }\OtherTok{λ}\NormalTok{ x }\OtherTok{→}\NormalTok{ snd q }\OtherTok{(}\NormalTok{fst f x}\OtherTok{))}\NormalTok{ q}
                                 \OtherTok{(}\NormalTok{cartfactor p q f}\OtherTok{)}
\NormalTok{cartfactorIsCart p q f x }\OtherTok{=}\NormalTok{ idIsEquiv}

\NormalTok{vertcartfactor≡ }\OtherTok{:} \OtherTok{∀} \OtherTok{\{}\NormalTok{ℓ ℓ\textquotesingle{} κ κ\textquotesingle{}}\OtherTok{\}} \OtherTok{(}\NormalTok{p }\OtherTok{:}\NormalTok{ Poly ℓ κ}\OtherTok{)} 
                  \OtherTok{→} \OtherTok{(}\NormalTok{q }\OtherTok{:}\NormalTok{ Poly ℓ\textquotesingle{} κ\textquotesingle{}}\OtherTok{)} \OtherTok{(}\NormalTok{f }\OtherTok{:}\NormalTok{ p ⇆ q}\OtherTok{)}
                  \OtherTok{→}\NormalTok{ EqLens p q f}
                           \OtherTok{(}\NormalTok{comp p }\OtherTok{(}\NormalTok{fst p , }\OtherTok{λ}\NormalTok{ x }\OtherTok{→}\NormalTok{ snd q }\OtherTok{(}\NormalTok{fst f x}\OtherTok{))}\NormalTok{ q}
                                 \OtherTok{(}\NormalTok{vertfactor p q f}\OtherTok{)}
                                 \OtherTok{(}\NormalTok{cartfactor p q f}\OtherTok{))}
\NormalTok{vertcartfactor≡ p q f a }\OtherTok{=}\NormalTok{ refl , }\OtherTok{(λ}\NormalTok{ b }\OtherTok{→}\NormalTok{ refl}\OtherTok{)}
\end{Highlighting}
\end{Shaded}

Of these two classes of morphisms in \(\mathbf{Poly}\), it is
\emph{Cartesian} lenses that shall be of principal interest to us. If we
view a polynomial \texttt{p\ =\ (A\ ,\ B)} as an \texttt{A}-indexed
family of types, given by \texttt{B}, then given a lens
\texttt{(f\ ,\ f♯)\ :\ p\ ⇆\ 𝔲} to the universe, we can think of the map
\texttt{f♯\ a\ :\ u\ (f\ a)\ →\ B\ a}, as an \emph{elimination form} for
the type \texttt{u\ (f\ a)}, i.e.~a way of \emph{using} elements of the
type \texttt{u\ (f\ a)}. If we then ask that \texttt{(f\ ,\ f♯)}
is Cartesian, this implies that the type \texttt{u\ (f\ a)} is completely
characterized (up to equivalence) by this elimination form, and in this
sense, that \texttt{𝔲} \emph{contains} the type \texttt{B\ a}, for all
\texttt{a\ :\ A}.%
\footnote{Those familiar with type theory may recognize
  this practice of defining types in terms of their elimination forms as
  characteristic of so-called \emph{negative} types (in opposition to
  \emph{positive} types, which are characterized by their introduction
  forms). Indeed, there are good reasons for this, having to do with the
  fact that negative types are equivalently those whose universal
  property is given by a \emph{representable} functor, rather than a
  \emph{co-representable} functor, which reflects the fact that natural
  models are defined in terms of \emph{presheaves} on a category of
  contexts, rather than \emph{co-presheaves.}}
We can therefore use Cartesian lenses to detect which types are
contained in a natural model \texttt{𝔲}.

A further fact about Cartesian lenses is that they are closed under
identity and composition, as a direct consequence of the closure of
equivalences under identity and composition:

\begin{Shaded}
\begin{Highlighting}[]
\NormalTok{idCart }\OtherTok{:} \OtherTok{∀} \OtherTok{\{}\NormalTok{ℓ κ}\OtherTok{\}} \OtherTok{(}\NormalTok{p }\OtherTok{:}\NormalTok{ Poly ℓ κ}\OtherTok{)}
         \OtherTok{→}\NormalTok{ isCartesian p p }\OtherTok{(}\NormalTok{id p}\OtherTok{)}
\NormalTok{idCart p a }\OtherTok{=}\NormalTok{ idIsEquiv}

\NormalTok{compCartesian }\OtherTok{:} \OtherTok{∀} \OtherTok{\{}\NormalTok{ℓ ℓ\textquotesingle{} ℓ\textquotesingle{}\textquotesingle{} κ κ\textquotesingle{} κ\textquotesingle{}\textquotesingle{}}\OtherTok{\}}
                \OtherTok{→} \OtherTok{(}\NormalTok{p }\OtherTok{:}\NormalTok{ Poly ℓ κ}\OtherTok{)} \OtherTok{(}\NormalTok{q }\OtherTok{:}\NormalTok{ Poly ℓ\textquotesingle{} κ\textquotesingle{}}\OtherTok{)} \OtherTok{(}\NormalTok{r }\OtherTok{:}\NormalTok{ Poly ℓ\textquotesingle{}\textquotesingle{} κ\textquotesingle{}\textquotesingle{}}\OtherTok{)}
                \OtherTok{→} \OtherTok{(}\NormalTok{f }\OtherTok{:}\NormalTok{ p ⇆ q}\OtherTok{)} \OtherTok{(}\NormalTok{g }\OtherTok{:}\NormalTok{ q ⇆ r}\OtherTok{)}
                \OtherTok{→}\NormalTok{ isCartesian p q f }\OtherTok{→}\NormalTok{ isCartesian q r g }
                \OtherTok{→}\NormalTok{ isCartesian p r }\OtherTok{(}\NormalTok{comp p q r f g}\OtherTok{)}
\NormalTok{compCartesian p q r f g cf cg a }\OtherTok{=} 
\NormalTok{    compIsEquiv }\OtherTok{(}\NormalTok{snd f a}\OtherTok{)} \OtherTok{(}\NormalTok{snd g }\OtherTok{(}\NormalTok{fst f a}\OtherTok{))} \OtherTok{(}\NormalTok{cf a}\OtherTok{)} \OtherTok{(}\NormalTok{cg }\OtherTok{(}\NormalTok{fst f a}\OtherTok{))}
\end{Highlighting}
\end{Shaded}

Hence there is a category \(\mathbf{Poly^{Cart}}\) defined as the wide
subcategory of \(\mathbf{Poly}\) whose morphisms are precisely Cartesian
lenses. As we shall see, much of the categorical structure of natural
models qua polynomial functors can be derived from the subtle interplay
between \(\mathbf{Poly^{Cart}}\) and \(\mathbf{Poly}\).

\subsection{\texorpdfstring{Epi-Mono Factorization on
\(\mathbf{Poly^{Cart}}\)}{Epi-Mono Factorization on \textbackslash mathbf\{Poly\^{}\{Cart\}\}}}\label{epi-mono-factorization-on-mathbfpolycart}

In fact, \(\mathbf{Poly^{Cart}}\) itself inherits a factorization system
from the epi-mono factorization on types considered previously.

Define a Cartesian lens \texttt{(f\ ,\ f♯)\ :\ p\ ⇆\ q} to be a
\emph{Cartesian embedding} if \texttt{f} is a monomorphism, and a
\emph{Cartesian surjection} if \texttt{f} is an epimorphism.

\begin{Shaded}
\begin{Highlighting}[]
\NormalTok{isCartesianEmbedding }\OtherTok{:} \OtherTok{∀} \OtherTok{\{}\NormalTok{ℓ ℓ\textquotesingle{} κ κ\textquotesingle{}}\OtherTok{\}} \OtherTok{(}\NormalTok{p }\OtherTok{:}\NormalTok{ Poly ℓ κ}\OtherTok{)} \OtherTok{(}\NormalTok{q }\OtherTok{:}\NormalTok{ Poly ℓ\textquotesingle{} κ\textquotesingle{}}\OtherTok{)}
                       \OtherTok{→} \OtherTok{(}\NormalTok{f }\OtherTok{:}\NormalTok{ p ⇆ q}\OtherTok{)} \OtherTok{→}\NormalTok{ isCartesian p q f }\OtherTok{→} \DataTypeTok{Set} \OtherTok{(}\NormalTok{ℓ ⊔ ℓ\textquotesingle{}}\OtherTok{)}
\NormalTok{isCartesianEmbedding p q }\OtherTok{(}\NormalTok{f , f♯}\OtherTok{)}\NormalTok{ cf }\OtherTok{=}\NormalTok{ isMono f}

\NormalTok{isCartesianSurjection }\OtherTok{:} \OtherTok{∀} \OtherTok{\{}\NormalTok{ℓ ℓ\textquotesingle{} κ κ\textquotesingle{}}\OtherTok{\}} \OtherTok{(}\NormalTok{p }\OtherTok{:}\NormalTok{ Poly ℓ κ}\OtherTok{)} \OtherTok{(}\NormalTok{q }\OtherTok{:}\NormalTok{ Poly ℓ\textquotesingle{} κ\textquotesingle{}}\OtherTok{)}
                        \OtherTok{→} \OtherTok{(}\NormalTok{f }\OtherTok{:}\NormalTok{ p ⇆ q}\OtherTok{)} \OtherTok{→}\NormalTok{ isCartesian p q f }\OtherTok{→} \DataTypeTok{Set}\NormalTok{ ℓ\textquotesingle{}}
\NormalTok{isCartesianSurjection p q }\OtherTok{(}\NormalTok{f , f♯}\OtherTok{)}\NormalTok{ cf }\OtherTok{=}\NormalTok{ isEpi f}
\end{Highlighting}
\end{Shaded}

Then every Cartesian lens can be factored into a Cartesian surjection
followed by a Cartesian embedding.

\begin{Shaded}
\begin{Highlighting}[]
\NormalTok{factorcart1 }\OtherTok{:} \OtherTok{∀} \OtherTok{\{}\NormalTok{ℓ ℓ\textquotesingle{} κ κ\textquotesingle{}}\OtherTok{\}} \OtherTok{(}\NormalTok{p }\OtherTok{:}\NormalTok{ Poly ℓ κ}\OtherTok{)} \OtherTok{(}\NormalTok{q }\OtherTok{:}\NormalTok{ Poly ℓ\textquotesingle{} κ\textquotesingle{}}\OtherTok{)}
              \OtherTok{→} \OtherTok{(}\NormalTok{f }\OtherTok{:}\NormalTok{ p ⇆ q}\OtherTok{)} \OtherTok{→}\NormalTok{ isCartesian p q f}
              \OtherTok{→}\NormalTok{ p ⇆ }\OtherTok{(}\NormalTok{Im }\OtherTok{(}\NormalTok{fst f}\OtherTok{)}\NormalTok{ , }\OtherTok{λ} \OtherTok{(}\NormalTok{x , }\OtherTok{\_)} \OtherTok{→}\NormalTok{ snd q x}\OtherTok{)}
\NormalTok{factorcart1 p q }\OtherTok{(}\NormalTok{f , f♯}\OtherTok{)}\NormalTok{ cf }\OtherTok{=} 
    \OtherTok{(}\NormalTok{factor1 f}\OtherTok{)}\NormalTok{ , f♯}

\NormalTok{factorcart1IsCart }\OtherTok{:} \OtherTok{∀} \OtherTok{\{}\NormalTok{ℓ ℓ\textquotesingle{} κ κ\textquotesingle{}}\OtherTok{\}} \OtherTok{(}\NormalTok{p }\OtherTok{:}\NormalTok{ Poly ℓ κ}\OtherTok{)} \OtherTok{(}\NormalTok{q }\OtherTok{:}\NormalTok{ Poly ℓ\textquotesingle{} κ\textquotesingle{}}\OtherTok{)}
                    \OtherTok{→} \OtherTok{(}\NormalTok{f }\OtherTok{:}\NormalTok{ p ⇆ q}\OtherTok{)} \OtherTok{(}\NormalTok{cf }\OtherTok{:}\NormalTok{ isCartesian p q f}\OtherTok{)}
                    \OtherTok{→}\NormalTok{ isCartesian p }
                                  \OtherTok{(}\NormalTok{Im }\OtherTok{(}\NormalTok{fst f}\OtherTok{)}\NormalTok{ , }\OtherTok{λ} \OtherTok{(}\NormalTok{x , }\OtherTok{\_)} \OtherTok{→}\NormalTok{ snd q x}\OtherTok{)}
                                  \OtherTok{(}\NormalTok{factorcart1 p q f cf}\OtherTok{)}
\NormalTok{factorcart1IsCart p q }\OtherTok{(}\NormalTok{f , f♯}\OtherTok{)}\NormalTok{ cf }\OtherTok{=}\NormalTok{ cf}

\NormalTok{factorcart1IsEpi }\OtherTok{:} \OtherTok{∀} \OtherTok{\{}\NormalTok{ℓ ℓ\textquotesingle{} κ κ\textquotesingle{}}\OtherTok{\}} \OtherTok{(}\NormalTok{p }\OtherTok{:}\NormalTok{ Poly ℓ κ}\OtherTok{)} \OtherTok{(}\NormalTok{q }\OtherTok{:}\NormalTok{ Poly ℓ\textquotesingle{} κ\textquotesingle{}}\OtherTok{)}
                   \OtherTok{→} \OtherTok{(}\NormalTok{f }\OtherTok{:}\NormalTok{ p ⇆ q}\OtherTok{)} \OtherTok{(}\NormalTok{cf }\OtherTok{:}\NormalTok{ isCartesian p q f}\OtherTok{)}
                   \OtherTok{→}\NormalTok{ isCartesianSurjection p }
                        \OtherTok{(}\NormalTok{Im }\OtherTok{(}\NormalTok{fst f}\OtherTok{)}\NormalTok{ , }\OtherTok{λ} \OtherTok{(}\NormalTok{x , }\OtherTok{\_)} \OtherTok{→}\NormalTok{ snd q x}\OtherTok{)}
                        \OtherTok{(}\NormalTok{factorcart1 p q f cf}\OtherTok{)}
                        \OtherTok{(}\NormalTok{factorcart1IsCart p q f cf}\OtherTok{)}
\NormalTok{factorcart1IsEpi p q }\OtherTok{(}\NormalTok{f , f♯}\OtherTok{)}\NormalTok{ cf }\OtherTok{=}\NormalTok{ factor1IsEpi f}

\NormalTok{factorcart2 }\OtherTok{:} \OtherTok{∀} \OtherTok{\{}\NormalTok{ℓ ℓ\textquotesingle{} κ κ\textquotesingle{}}\OtherTok{\}} \OtherTok{(}\NormalTok{p }\OtherTok{:}\NormalTok{ Poly ℓ κ}\OtherTok{)} \OtherTok{(}\NormalTok{q }\OtherTok{:}\NormalTok{ Poly ℓ\textquotesingle{} κ\textquotesingle{}}\OtherTok{)}
              \OtherTok{→} \OtherTok{(}\NormalTok{f }\OtherTok{:}\NormalTok{ p ⇆ q}\OtherTok{)} \OtherTok{→}\NormalTok{ isCartesian p q f}
              \OtherTok{→} \OtherTok{(}\NormalTok{Im }\OtherTok{(}\NormalTok{fst f}\OtherTok{)}\NormalTok{ , }\OtherTok{λ} \OtherTok{(}\NormalTok{x , }\OtherTok{\_)} \OtherTok{→}\NormalTok{ snd q x}\OtherTok{)}\NormalTok{ ⇆ q}
\NormalTok{factorcart2 p q }\OtherTok{(}\NormalTok{f , f♯}\OtherTok{)}\NormalTok{ cf }\OtherTok{=} \OtherTok{(}\NormalTok{factor2 f}\OtherTok{)}\NormalTok{ , }\OtherTok{λ} \OtherTok{(}\NormalTok{x , }\OtherTok{\_)}\NormalTok{ y }\OtherTok{→}\NormalTok{ y}

\NormalTok{factorcart2IsCart }\OtherTok{:} \OtherTok{∀} \OtherTok{\{}\NormalTok{ℓ ℓ\textquotesingle{} κ κ\textquotesingle{}}\OtherTok{\}} \OtherTok{(}\NormalTok{p }\OtherTok{:}\NormalTok{ Poly ℓ κ}\OtherTok{)} \OtherTok{(}\NormalTok{q }\OtherTok{:}\NormalTok{ Poly ℓ\textquotesingle{} κ\textquotesingle{}}\OtherTok{)}
                    \OtherTok{→} \OtherTok{(}\NormalTok{f }\OtherTok{:}\NormalTok{ p ⇆ q}\OtherTok{)} \OtherTok{(}\NormalTok{cf }\OtherTok{:}\NormalTok{ isCartesian p q f}\OtherTok{)}
                    \OtherTok{→}\NormalTok{ isCartesian }\OtherTok{(}\NormalTok{Im }\OtherTok{(}\NormalTok{fst f}\OtherTok{)}\NormalTok{ , }\OtherTok{λ} \OtherTok{(}\NormalTok{x , }\OtherTok{\_)} \OtherTok{→}\NormalTok{ snd q x}\OtherTok{)}\NormalTok{ q}
                                  \OtherTok{(}\NormalTok{factorcart2 p q f cf}\OtherTok{)}
\NormalTok{factorcart2IsCart p q }\OtherTok{(}\NormalTok{f , f♯}\OtherTok{)}\NormalTok{ cf x }\OtherTok{=}\NormalTok{ idIsEquiv}

\NormalTok{factorcart2IsMono }\OtherTok{:} \OtherTok{∀} \OtherTok{\{}\NormalTok{ℓ ℓ\textquotesingle{} κ κ\textquotesingle{}}\OtherTok{\}} \OtherTok{(}\NormalTok{p }\OtherTok{:}\NormalTok{ Poly ℓ κ}\OtherTok{)} \OtherTok{(}\NormalTok{q }\OtherTok{:}\NormalTok{ Poly ℓ\textquotesingle{} κ\textquotesingle{}}\OtherTok{)}
                    \OtherTok{→} \OtherTok{(}\NormalTok{f }\OtherTok{:}\NormalTok{ p ⇆ q}\OtherTok{)} \OtherTok{(}\NormalTok{cf }\OtherTok{:}\NormalTok{ isCartesian p q f}\OtherTok{)}
                    \OtherTok{→}\NormalTok{ isCartesianEmbedding}
                        \OtherTok{(}\NormalTok{Im }\OtherTok{(}\NormalTok{fst f}\OtherTok{)}\NormalTok{ , }\OtherTok{λ} \OtherTok{(}\NormalTok{x , }\OtherTok{\_)} \OtherTok{→}\NormalTok{ snd q x}\OtherTok{)}\NormalTok{ q}
                        \OtherTok{(}\NormalTok{factorcart2 p q f cf}\OtherTok{)}
                        \OtherTok{(}\NormalTok{factorcart2IsCart p q f cf}\OtherTok{)}
\NormalTok{factorcart2IsMono p q }\OtherTok{(}\NormalTok{f , f♯}\OtherTok{)}\NormalTok{ cf }\OtherTok{=}\NormalTok{ factor2IsMono f}

\NormalTok{factorcart≡ }\OtherTok{:} \OtherTok{∀} \OtherTok{\{}\NormalTok{ℓ ℓ\textquotesingle{} κ κ\textquotesingle{}}\OtherTok{\}} \OtherTok{(}\NormalTok{p }\OtherTok{:}\NormalTok{ Poly ℓ κ}\OtherTok{)} \OtherTok{(}\NormalTok{q }\OtherTok{:}\NormalTok{ Poly ℓ\textquotesingle{} κ\textquotesingle{}}\OtherTok{)}
              \OtherTok{→} \OtherTok{(}\NormalTok{f }\OtherTok{:}\NormalTok{ p ⇆ q}\OtherTok{)} \OtherTok{(}\NormalTok{cf }\OtherTok{:}\NormalTok{ isCartesian p q f}\OtherTok{)}
              \OtherTok{→}\NormalTok{ EqLens p q f}
                       \OtherTok{(}\NormalTok{comp p }\OtherTok{(}\NormalTok{Im }\OtherTok{(}\NormalTok{fst f}\OtherTok{)}\NormalTok{ , }\OtherTok{λ} \OtherTok{(}\NormalTok{x , }\OtherTok{\_)} \OtherTok{→}\NormalTok{ snd q x}\OtherTok{)}\NormalTok{ q}
                             \OtherTok{(}\NormalTok{factorcart1 p q f cf}\OtherTok{)}
                             \OtherTok{(}\NormalTok{factorcart2 p q f cf}\OtherTok{))}
\NormalTok{factorcart≡ p q f cf x }\OtherTok{=}\NormalTok{ refl , }\OtherTok{λ}\NormalTok{ y }\OtherTok{→}\NormalTok{ refl}
\end{Highlighting}
\end{Shaded}

\section{Composition of Polynomial
Functors}\label{composition-of-polynomial-functors}

As endofunctors on \(\mathbf{Type}\), polynomial functors may
straightforwardly be composed. To show that the resulting composite is
itself (equivalent to) a polynomial functor, we can reason via the
following chain of equivalences: given polynomials \texttt{(A\ ,\ B)}
and \texttt{(C\ ,\ D)}, their composite, evaluated at a type \texttt{y}
is \[
\begin{array}{rl}
& \sum_{a : A} \prod_{b : B(a)} \sum_{c : C} \yon^{D(c)}\\
\simeq & \sum_{a : A} \sum_{f : B(a) \to C} \prod_{b : B(a)} \yon^{D(f(b))}\\
\simeq & \sum_{(a , f) : \sum_{a : A} C^{B(a)}} \yon^{\sum_{b : B(a)} D(f(b))}
\end{array}
\] This then defines a monoidal product \(◃\) on \(\mathbf{Poly}\) with
monoidal unit given by the identity functor \texttt{𝕪}:

\begin{Shaded}
\begin{Highlighting}[]
\OtherTok{\_}\NormalTok{◃}\OtherTok{\_} \OtherTok{:} \OtherTok{∀} \OtherTok{\{}\NormalTok{ℓ ℓ\textquotesingle{} κ κ\textquotesingle{}}\OtherTok{\}} \OtherTok{→}\NormalTok{ Poly ℓ κ }\OtherTok{→}\NormalTok{ Poly ℓ\textquotesingle{} κ\textquotesingle{} }\OtherTok{→}\NormalTok{ Poly }\OtherTok{(}\NormalTok{ℓ ⊔ κ ⊔ ℓ\textquotesingle{}}\OtherTok{)} \OtherTok{(}\NormalTok{κ ⊔ κ\textquotesingle{}}\OtherTok{)}
\OtherTok{(}\NormalTok{A , B}\OtherTok{)}\NormalTok{ ◃ }\OtherTok{(}\NormalTok{C , D}\OtherTok{)} \OtherTok{=} \OtherTok{(}\NormalTok{Σ A }\OtherTok{(λ}\NormalTok{ a }\OtherTok{→}\NormalTok{ B a }\OtherTok{→}\NormalTok{ C}\OtherTok{)}\NormalTok{ , }\OtherTok{λ} \OtherTok{(}\NormalTok{a , f}\OtherTok{)} \OtherTok{→}\NormalTok{ Σ }\OtherTok{(}\NormalTok{B a}\OtherTok{)} \OtherTok{(λ}\NormalTok{ b }\OtherTok{→}\NormalTok{ D }\OtherTok{(}\NormalTok{f b}\OtherTok{)))}

\NormalTok{◃Lens }\OtherTok{:} \OtherTok{∀} \OtherTok{\{}\NormalTok{ℓ ℓ\textquotesingle{} ℓ\textquotesingle{}\textquotesingle{} ℓ\textquotesingle{}\textquotesingle{}\textquotesingle{} κ κ\textquotesingle{} κ\textquotesingle{}\textquotesingle{} κ\textquotesingle{}\textquotesingle{}\textquotesingle{}}\OtherTok{\}}
        \OtherTok{→} \OtherTok{(}\NormalTok{p }\OtherTok{:}\NormalTok{ Poly ℓ κ}\OtherTok{)} \OtherTok{(}\NormalTok{p\textquotesingle{} }\OtherTok{:}\NormalTok{ Poly ℓ\textquotesingle{} κ\textquotesingle{}}\OtherTok{)} 
        \OtherTok{→} \OtherTok{(}\NormalTok{q }\OtherTok{:}\NormalTok{ Poly ℓ\textquotesingle{}\textquotesingle{} κ\textquotesingle{}\textquotesingle{}}\OtherTok{)} \OtherTok{(}\NormalTok{q\textquotesingle{} }\OtherTok{:}\NormalTok{ Poly ℓ\textquotesingle{}\textquotesingle{}\textquotesingle{} κ\textquotesingle{}\textquotesingle{}\textquotesingle{}}\OtherTok{)}
        \OtherTok{→}\NormalTok{ p ⇆ p\textquotesingle{} }\OtherTok{→}\NormalTok{ q ⇆ q\textquotesingle{} }\OtherTok{→} \OtherTok{(}\NormalTok{p ◃ q}\OtherTok{)}\NormalTok{ ⇆ }\OtherTok{(}\NormalTok{p\textquotesingle{} ◃ q\textquotesingle{}}\OtherTok{)}
\NormalTok{◃Lens p p\textquotesingle{} q q\textquotesingle{} }\OtherTok{(}\NormalTok{f , g}\OtherTok{)} \OtherTok{(}\NormalTok{h , k}\OtherTok{)} \OtherTok{=}
    \OtherTok{((λ} \OtherTok{(}\NormalTok{a , c}\OtherTok{)} \OtherTok{→} \OtherTok{(}\NormalTok{f a , }\OtherTok{λ}\NormalTok{ b\textquotesingle{} }\OtherTok{→}\NormalTok{ h }\OtherTok{(}\NormalTok{c }\OtherTok{(}\NormalTok{g a b\textquotesingle{}}\OtherTok{))))}
\NormalTok{    , }\OtherTok{λ} \OtherTok{(}\NormalTok{a , c}\OtherTok{)} \OtherTok{(}\NormalTok{b\textquotesingle{} , d\textquotesingle{}}\OtherTok{)} \OtherTok{→} \OtherTok{((}\NormalTok{g a b\textquotesingle{}}\OtherTok{)}\NormalTok{ , k }\OtherTok{(}\NormalTok{c }\OtherTok{(}\NormalTok{g a b\textquotesingle{}}\OtherTok{))}\NormalTok{ d\textquotesingle{}}\OtherTok{))}
\end{Highlighting}
\end{Shaded}
where \texttt{◃Lens} is the action of \texttt{◃} on lenses.

By construction, the existence of a Cartesian lens
\texttt{(σ\ ,\ σ♯)\ :\ 𝔲\ ◃\ 𝔲\ ⇆\ 𝔲} effectively shows that \texttt{𝔲}
is closed under \texttt{Σ}-types, since:

\begin{itemize}
\tightlist
\item
  \texttt{σ} maps a pair (A , B) consisting of \texttt{A\ :\ 𝓤} and
  \texttt{B\ :\ (u\ A)\ →\ 𝓤} to a term \texttt{σ(A,B)} representing the
  \texttt{Σ} type. This corresponds to the type formation rule
  \[ \inferrule{\Gamma \vdash A : \mathsf{Type}\\ \Gamma, x : A \vdash B[x] ~ \mathsf{Type}}{\Gamma \vdash \Sigma x : A . B[x] ~ \mathsf{Type}} \]\dnote{I like the $B[x]$-bracket notation, but didn't you use $B(x)$-paren notation above? Or is there a subtle distinction coming up that I've missed?}
\item
  For all \texttt{(A\ ,\ B)} as above, \texttt{σ♯\ (A\ ,\ B)} takes a
  term of type \texttt{σ\ (A\ ,\ B)} and yields a term
  \texttt{fst\ (σ♯\ (A\ ,\ B))\ :\ A} along with a term
  \texttt{snd\ (σ♯\ (A\ ,\ B))\ :\ B\ (fst\ (σ♯\ (A\ ,\ B)))},
  corresponding to the elimination rules \[
  \inferrule{\Gamma \vdash p : \Sigma x : A . B[x]}{\Gamma \vdash \pi_1(p) : A} \quad \inferrule{\Gamma \vdash p : \Sigma x : A . B[x]}{\Gamma \vdash \pi_2(p) : B[\pi_1(p)]} \]
\item
  The fact that \texttt{σ♯\ (A\ ,\ B)} is an equivalence implies it
  has an inverse
  \texttt{σ♯⁻¹\ (A\ ,\ B)\ :\ Σ\ (u\ A)\ (λ\ x\ →\ u\ (B\ x))\ →\ u\ (σ\ (A\ ,\ B))},
  which takes a pair of terms to a term of the corresponding pair type,
  and thus corresponds to the introduction rule
  \[ \inferrule{\Gamma \vdash a : A\\ \Gamma \vdash b : B[a]}{\Gamma \vdash (a , b) : \Sigma x : A . B[x]} \]
\item
  The fact that \(σ♯⁻¹ (A , B)\) is both a left and a right inverse to
  \(σ♯\) then implies the usual \(β\) and \(η\) laws for dependent pair
  types
  \[ \pi_1(a , b) = a \quad \pi_2(a , b) = b \quad p = (\pi_1(p) , \pi_2(p)) \]
\end{itemize}

Similarly, the existence of a Cartesian lens \((η , η♯) : 𝕪\ ⇆\ 𝔲\)
implies that \(𝔲\) contains (a type equivalent to) the unit type, in
that:

\begin{itemize}
\tightlist
\item
  There is an element \texttt{η\ tt\ :\ 𝓤} which represents the unit
  type. This corresponds to the type formation rule
  \[ \inferrule{~}{\Gamma \vdash \top : \mathsf{Type}}\]
\item
  The ``elimination rule'' \texttt{η♯\ tt\ :\ u(η\ tt)\ →\ ⊤}, applied
  to an element \texttt{x\ :\ u(η\ tt)} is trivial, in that it simply
  discards \texttt{x}. This corresponds to the fact that, in the
  ordinary type-theoretic presentation, \(\top\) does not have an
  elimination rule.
\item
  However, since this trivial elimination rule has an inverse
  \texttt{η♯⁻¹\ tt\ :\ ⊤\ →\ u\ (η\ tt)}, it follows that there is a
  (unique) element \texttt{η♯⁻¹\ tt\ tt\ :\ u\ (η\ tt)}, which
  corresponds to the introduction rule for \(\top\):
  \[\inferrule{~}{\Gamma \vdash \mathsf{tt} : \top}\]
\item
  Moreover, the uniqueness of this element corresponds to the
  \(\eta\)-law for \(\top\):
  \[\frac{\Gamma \vdash x : \top}{\Gamma \vdash x = \mathsf{tt}}\]
\end{itemize}

But then, what sorts of laws can we expect Cartesian lenses as above to
obey, and is the existence of such a lens all that is needed to ensure
that the natural model \(𝔲\) has dependent pair types in the original
sense of Awodey \& Newstead's definition in terms of Cartesian
(pseudo)monads, or is some further data required? And what about
\texttt{Π} types, or other type formers? To answer these questions, we
will need to study the structure of \texttt{◃}, along with some closely
related functors, in a bit more detail. In particular, we shall see that
the structure of \texttt{◃} as a monoidal product on \(\mathbf{Poly}\)
reflects many of the basic identities one expects to hold of \texttt{Σ}
types.

For instance, the associativity of \texttt{◃} corresponds to the
associativity of \texttt{Σ}-types,

\begin{Shaded}
\begin{Highlighting}[]
\NormalTok{◃assoc }\OtherTok{:} \OtherTok{∀} \OtherTok{\{}\NormalTok{ℓ ℓ\textquotesingle{} ℓ\textquotesingle{}\textquotesingle{} κ κ\textquotesingle{} κ\textquotesingle{}\textquotesingle{}}\OtherTok{\}}
         \OtherTok{→} \OtherTok{(}\NormalTok{p }\OtherTok{:}\NormalTok{ Poly ℓ κ}\OtherTok{)} \OtherTok{(}\NormalTok{q }\OtherTok{:}\NormalTok{ Poly ℓ\textquotesingle{} κ\textquotesingle{}}\OtherTok{)} \OtherTok{(}\NormalTok{r }\OtherTok{:}\NormalTok{ Poly ℓ\textquotesingle{}\textquotesingle{} κ\textquotesingle{}\textquotesingle{}}\OtherTok{)}
         \OtherTok{→} \OtherTok{((}\NormalTok{p ◃ q}\OtherTok{)}\NormalTok{ ◃ r}\OtherTok{)}\NormalTok{ ⇆ }\OtherTok{(}\NormalTok{p ◃ }\OtherTok{(}\NormalTok{q ◃ r}\OtherTok{))}
\NormalTok{◃assoc p q r }\OtherTok{=} 
    \OtherTok{((λ} \OtherTok{((}\NormalTok{a , f}\OtherTok{)}\NormalTok{ , g}\OtherTok{)} \OtherTok{→} \OtherTok{(}\NormalTok{a , }\OtherTok{(λ}\NormalTok{ b }\OtherTok{→} \OtherTok{(}\NormalTok{f b , }\OtherTok{λ}\NormalTok{ d }\OtherTok{→}\NormalTok{ g }\OtherTok{(}\NormalTok{b , d}\OtherTok{)))))} 
\NormalTok{    , }\OtherTok{λ} \OtherTok{((}\NormalTok{a , f}\OtherTok{)}\NormalTok{ , g}\OtherTok{)} \OtherTok{(}\NormalTok{b , }\OtherTok{(}\NormalTok{d , x}\OtherTok{))} \OtherTok{→} \OtherTok{((}\NormalTok{b , d}\OtherTok{)}\NormalTok{ , x}\OtherTok{))}

\NormalTok{◃assocCart }\OtherTok{:} \OtherTok{∀} \OtherTok{\{}\NormalTok{ℓ ℓ\textquotesingle{} ℓ\textquotesingle{}\textquotesingle{} κ κ\textquotesingle{} κ\textquotesingle{}\textquotesingle{}}\OtherTok{\}}
             \OtherTok{→} \OtherTok{(}\NormalTok{p }\OtherTok{:}\NormalTok{ Poly ℓ κ}\OtherTok{)} \OtherTok{(}\NormalTok{q }\OtherTok{:}\NormalTok{ Poly ℓ\textquotesingle{} κ\textquotesingle{}}\OtherTok{)} \OtherTok{(}\NormalTok{r }\OtherTok{:}\NormalTok{ Poly ℓ\textquotesingle{}\textquotesingle{} κ\textquotesingle{}\textquotesingle{}}\OtherTok{)}
             \OtherTok{→}\NormalTok{ isCartesian }\OtherTok{((}\NormalTok{p ◃ q}\OtherTok{)}\NormalTok{ ◃ r}\OtherTok{)} \OtherTok{(}\NormalTok{p ◃ }\OtherTok{(}\NormalTok{q ◃ r}\OtherTok{))} \OtherTok{(}\NormalTok{◃assoc p q r}\OtherTok{)}
\NormalTok{◃assocCart p q r }\OtherTok{(}\NormalTok{a , f}\OtherTok{)} \OtherTok{=} 
\NormalTok{    Iso→isEquiv }\OtherTok{(} \OtherTok{(λ} \OtherTok{((}\NormalTok{b , d}\OtherTok{)}\NormalTok{ , x}\OtherTok{)} \OtherTok{→}\NormalTok{ b , d , x}\OtherTok{)}
\NormalTok{                , }\OtherTok{(} \OtherTok{(λ} \OtherTok{(}\NormalTok{b , d , x}\OtherTok{)} \OtherTok{→}\NormalTok{ refl}\OtherTok{)} 
\NormalTok{                  , }\OtherTok{λ} \OtherTok{((}\NormalTok{b , d}\OtherTok{)}\NormalTok{ , x}\OtherTok{)} \OtherTok{→}\NormalTok{ refl}\OtherTok{))}
\end{Highlighting}
\end{Shaded}
while the left and right unitors of \texttt{◃} correspond to the fact
that \texttt{⊤} is both a left and a right unit for \texttt{Σ}-types.\dnote{It might be good to add the corresponding type-theoretic consequences here?}

\begin{Shaded}
\begin{Highlighting}[]
\NormalTok{◃unitl }\OtherTok{:} \OtherTok{∀} \OtherTok{\{}\NormalTok{ℓ κ}\OtherTok{\}} \OtherTok{(}\NormalTok{p }\OtherTok{:}\NormalTok{ Poly ℓ κ}\OtherTok{)} \OtherTok{→} \OtherTok{(}\NormalTok{𝕪 ◃ p}\OtherTok{)}\NormalTok{ ⇆ p}
\NormalTok{◃unitl p }\OtherTok{=} \OtherTok{(λ} \OtherTok{(}\NormalTok{tt , a}\OtherTok{)} \OtherTok{→}\NormalTok{ a tt}\OtherTok{)}\NormalTok{ , }\OtherTok{λ} \OtherTok{(}\NormalTok{tt , a}\OtherTok{)}\NormalTok{ x }\OtherTok{→}\NormalTok{ tt , x}

\NormalTok{◃unitlCart }\OtherTok{:} \OtherTok{∀} \OtherTok{\{}\NormalTok{ℓ κ}\OtherTok{\}} \OtherTok{(}\NormalTok{p }\OtherTok{:}\NormalTok{ Poly ℓ κ}\OtherTok{)} 
             \OtherTok{→}\NormalTok{ isCartesian }\OtherTok{(}\NormalTok{𝕪 ◃ p}\OtherTok{)}\NormalTok{ p }\OtherTok{(}\NormalTok{◃unitl p}\OtherTok{)}
\NormalTok{◃unitlCart p }\OtherTok{(}\NormalTok{tt , a}\OtherTok{)} \OtherTok{=} 
\NormalTok{    Iso→isEquiv }\OtherTok{(} \OtherTok{(λ} \OtherTok{(}\NormalTok{tt , b}\OtherTok{)} \OtherTok{→}\NormalTok{ b}\OtherTok{)} 
\NormalTok{                , }\OtherTok{(λ}\NormalTok{ b\textquotesingle{} }\OtherTok{→}\NormalTok{ refl}\OtherTok{)} 
\NormalTok{                , }\OtherTok{(λ}\NormalTok{ b\textquotesingle{} }\OtherTok{→}\NormalTok{ refl}\OtherTok{)} \OtherTok{)}

\NormalTok{◃unitr }\OtherTok{:} \OtherTok{∀} \OtherTok{\{}\NormalTok{ℓ κ}\OtherTok{\}} \OtherTok{(}\NormalTok{p }\OtherTok{:}\NormalTok{ Poly ℓ κ}\OtherTok{)} \OtherTok{→} \OtherTok{(}\NormalTok{p ◃ 𝕪}\OtherTok{)}\NormalTok{ ⇆ p}
\NormalTok{◃unitr p }\OtherTok{=} \OtherTok{(λ} \OtherTok{(}\NormalTok{a , f}\OtherTok{)} \OtherTok{→}\NormalTok{ a}\OtherTok{)}\NormalTok{ , }\OtherTok{λ} \OtherTok{(}\NormalTok{a , f}\OtherTok{)}\NormalTok{ b }\OtherTok{→}\NormalTok{ b , tt}

\NormalTok{◃unitrCart }\OtherTok{:} \OtherTok{∀} \OtherTok{\{}\NormalTok{ℓ κ}\OtherTok{\}} \OtherTok{(}\NormalTok{p }\OtherTok{:}\NormalTok{ Poly ℓ κ}\OtherTok{)} 
             \OtherTok{→}\NormalTok{ isCartesian }\OtherTok{(}\NormalTok{p ◃ 𝕪}\OtherTok{)}\NormalTok{ p }\OtherTok{(}\NormalTok{◃unitr p}\OtherTok{)}
\NormalTok{◃unitrCart p }\OtherTok{(}\NormalTok{a , f}\OtherTok{)} \OtherTok{=}
\NormalTok{    Iso→isEquiv }\OtherTok{(} \OtherTok{(λ} \OtherTok{(}\NormalTok{b , tt}\OtherTok{)} \OtherTok{→}\NormalTok{ b}\OtherTok{)} 
\NormalTok{                , }\OtherTok{(λ}\NormalTok{ b }\OtherTok{→}\NormalTok{ refl}\OtherTok{)} 
\NormalTok{                , }\OtherTok{(λ} \OtherTok{(}\NormalTok{b , tt}\OtherTok{)} \OtherTok{→}\NormalTok{ refl}\OtherTok{)} \OtherTok{)}
\end{Highlighting}
\end{Shaded}

In fact, \texttt{◃} restricts to a monoidal product on
\(\mathbf{Poly^{Cart}}\), since the functorial action of \texttt{◃} on
lenses preserves Cartesian lenses:

\begin{Shaded}
\begin{Highlighting}[]
\NormalTok{◃LensCart }\OtherTok{:} \OtherTok{∀} \OtherTok{\{}\NormalTok{ℓ ℓ\textquotesingle{} ℓ\textquotesingle{}\textquotesingle{} ℓ\textquotesingle{}\textquotesingle{}\textquotesingle{} κ κ\textquotesingle{} κ\textquotesingle{}\textquotesingle{} κ\textquotesingle{}\textquotesingle{}\textquotesingle{}}\OtherTok{\}}
            \OtherTok{→} \OtherTok{(}\NormalTok{p }\OtherTok{:}\NormalTok{ Poly ℓ κ}\OtherTok{)} \OtherTok{(}\NormalTok{q }\OtherTok{:}\NormalTok{ Poly ℓ\textquotesingle{} κ\textquotesingle{}}\OtherTok{)}
            \OtherTok{→} \OtherTok{(}\NormalTok{r }\OtherTok{:}\NormalTok{ Poly ℓ\textquotesingle{}\textquotesingle{} κ\textquotesingle{}\textquotesingle{}}\OtherTok{)} \OtherTok{(}\NormalTok{s }\OtherTok{:}\NormalTok{ Poly ℓ\textquotesingle{}\textquotesingle{}\textquotesingle{} κ\textquotesingle{}\textquotesingle{}\textquotesingle{}}\OtherTok{)}
            \OtherTok{→} \OtherTok{(}\NormalTok{f }\OtherTok{:}\NormalTok{ p ⇆ q}\OtherTok{)} \OtherTok{(}\NormalTok{g }\OtherTok{:}\NormalTok{ r ⇆ s}\OtherTok{)}
            \OtherTok{→}\NormalTok{ isCartesian p q f }\OtherTok{→}\NormalTok{ isCartesian r s g}
            \OtherTok{→}\NormalTok{ isCartesian }\OtherTok{(}\NormalTok{p ◃ r}\OtherTok{)} \OtherTok{(}\NormalTok{q ◃ s}\OtherTok{)}
                          \OtherTok{(}\NormalTok{◃Lens p q r s f g}\OtherTok{)}
\NormalTok{◃LensCart p q r s }\OtherTok{(}\NormalTok{f , f♯}\OtherTok{)} \OtherTok{(}\NormalTok{g , g♯}\OtherTok{)}\NormalTok{ cf cg }\OtherTok{(}\NormalTok{a , h}\OtherTok{)} \OtherTok{=} 
\NormalTok{    pairEquiv }\OtherTok{(}\NormalTok{f♯ a}\OtherTok{)} \OtherTok{(λ}\NormalTok{ x }\OtherTok{→}\NormalTok{ g♯ }\OtherTok{(}\NormalTok{h }\OtherTok{(}\NormalTok{f♯ a x}\OtherTok{)))} 
              \OtherTok{(}\NormalTok{cf a}\OtherTok{)} \OtherTok{(λ}\NormalTok{ x }\OtherTok{→}\NormalTok{ cg }\OtherTok{(}\NormalTok{h }\OtherTok{(}\NormalTok{f♯ a x}\OtherTok{)))}
\end{Highlighting}
\end{Shaded}

We should expect, then, for these equivalences to be somehow reflected
in the structure of a Cartesian lenses \texttt{η\ :\ 𝕪\ ⇆\ 𝔲} and
\texttt{μ\ :\ 𝔲\ ◃\ 𝔲\ ⇆\ 𝔲}. This would be the case, e.g., if the
following diagrams in \(\mathbf{Poly^{Cart}}\) were to commute \[
\begin{tikzcd}
    {y \triangleleft \mathfrak{u}} & {\mathfrak{u} \triangleleft \mathfrak{u} } & {\mathfrak{u} \triangleleft y} \\
    & {\mathfrak{u}}
    \arrow["{\eta \triangleleft \mathfrak{u}}", from=1-1, to=1-2]
    \arrow["{\mathsf{\triangleleft unitl}}"{description}, from=1-1, to=2-2]
    \arrow["\mu", from=1-2, to=2-2]
    \arrow["{\mathfrak{u} \triangleleft \eta}"', from=1-3, to=1-2]
    \arrow["{\mathsf{\triangleleft unitr}}"{description}, from=1-3, to=2-2]
\end{tikzcd} \qquad \begin{tikzcd}
    {(\mathfrak{u} \triangleleft \mathfrak{u}) \triangleleft \mathfrak{u}} &[5pt]{\mathfrak{u} \triangleleft (\mathfrak{u} \triangleleft \mathfrak{u})} & {\mathfrak{u} \triangleleft \mathfrak{u}} \\
    {\mathfrak{u} \triangleleft \mathfrak{u}} && {\mathfrak{u}}
    \arrow["{\mathsf{\triangleleft assoc}}", from=1-1, to=1-2]
    \arrow["{\mu \triangleleft \mathfrak{u}}"', from=1-1, to=2-1]
    \arrow["{\mathfrak{u} \triangleleft \mu}", from=1-2, to=1-3]
    \arrow["\mu", from=1-3, to=2-3]
    \arrow["\mu"', from=2-1, to=2-3]
\end{tikzcd}
\]

One may recognize these as the usual diagrams for a monoid in a monoidal
category, hence (since \texttt{◃} corresponds to composition of
polynomial endofunctors) for a \emph{monad} as typically defined.
However, because of the higher-categorical structure of types in HoTT,
we should not only ask for these diagrams to commute, but for the cells
exhibiting that these diagrams commute to themselves be subject to
higher coherences, and so on, giving \texttt{𝔲} not the structure of a
(Cartesian) monad, but rather of a (Cartesian) \emph{\(\infty\)-monad}.

Yet demonstrating that \(𝔲\) is an \(\infty\)-monad involves specifying
a potentially infinite amount of coherence data. Have we therefore
traded both the Scylla of equality-up-to-isomorphism and the Charybdis
of strictness for an even worse fate of higher coherence hell? The
answer to this question, surprisingly, is negative, as there is a way to
implicitly derive all of this data from a single axiom, which
corresponds to the characteristic axiom of HoTT itself: univalence. To
show this, we now introduce the central concept of this paper -- that of
a \emph{polynomial universe}.

\chapter{Polynomial Universes}\label{polynomial-universes}

\dnote{Introduce the section}

\section{Univalence}\label{univalence}

For any polynomial \texttt{𝔲\ =\ (A\ ,\ B)} and elements
\texttt{a,b\ :\ A}, we may define a function that takes a proof of
\texttt{a\ ≡\ b} to an equivalence \texttt{B\ a\ ≃\ B\ b}.

\begin{Shaded}
\begin{Highlighting}[]
\NormalTok{idToEquiv }\OtherTok{:} \OtherTok{∀} \OtherTok{\{}\NormalTok{ℓ κ}\OtherTok{\}} \OtherTok{(}\NormalTok{p }\OtherTok{:}\NormalTok{ Poly ℓ κ}\OtherTok{)} \OtherTok{(}\NormalTok{a b }\OtherTok{:}\NormalTok{ fst p}\OtherTok{)}
            \OtherTok{→}\NormalTok{ a ≡ b }\OtherTok{→}\NormalTok{ Σ }\OtherTok{(}\NormalTok{snd p a }\OtherTok{→}\NormalTok{ snd p b}\OtherTok{)}\NormalTok{ isEquiv}
\NormalTok{idToEquiv p a b e }\OtherTok{=} 
\NormalTok{      transp }\OtherTok{(}\NormalTok{snd p}\OtherTok{)}\NormalTok{ e}
\NormalTok{    , Iso→isEquiv }\OtherTok{(}\NormalTok{transp }\OtherTok{(}\NormalTok{snd p}\OtherTok{)} \OtherTok{(}\NormalTok{sym e}\OtherTok{)}\NormalTok{ , }\OtherTok{(}\NormalTok{syml e , symr e}\OtherTok{))}
\end{Highlighting}
\end{Shaded}

We say that a polynomial \texttt{𝔲} is \emph{univalent} if for all
\texttt{a,b\ :\ A}, this function is an equivalence.

\begin{Shaded}
\begin{Highlighting}[]
\NormalTok{isUnivalent }\OtherTok{:} \OtherTok{∀} \OtherTok{\{}\NormalTok{ℓ κ}\OtherTok{\}} \OtherTok{→}\NormalTok{ Poly ℓ κ }\OtherTok{→}\NormalTok{ Type }\OtherTok{(}\NormalTok{ℓ ⊔ κ}\OtherTok{)}
\NormalTok{isUnivalent }\OtherTok{(}\NormalTok{A , B}\OtherTok{)} \OtherTok{=} 
    \OtherTok{(}\NormalTok{a b }\OtherTok{:}\NormalTok{ A}\OtherTok{)} \OtherTok{→}\NormalTok{ isEquiv }\OtherTok{(}\NormalTok{idToEquiv }\OtherTok{(}\NormalTok{A , B}\OtherTok{)}\NormalTok{ a b}\OtherTok{)}
\end{Highlighting}
\end{Shaded}

We call this property of polynomials univalence by analogy with the
usual univalence axiom of HoTT. Indeed, the univalence axiom simply
states that the polynomial functor \texttt{(Type\ ,\ λ\ X\ →\ X)} is
itself univalent.

\begin{Shaded}
\begin{Highlighting}[]
\KeywordTok{postulate}
\NormalTok{    ua }\OtherTok{:} \OtherTok{∀} \OtherTok{\{}\NormalTok{ℓ}\OtherTok{\}} \OtherTok{→}\NormalTok{ isUnivalent }\OtherTok{(}\NormalTok{Type ℓ , }\OtherTok{λ}\NormalTok{ X }\OtherTok{→}\NormalTok{ X}\OtherTok{)}
\end{Highlighting}
\end{Shaded}

A key property of polynomial universes -- qua polynomial functors -- is
that every polynomial universe \texttt{𝔲} is a \emph{subterminal object}
in \(\mathbf{Poly^{Cart}}\), i.e.~for any other polynomial \texttt{p},
the type of Cartesian lenses \texttt{p\ ⇆\ 𝔲} is a proposition, i.e.~any
two Cartesian lenses with codomain \texttt{𝔲} are equal. \dnote{What do you think of saying ``i.e.~we say a polynomial $v$ is \emph{univalent} if any
two Cartesian lenses with codomain $v$ are equal''?}

\begin{Shaded}
\begin{Highlighting}[]
\NormalTok{isSubterminal }\OtherTok{:} \OtherTok{∀} \OtherTok{\{}\NormalTok{ℓ κ}\OtherTok{\}} \OtherTok{(}\NormalTok{u }\OtherTok{:}\NormalTok{ Poly ℓ κ}\OtherTok{)} \OtherTok{→}\NormalTok{ Setω}
\NormalTok{isSubterminal u }\OtherTok{=} \OtherTok{∀} \OtherTok{\{}\NormalTok{ℓ\textquotesingle{} κ\textquotesingle{}}\OtherTok{\}} \OtherTok{(}\NormalTok{p }\OtherTok{:}\NormalTok{ Poly ℓ\textquotesingle{} κ\textquotesingle{}}\OtherTok{)}
                  \OtherTok{→} \OtherTok{(}\NormalTok{f g }\OtherTok{:}\NormalTok{ p ⇆ u}\OtherTok{)}
                  \OtherTok{→}\NormalTok{ isCartesian p u f}
                  \OtherTok{→}\NormalTok{ isCartesian p u g}
                  \OtherTok{→}\NormalTok{ EqLens p u f g}
\end{Highlighting}
\end{Shaded}

To show this, we first prove the following \emph{transport lemma}, which
says that transporting along an identity \texttt{a\ ≡\ b} induced by an
equivalence \texttt{f\ :\ B\ a\ ≃\ B\ b} in a univalent polynomial
\texttt{p\ =\ (A\ ,\ B)} is equivalent to applying \texttt{f}.

\begin{Shaded}
\begin{Highlighting}[]
\NormalTok{transpLemma }\OtherTok{:} \OtherTok{∀} \OtherTok{\{}\NormalTok{ℓ κ}\OtherTok{\}} \OtherTok{\{}\NormalTok{𝔲 }\OtherTok{:}\NormalTok{ Poly ℓ κ}\OtherTok{\}}
              \OtherTok{→} \OtherTok{(}\NormalTok{univ }\OtherTok{:}\NormalTok{ isUnivalent 𝔲}\OtherTok{)} 
              \OtherTok{→} \OtherTok{\{}\NormalTok{a b }\OtherTok{:}\NormalTok{ fst 𝔲}\OtherTok{\}} \OtherTok{(}\NormalTok{f }\OtherTok{:}\NormalTok{ snd 𝔲 a }\OtherTok{→}\NormalTok{ snd 𝔲 b}\OtherTok{)}
              \OtherTok{→} \OtherTok{(}\NormalTok{ef }\OtherTok{:}\NormalTok{ isEquiv f}\OtherTok{)} \OtherTok{(}\NormalTok{x }\OtherTok{:}\NormalTok{ snd 𝔲 a}\OtherTok{)}
              \OtherTok{→}\NormalTok{ transp }\OtherTok{(}\NormalTok{snd 𝔲}\OtherTok{)} \OtherTok{(}\NormalTok{inv }\OtherTok{(}\NormalTok{univ a b}\OtherTok{)} \OtherTok{(}\NormalTok{f , ef}\OtherTok{))}\NormalTok{ x ≡ f x}
\NormalTok{transpLemma }\OtherTok{\{}\NormalTok{𝔲 }\OtherTok{=}\NormalTok{ 𝔲}\OtherTok{\}}\NormalTok{ univ }\OtherTok{\{}\NormalTok{a }\OtherTok{=}\NormalTok{ a}\OtherTok{\}} \OtherTok{\{}\NormalTok{b }\OtherTok{=}\NormalTok{ b}\OtherTok{\}}\NormalTok{ f ef x }\OtherTok{=} 
\NormalTok{    coAp }\OtherTok{(}\NormalTok{ap fst }\OtherTok{(}\NormalTok{snd }\OtherTok{(}\NormalTok{snd }\OtherTok{(}\NormalTok{univ a b}\OtherTok{))} \OtherTok{((}\NormalTok{f , ef}\OtherTok{))))}\NormalTok{ x}
\end{Highlighting}
\end{Shaded}

The result then follows:

\begin{Shaded}
\begin{Highlighting}[]
\NormalTok{univ→Subterminal }\OtherTok{:} \OtherTok{∀} \OtherTok{\{}\NormalTok{ℓ κ}\OtherTok{\}} \OtherTok{(}\NormalTok{u }\OtherTok{:}\NormalTok{ Poly ℓ κ}\OtherTok{)}
                   \OtherTok{→}\NormalTok{ isUnivalent u}
                   \OtherTok{→}\NormalTok{ isSubterminal u}
\NormalTok{univ→Subterminal u univ p f g cf cg a }\OtherTok{=} 
    \OtherTok{(}\NormalTok{ inv univfg }\OtherTok{(}\NormalTok{fg⁻¹ , efg⁻¹}\OtherTok{)} 
\NormalTok{    , }\OtherTok{(λ}\NormalTok{ b }\OtherTok{→}\NormalTok{ sym }\OtherTok{((}\NormalTok{snd g a }\OtherTok{(}\NormalTok{transp }\OtherTok{(}\NormalTok{snd u}\OtherTok{)}  \OtherTok{(}\NormalTok{inv univfg }\OtherTok{(}\NormalTok{fg⁻¹ , efg⁻¹}\OtherTok{))}\NormalTok{ b}\OtherTok{))} 
\NormalTok{                  ≡〈 }\OtherTok{(}\NormalTok{ap }\OtherTok{(}\NormalTok{snd g a}\OtherTok{)} \OtherTok{(}\NormalTok{transpLemma univ fg⁻¹ efg⁻¹ b}\OtherTok{))}\NormalTok{ 〉 }
                  \OtherTok{((}\NormalTok{snd g a }\OtherTok{(}\NormalTok{fg⁻¹ b}\OtherTok{))} 
\NormalTok{                  ≡〈 snd }\OtherTok{(}\NormalTok{snd }\OtherTok{(}\NormalTok{cg a}\OtherTok{))} \OtherTok{(}\NormalTok{snd f a b}\OtherTok{)}\NormalTok{ 〉 }
                  \OtherTok{((}\NormalTok{snd f a b}\OtherTok{)}\NormalTok{ □}\OtherTok{)))))}
    \KeywordTok{where}\NormalTok{ univfg }\OtherTok{:}\NormalTok{ isEquiv }\OtherTok{(}\NormalTok{idToEquiv u }\OtherTok{(}\NormalTok{fst f a}\OtherTok{)} \OtherTok{(}\NormalTok{fst g a}\OtherTok{))}
\NormalTok{          univfg }\OtherTok{=}\NormalTok{ univ }\OtherTok{(}\NormalTok{fst f a}\OtherTok{)} \OtherTok{(}\NormalTok{fst g a}\OtherTok{)}
\NormalTok{          fg⁻¹ }\OtherTok{:}\NormalTok{ snd u }\OtherTok{(}\NormalTok{fst f a}\OtherTok{)} \OtherTok{→}\NormalTok{ snd u }\OtherTok{(}\NormalTok{fst g a}\OtherTok{)}
\NormalTok{          fg⁻¹ x }\OtherTok{=}\NormalTok{ inv }\OtherTok{(}\NormalTok{cg a}\OtherTok{)} \OtherTok{(}\NormalTok{snd f a x}\OtherTok{)}
\NormalTok{          efg⁻¹ }\OtherTok{:}\NormalTok{ isEquiv fg⁻¹}
\NormalTok{          efg⁻¹ }\OtherTok{=}\NormalTok{ compIsEquiv }\OtherTok{(}\NormalTok{inv }\OtherTok{(}\NormalTok{cg a}\OtherTok{))} \OtherTok{(}\NormalTok{snd f a}\OtherTok{)} 
                              \OtherTok{(}\NormalTok{invIsEquiv }\OtherTok{(}\NormalTok{cg a}\OtherTok{))} \OtherTok{(}\NormalTok{cf a}\OtherTok{)}
\end{Highlighting}
\end{Shaded}

We shall refer to polynomial functors with this property of being
subterminal objects in \(\mathbf{Poly^{Cart}}\) as \emph{polynomial
universes.} As we shall see, such polynomial universes have many
desirable properties as models of type theory.

If we think of a polynomial \texttt{p} as representing a family of
types, then what this tells us is that if \texttt{𝔲} is a polynomial
universe, there is essentially at most one way for \texttt{𝔲} to contain
the types represented by \texttt{p}, where Containment is here
understood as  existence of a Cartesian lens \texttt{p\ ⇆\ 𝔲}. In this
case, we say that \texttt{𝔲} \emph{classifies} the types represented by
\texttt{p}.

As a direct consequence of this fact\dnote{which fact? say again, since the last thing discussed was just terminology.}, it follows that every diagram
consisting of parallel Cartesian lenses into a polynomial universe
automatically commutes, and moreover, every higher diagram that can be
formed between the cells exhibiting such commutation must also commute,
etc.

Hence, due to the above theorem
and the closure of Cartesian lenses under composition, it follows that \texttt{𝔲} \emph{automatically} satisfies the laws of a monad if there
are Cartesian lenses \texttt{η\ :\ 𝕪\ ⇆\ 𝔲} and
\texttt{μ\ :\ 𝔲\ ◃\ 𝔲\ ⇆\ 𝔲}.

\begin{Shaded}
\begin{Highlighting}[]
\NormalTok{univ◃unitl }\OtherTok{:} \OtherTok{∀} \OtherTok{\{}\NormalTok{ℓ κ}\OtherTok{\}} \OtherTok{(}\NormalTok{𝔲 }\OtherTok{:}\NormalTok{ Poly ℓ κ}\OtherTok{)} \OtherTok{→}\NormalTok{ isUnivalent 𝔲}
             \OtherTok{→} \OtherTok{(}\NormalTok{η }\OtherTok{:}\NormalTok{ 𝕪 ⇆ 𝔲}\OtherTok{)} \OtherTok{(}\NormalTok{μ }\OtherTok{:} \OtherTok{(}\NormalTok{𝔲 ◃ 𝔲}\OtherTok{)}\NormalTok{ ⇆ 𝔲}\OtherTok{)}
             \OtherTok{→}\NormalTok{ isCartesian 𝕪 𝔲 η }\OtherTok{→}\NormalTok{ isCartesian }\OtherTok{(}\NormalTok{𝔲 ◃ 𝔲}\OtherTok{)}\NormalTok{ 𝔲 μ}
             \OtherTok{→}\NormalTok{ EqLens }\OtherTok{(}\NormalTok{𝕪 ◃ 𝔲}\OtherTok{)}\NormalTok{ 𝔲 }
                      \OtherTok{(}\NormalTok{◃unitl 𝔲}\OtherTok{)}
                      \OtherTok{(}\NormalTok{comp }\OtherTok{(}\NormalTok{𝕪 ◃ 𝔲}\OtherTok{)} \OtherTok{(}\NormalTok{𝔲 ◃ 𝔲}\OtherTok{)}\NormalTok{ 𝔲}
                            \OtherTok{(}\NormalTok{◃Lens 𝕪 𝔲 𝔲 𝔲 η }\OtherTok{(}\NormalTok{id 𝔲}\OtherTok{))}\NormalTok{ μ}\OtherTok{)}
                            
\NormalTok{univ◃unitl 𝔲 univ η μ cη cμ }\OtherTok{=}
\NormalTok{    univ→Subterminal }
\NormalTok{        𝔲 univ }\OtherTok{(}\NormalTok{𝕪 ◃ 𝔲}\OtherTok{)} \OtherTok{(}\NormalTok{◃unitl 𝔲}\OtherTok{)}
        \OtherTok{(}\NormalTok{comp }\OtherTok{(}\NormalTok{𝕪 ◃ 𝔲}\OtherTok{)} \OtherTok{(}\NormalTok{𝔲 ◃ 𝔲}\OtherTok{)}\NormalTok{ 𝔲 }
              \OtherTok{(}\NormalTok{◃Lens 𝕪 𝔲 𝔲 𝔲 η }\OtherTok{(}\NormalTok{id 𝔲}\OtherTok{))}\NormalTok{ μ}\OtherTok{)} 
        \OtherTok{(}\NormalTok{◃unitlCart 𝔲}\OtherTok{)} 
        \OtherTok{(}\NormalTok{compCartesian }\OtherTok{(}\NormalTok{𝕪 ◃ 𝔲}\OtherTok{)} \OtherTok{(}\NormalTok{𝔲 ◃ 𝔲}\OtherTok{)}\NormalTok{ 𝔲}
                       \OtherTok{(}\NormalTok{◃Lens 𝕪 𝔲 𝔲 𝔲 η }\OtherTok{(}\NormalTok{id 𝔲}\OtherTok{))}\NormalTok{ μ }
                       \OtherTok{(}\NormalTok{◃LensCart 𝕪 𝔲 𝔲 𝔲 η }\OtherTok{(}\NormalTok{id 𝔲}\OtherTok{)} 
\NormalTok{                                  cη }\OtherTok{(}\NormalTok{idCart 𝔲}\OtherTok{))}\NormalTok{ cμ}\OtherTok{)}

\NormalTok{univ◃unitr }\OtherTok{:} \OtherTok{∀} \OtherTok{\{}\NormalTok{ℓ κ}\OtherTok{\}} \OtherTok{(}\NormalTok{𝔲 }\OtherTok{:}\NormalTok{ Poly ℓ κ}\OtherTok{)} \OtherTok{→}\NormalTok{ isUnivalent 𝔲}
             \OtherTok{→} \OtherTok{(}\NormalTok{η }\OtherTok{:}\NormalTok{ 𝕪 ⇆ 𝔲}\OtherTok{)} \OtherTok{(}\NormalTok{μ }\OtherTok{:} \OtherTok{(}\NormalTok{𝔲 ◃ 𝔲}\OtherTok{)}\NormalTok{ ⇆ 𝔲}\OtherTok{)}
             \OtherTok{→}\NormalTok{ isCartesian 𝕪 𝔲 η }\OtherTok{→}\NormalTok{ isCartesian }\OtherTok{(}\NormalTok{𝔲 ◃ 𝔲}\OtherTok{)}\NormalTok{ 𝔲 μ}
             \OtherTok{→}\NormalTok{ EqLens }\OtherTok{(}\NormalTok{𝔲 ◃ 𝕪}\OtherTok{)}\NormalTok{ 𝔲}
                      \OtherTok{(}\NormalTok{◃unitr 𝔲}\OtherTok{)}
                      \OtherTok{(}\NormalTok{comp }\OtherTok{(}\NormalTok{𝔲 ◃ 𝕪}\OtherTok{)} \OtherTok{(}\NormalTok{𝔲 ◃ 𝔲}\OtherTok{)}\NormalTok{ 𝔲}
                            \OtherTok{(}\NormalTok{◃Lens 𝔲 𝔲 𝕪 𝔲 }\OtherTok{(}\NormalTok{id 𝔲}\OtherTok{)}\NormalTok{ η}\OtherTok{)}\NormalTok{ μ}\OtherTok{)}
\NormalTok{univ◃unitr 𝔲 univ η μ cη cμ }\OtherTok{=}
\NormalTok{    univ→Subterminal }
\NormalTok{        𝔲 univ }\OtherTok{(}\NormalTok{𝔲 ◃ 𝕪}\OtherTok{)} \OtherTok{(}\NormalTok{◃unitr 𝔲}\OtherTok{)} 
        \OtherTok{(}\NormalTok{comp }\OtherTok{(}\NormalTok{𝔲 ◃ 𝕪}\OtherTok{)} \OtherTok{(}\NormalTok{𝔲 ◃ 𝔲}\OtherTok{)}\NormalTok{ 𝔲 }
              \OtherTok{(}\NormalTok{◃Lens 𝔲 𝔲 𝕪 𝔲 }\OtherTok{(}\NormalTok{id 𝔲}\OtherTok{)}\NormalTok{ η}\OtherTok{)}\NormalTok{ μ}\OtherTok{)} 
        \OtherTok{(}\NormalTok{◃unitrCart 𝔲}\OtherTok{)} 
        \OtherTok{(}\NormalTok{compCartesian }\OtherTok{(}\NormalTok{𝔲 ◃ 𝕪}\OtherTok{)} \OtherTok{(}\NormalTok{𝔲 ◃ 𝔲}\OtherTok{)}\NormalTok{ 𝔲 }
                       \OtherTok{(}\NormalTok{◃Lens 𝔲 𝔲 𝕪 𝔲 }\OtherTok{(}\NormalTok{id 𝔲}\OtherTok{)}\NormalTok{ η}\OtherTok{)}\NormalTok{ μ }
                       \OtherTok{(}\NormalTok{◃LensCart 𝔲 𝔲 𝕪 𝔲 }\OtherTok{(}\NormalTok{id 𝔲}\OtherTok{)}\NormalTok{ η }
                                  \OtherTok{(}\NormalTok{idCart 𝔲}\OtherTok{)}\NormalTok{ cη}\OtherTok{)}\NormalTok{ cμ}\OtherTok{)}


\NormalTok{univ◃assoc }\OtherTok{:} \OtherTok{∀} \OtherTok{\{}\NormalTok{ℓ κ}\OtherTok{\}} \OtherTok{(}\NormalTok{𝔲 }\OtherTok{:}\NormalTok{ Poly ℓ κ}\OtherTok{)} \OtherTok{→}\NormalTok{ isUnivalent 𝔲}
             \OtherTok{→} \OtherTok{(}\NormalTok{η }\OtherTok{:}\NormalTok{ 𝕪 ⇆ 𝔲}\OtherTok{)} \OtherTok{(}\NormalTok{μ }\OtherTok{:} \OtherTok{(}\NormalTok{𝔲 ◃ 𝔲}\OtherTok{)}\NormalTok{ ⇆ 𝔲}\OtherTok{)}
             \OtherTok{→}\NormalTok{ isCartesian 𝕪 𝔲 η }\OtherTok{→}\NormalTok{ isCartesian }\OtherTok{(}\NormalTok{𝔲 ◃ 𝔲}\OtherTok{)}\NormalTok{ 𝔲 μ}
             \OtherTok{→}\NormalTok{ EqLens }\OtherTok{((}\NormalTok{𝔲 ◃ 𝔲}\OtherTok{)}\NormalTok{ ◃ 𝔲}\OtherTok{)}\NormalTok{ 𝔲}
                      \OtherTok{(}\NormalTok{comp }\OtherTok{((}\NormalTok{𝔲 ◃ 𝔲}\OtherTok{)}\NormalTok{ ◃ 𝔲}\OtherTok{)} \OtherTok{(}\NormalTok{𝔲 ◃ 𝔲}\OtherTok{)}\NormalTok{ 𝔲}
                            \OtherTok{(}\NormalTok{◃Lens }\OtherTok{(}\NormalTok{𝔲 ◃ 𝔲}\OtherTok{)}\NormalTok{ 𝔲 𝔲 𝔲 μ }\OtherTok{(}\NormalTok{id 𝔲}\OtherTok{))}\NormalTok{ μ}\OtherTok{)}
                      \OtherTok{(}\NormalTok{comp }\OtherTok{((}\NormalTok{𝔲 ◃ 𝔲}\OtherTok{)}\NormalTok{ ◃ 𝔲}\OtherTok{)} \OtherTok{(}\NormalTok{𝔲 ◃ }\OtherTok{(}\NormalTok{𝔲 ◃ 𝔲}\OtherTok{))}\NormalTok{ 𝔲}
                            \OtherTok{(}\NormalTok{◃assoc 𝔲 𝔲 𝔲}\OtherTok{)}
                            \OtherTok{(}\NormalTok{comp }\OtherTok{(}\NormalTok{𝔲 ◃ }\OtherTok{(}\NormalTok{𝔲 ◃ 𝔲}\OtherTok{))} \OtherTok{(}\NormalTok{𝔲 ◃ 𝔲}\OtherTok{)}\NormalTok{ 𝔲}
                                  \OtherTok{(}\NormalTok{◃Lens 𝔲 𝔲 }\OtherTok{(}\NormalTok{𝔲 ◃ 𝔲}\OtherTok{)}\NormalTok{ 𝔲 }
                                         \OtherTok{(}\NormalTok{id 𝔲}\OtherTok{)}\NormalTok{ μ}\OtherTok{)}\NormalTok{ μ}\OtherTok{))}
\NormalTok{univ◃assoc 𝔲 univ η μ cη cμ }\OtherTok{=}
\NormalTok{    univ→Subterminal }
\NormalTok{        𝔲 univ }\OtherTok{((}\NormalTok{𝔲 ◃ 𝔲}\OtherTok{)}\NormalTok{ ◃ 𝔲}\OtherTok{)} 
        \OtherTok{(}\NormalTok{comp }\OtherTok{((}\NormalTok{𝔲 ◃ 𝔲}\OtherTok{)}\NormalTok{ ◃ 𝔲}\OtherTok{)} \OtherTok{(}\NormalTok{𝔲 ◃ 𝔲}\OtherTok{)}\NormalTok{ 𝔲 }
              \OtherTok{(}\NormalTok{◃Lens }\OtherTok{(}\NormalTok{𝔲 ◃ 𝔲}\OtherTok{)}\NormalTok{ 𝔲 𝔲 𝔲 μ }\OtherTok{(}\NormalTok{id 𝔲}\OtherTok{))}\NormalTok{ μ}\OtherTok{)} 
        \OtherTok{(}\NormalTok{comp }\OtherTok{((}\NormalTok{𝔲 ◃ 𝔲}\OtherTok{)}\NormalTok{ ◃ 𝔲}\OtherTok{)} \OtherTok{(}\NormalTok{𝔲 ◃ }\OtherTok{(}\NormalTok{𝔲 ◃ 𝔲}\OtherTok{))}\NormalTok{ 𝔲 }
              \OtherTok{(}\NormalTok{◃assoc 𝔲 𝔲 𝔲}\OtherTok{)} 
              \OtherTok{(}\NormalTok{comp }\OtherTok{(}\NormalTok{𝔲 ◃ }\OtherTok{(}\NormalTok{𝔲 ◃ 𝔲}\OtherTok{))} \OtherTok{(}\NormalTok{𝔲 ◃ 𝔲}\OtherTok{)}\NormalTok{ 𝔲 }
                    \OtherTok{(}\NormalTok{◃Lens 𝔲 𝔲 }\OtherTok{(}\NormalTok{𝔲 ◃ 𝔲}\OtherTok{)}\NormalTok{ 𝔲 }\OtherTok{(}\NormalTok{id 𝔲}\OtherTok{)}\NormalTok{ μ}\OtherTok{)}\NormalTok{ μ}\OtherTok{))} 
        \OtherTok{(}\NormalTok{compCartesian }\OtherTok{((}\NormalTok{𝔲 ◃ 𝔲}\OtherTok{)}\NormalTok{ ◃ 𝔲}\OtherTok{)} \OtherTok{(}\NormalTok{𝔲 ◃ 𝔲}\OtherTok{)}\NormalTok{ 𝔲 }
                       \OtherTok{(}\NormalTok{◃Lens }\OtherTok{(}\NormalTok{𝔲 ◃ 𝔲}\OtherTok{)}\NormalTok{ 𝔲 𝔲 𝔲 μ }\OtherTok{(}\NormalTok{id 𝔲}\OtherTok{))}\NormalTok{ μ }
                       \OtherTok{(}\NormalTok{◃LensCart }\OtherTok{(}\NormalTok{𝔲 ◃ 𝔲}\OtherTok{)}\NormalTok{ 𝔲 𝔲 𝔲 μ }\OtherTok{(}\NormalTok{id 𝔲}\OtherTok{)} 
\NormalTok{                                  cμ }\OtherTok{(}\NormalTok{idCart 𝔲}\OtherTok{))}\NormalTok{ cμ}\OtherTok{)}
        \OtherTok{(}\NormalTok{compCartesian }\OtherTok{((}\NormalTok{𝔲 ◃ 𝔲}\OtherTok{)}\NormalTok{ ◃ 𝔲}\OtherTok{)} \OtherTok{(}\NormalTok{𝔲 ◃ }\OtherTok{(}\NormalTok{𝔲 ◃ 𝔲}\OtherTok{))}\NormalTok{ 𝔲 }
                       \OtherTok{(}\NormalTok{◃assoc 𝔲 𝔲 𝔲}\OtherTok{)} 
                       \OtherTok{(}\NormalTok{comp }\OtherTok{(}\NormalTok{𝔲 ◃ }\OtherTok{(}\NormalTok{𝔲 ◃ 𝔲}\OtherTok{))} \OtherTok{(}\NormalTok{𝔲 ◃ 𝔲}\OtherTok{)}\NormalTok{ 𝔲 }
                             \OtherTok{(}\NormalTok{◃Lens 𝔲 𝔲 }\OtherTok{(}\NormalTok{𝔲 ◃ 𝔲}\OtherTok{)}\NormalTok{ 𝔲 }
                                    \OtherTok{(}\NormalTok{id 𝔲}\OtherTok{)}\NormalTok{ μ}\OtherTok{)}\NormalTok{ μ}\OtherTok{)} 
                       \OtherTok{(}\NormalTok{◃assocCart 𝔲 𝔲 𝔲}\OtherTok{)}
                       \OtherTok{(}\NormalTok{compCartesian }
                          \OtherTok{(}\NormalTok{𝔲 ◃ }\OtherTok{(}\NormalTok{𝔲 ◃ 𝔲}\OtherTok{))} \OtherTok{(}\NormalTok{𝔲 ◃ 𝔲}\OtherTok{)}\NormalTok{ 𝔲 }
                          \OtherTok{(}\NormalTok{◃Lens 𝔲 𝔲 }\OtherTok{(}\NormalTok{𝔲 ◃ 𝔲}\OtherTok{)}\NormalTok{ 𝔲 }\OtherTok{(}\NormalTok{id 𝔲}\OtherTok{)}\NormalTok{ μ}\OtherTok{)}\NormalTok{ μ }
                          \OtherTok{(}\NormalTok{◃LensCart 𝔲 𝔲 }\OtherTok{(}\NormalTok{𝔲 ◃ 𝔲}\OtherTok{)}\NormalTok{ 𝔲 }\OtherTok{(}\NormalTok{id 𝔲}\OtherTok{)}\NormalTok{ μ }
                                     \OtherTok{(}\NormalTok{idCart 𝔲}\OtherTok{)}\NormalTok{ cμ}\OtherTok{)}\NormalTok{ cμ}\OtherTok{))}
\end{Highlighting}
\end{Shaded}

And more generally, if written out, all the higher coherences of an \(\infty\)-monad
would follow from the contractibility of the types of Cartesian lenses
\texttt{p\ ⇆\ 𝔲} that can be formed using \texttt{μ} and \texttt{η}.

\subsection{Rezk Completion of Polynomial
Functors}\label{rezk-completion-of-polynomial-functors}

We have so far seen that polynomial universes are quite special objects
in the theory of polynomial functors in HoTT, but what good would such
special objects do us if they turned out to be exceedingly rare or
<<<<<<< HEAD
difficult to construct? In fact, we can show that for \emph{any}
polynomial functor, there exists a corresponding polynomial universe,
using a familiar construct from the theory of categories in HoTT -- the
\emph{Rezk Completion.} We will show that this construction allows us to
quotient any polynomial functor to a corresponding polynomial universe.
=======
difficult to construct? Moreover, although we have just demonstrated
that any polynomial universe is a natural model in Awodey \& Newstead's
sense \dnote{We did? I missed this. I'm guessing it's on its way :)}, one might be inclined to wonder what can be said about the
converse direction -- does every natural model give rise to a polynomial
universe?

In fact, we can answer this latter question in the affirmative, using a
familiar construct from the theory of categories in HoTT -- the
\emph{Rezk Completion.} In the case of polynomial functors/natural
models, we will show that this construction allows us to quotient any
polynomial functor to a corresponding polynomial universe, which
classifies the unit type and \texttt{Σ} types if the
original polynomial does.
>>>>>>> c4227511ca42871e9e61301e46f89843d55b675a

By our assumption of the univalence axiom, every polynomial functor
\texttt{p} is classified by \emph{some} univalent polynomial:

\begin{Shaded}
\begin{Highlighting}[]
\NormalTok{classifier }\OtherTok{:} \OtherTok{∀} \OtherTok{\{}\NormalTok{ℓ κ}\OtherTok{\}} \OtherTok{(}\NormalTok{p }\OtherTok{:}\NormalTok{ Poly ℓ κ}\OtherTok{)} \OtherTok{→}\NormalTok{ p ⇆ }\OtherTok{(}\NormalTok{Type κ , }\OtherTok{λ}\NormalTok{ X }\OtherTok{→}\NormalTok{ X}\OtherTok{)}
\NormalTok{classifier }\OtherTok{(}\NormalTok{A , B}\OtherTok{)} \OtherTok{=} \OtherTok{(}\NormalTok{B , }\OtherTok{λ}\NormalTok{ a b }\OtherTok{→}\NormalTok{ b}\OtherTok{)}

\NormalTok{classifierCart }\OtherTok{:} \OtherTok{∀} \OtherTok{\{}\NormalTok{ℓ κ}\OtherTok{\}} \OtherTok{(}\NormalTok{p }\OtherTok{:}\NormalTok{ Poly ℓ κ}\OtherTok{)} 
                 \OtherTok{→}\NormalTok{ isCartesian p }\OtherTok{(}\NormalTok{Type κ , }\OtherTok{λ}\NormalTok{ X }\OtherTok{→}\NormalTok{ X}\OtherTok{)}
                               \OtherTok{(}\NormalTok{classifier p}\OtherTok{)}
\NormalTok{classifierCart p a }\OtherTok{=}\NormalTok{ idIsEquiv}
\end{Highlighting}
\end{Shaded}

We then obtain the Rezk completion of \texttt{p} as the image
factorization in \(\mathbf{Poly^{Cart}}\) of this classifying lens:

\begin{Shaded}
\begin{Highlighting}[]
\NormalTok{Rezk }\OtherTok{:} \OtherTok{∀} \OtherTok{\{}\NormalTok{ℓ κ}\OtherTok{\}} \OtherTok{(}\NormalTok{p }\OtherTok{:}\NormalTok{ Poly ℓ κ}\OtherTok{)} \OtherTok{→}\NormalTok{ Poly }\OtherTok{(}\NormalTok{lsuc κ}\OtherTok{)}\NormalTok{ κ}
\NormalTok{Rezk }\OtherTok{(}\NormalTok{A , B}\OtherTok{)} \OtherTok{=} \OtherTok{(}\NormalTok{Im B}\OtherTok{)}\NormalTok{ , }\OtherTok{(λ} \OtherTok{(}\NormalTok{X , }\OtherTok{\_)} \OtherTok{→}\NormalTok{ X}\OtherTok{)}

\NormalTok{→Rezk }\OtherTok{:} \OtherTok{∀} \OtherTok{\{}\NormalTok{ℓ κ}\OtherTok{\}} \OtherTok{(}\NormalTok{p }\OtherTok{:}\NormalTok{ Poly ℓ κ}\OtherTok{)} \OtherTok{→}\NormalTok{ p ⇆ }\OtherTok{(}\NormalTok{Rezk p}\OtherTok{)}
\NormalTok{→Rezk }\OtherTok{\{}\NormalTok{κ }\OtherTok{=}\NormalTok{ κ}\OtherTok{\}}\NormalTok{ p }\OtherTok{=} 
\NormalTok{    factorcart1 p }\OtherTok{(}\NormalTok{Type κ , }\OtherTok{λ}\NormalTok{ X }\OtherTok{→}\NormalTok{ X}\OtherTok{)} 
                  \OtherTok{(}\NormalTok{classifier p}\OtherTok{)} 
                  \OtherTok{(}\NormalTok{classifierCart p}\OtherTok{)}

\NormalTok{Rezk→ }\OtherTok{:} \OtherTok{∀} \OtherTok{\{}\NormalTok{ℓ κ}\OtherTok{\}} \OtherTok{(}\NormalTok{p }\OtherTok{:}\NormalTok{ Poly ℓ κ}\OtherTok{)} \OtherTok{→} \OtherTok{(}\NormalTok{Rezk p}\OtherTok{)}\NormalTok{ ⇆ }\OtherTok{(}\NormalTok{Type κ , }\OtherTok{λ}\NormalTok{ X }\OtherTok{→}\NormalTok{ X}\OtherTok{)}
\NormalTok{Rezk→ }\OtherTok{\{}\NormalTok{κ }\OtherTok{=}\NormalTok{ κ}\OtherTok{\}}\NormalTok{ p }\OtherTok{=}
\NormalTok{    factorcart2 p }\OtherTok{(}\NormalTok{Type κ , }\OtherTok{λ}\NormalTok{ X }\OtherTok{→}\NormalTok{ X}\OtherTok{)} 
                  \OtherTok{(}\NormalTok{classifier p}\OtherTok{)} 
                  \OtherTok{(}\NormalTok{classifierCart p}\OtherTok{)}
\end{Highlighting}
\end{Shaded}

Because the map \texttt{Rezk→} defined above is a Cartesian embedding,
and the polynomial \texttt{(Type\ κ\ ,\ λ\ X\ →\ X)} is univalent, it
follows that \texttt{Rezk\ p} is a polynomial universe:

\begin{Shaded}
\begin{Highlighting}[]
\NormalTok{RezkSubterminal }\OtherTok{:} \OtherTok{∀} \OtherTok{\{}\NormalTok{ℓ κ}\OtherTok{\}} \OtherTok{(}\NormalTok{p }\OtherTok{:}\NormalTok{ Poly ℓ κ}\OtherTok{)} \OtherTok{→}\NormalTok{ isSubterminal }\OtherTok{(}\NormalTok{Rezk p}\OtherTok{)}
\NormalTok{RezkSubterminal }\OtherTok{\{}\NormalTok{κ }\OtherTok{=}\NormalTok{ κ}\OtherTok{\}}\NormalTok{ p q }\OtherTok{(}\NormalTok{f , f♯}\OtherTok{)} \OtherTok{(}\NormalTok{g , g♯}\OtherTok{)}\NormalTok{ cf cg x }\OtherTok{=}
    \OtherTok{(}\NormalTok{ pairEq }\OtherTok{(}\NormalTok{inv }\OtherTok{(}\NormalTok{ua }\OtherTok{(}\NormalTok{fst }\OtherTok{(}\NormalTok{f x}\OtherTok{))} \OtherTok{(}\NormalTok{fst }\OtherTok{(}\NormalTok{g x}\OtherTok{)))} 
                  \OtherTok{(} \OtherTok{(λ}\NormalTok{ y }\OtherTok{→}\NormalTok{ inv }\OtherTok{(}\NormalTok{cg x}\OtherTok{)} \OtherTok{(}\NormalTok{f♯ x y}\OtherTok{))} 
\NormalTok{                  , compIsEquiv }\OtherTok{(}\NormalTok{inv }\OtherTok{(}\NormalTok{cg x}\OtherTok{))} 
                                \OtherTok{(}\NormalTok{f♯ x}\OtherTok{)} 
                                \OtherTok{(}\NormalTok{invIsEquiv }\OtherTok{(}\NormalTok{cg x}\OtherTok{))} 
                                \OtherTok{(}\NormalTok{cf x}\OtherTok{)))}\NormalTok{ ∥{-}∥IsProp }
\NormalTok{    , }\OtherTok{λ}\NormalTok{ y }\OtherTok{→}\NormalTok{ f♯ x y }
\NormalTok{            ≡〈 sym }\OtherTok{(}\NormalTok{g♯ x }\OtherTok{(}\NormalTok{transp }\OtherTok{(λ}\NormalTok{ X }\OtherTok{→}\NormalTok{ X}\OtherTok{)} 
                                  \OtherTok{(}\NormalTok{inv }\OtherTok{(}\NormalTok{ua }\OtherTok{(}\NormalTok{fst }\OtherTok{(}\NormalTok{f x}\OtherTok{))} \OtherTok{(}\NormalTok{fst }\OtherTok{(}\NormalTok{g x}\OtherTok{)))} 
                                       \OtherTok{((λ}\NormalTok{ z }\OtherTok{→}\NormalTok{ inv }\OtherTok{(}\NormalTok{cg x}\OtherTok{)} \OtherTok{(}\NormalTok{f♯ x z}\OtherTok{))}\NormalTok{ , }\OtherTok{(}\NormalTok{compIsEquiv }\OtherTok{(}\NormalTok{inv }\OtherTok{(}\NormalTok{cg x}\OtherTok{))} \OtherTok{(}\NormalTok{f♯ x}\OtherTok{)} 
                                                    \OtherTok{(}\NormalTok{invIsEquiv }\OtherTok{(}\NormalTok{cg x}\OtherTok{))} 
                                                    \OtherTok{(}\NormalTok{cf x}\OtherTok{))))}\NormalTok{ y}\OtherTok{)}
\NormalTok{                    ≡〈 }\OtherTok{(}\NormalTok{ap }\OtherTok{(}\NormalTok{g♯ x}\OtherTok{)} 
                          \OtherTok{(}\NormalTok{transpLemma ua }
                             \OtherTok{(λ}\NormalTok{ z }\OtherTok{→}\NormalTok{ inv }\OtherTok{(}\NormalTok{cg x}\OtherTok{)} \OtherTok{(}\NormalTok{f♯ x z}\OtherTok{))} 
                             \OtherTok{(}\NormalTok{compIsEquiv }\OtherTok{(}\NormalTok{inv }\OtherTok{(}\NormalTok{cg x}\OtherTok{))} \OtherTok{(}\NormalTok{f♯ x}\OtherTok{)} 
                                          \OtherTok{(}\NormalTok{invIsEquiv }\OtherTok{(}\NormalTok{cg x}\OtherTok{))} \OtherTok{(}\NormalTok{cf x}\OtherTok{))} 
\NormalTok{                             y}\OtherTok{))}\NormalTok{ 〉 }
\NormalTok{                    snd }\OtherTok{(}\NormalTok{snd }\OtherTok{(}\NormalTok{cg x}\OtherTok{))} \OtherTok{(}\NormalTok{f♯ x y}\OtherTok{))}\NormalTok{ 〉 }
\NormalTok{            ap }\OtherTok{(}\NormalTok{g♯ x}\OtherTok{)} \OtherTok{(}\NormalTok{sym }\OtherTok{(}\NormalTok{lemma1 ∥{-}∥IsProp y}\OtherTok{))} \OtherTok{)}
    \KeywordTok{where}\NormalTok{ lemma1 }\OtherTok{:} \OtherTok{\{}\NormalTok{a b }\OtherTok{:}\NormalTok{ fst }\OtherTok{(}\NormalTok{Rezk p}\OtherTok{)\}}
                   \OtherTok{→} \OtherTok{\{}\NormalTok{e }\OtherTok{:}\NormalTok{ fst a ≡ fst b}\OtherTok{\}} 
                   \OtherTok{→} \OtherTok{(}\NormalTok{e\textquotesingle{} }\OtherTok{:}\NormalTok{ transp }\OtherTok{(λ}\NormalTok{ c }\OtherTok{→}\NormalTok{ ∥ }\OtherTok{(}\NormalTok{Fibre }\OtherTok{(}\NormalTok{snd p}\OtherTok{)}\NormalTok{ c}\OtherTok{)}\NormalTok{ ∥}\OtherTok{)} 
\NormalTok{                                  e }\OtherTok{(}\NormalTok{snd a}\OtherTok{)} 
\NormalTok{                           ≡ }\OtherTok{(}\NormalTok{snd b}\OtherTok{))}
                   \OtherTok{→} \OtherTok{(}\NormalTok{z }\OtherTok{:}\NormalTok{ fst a}\OtherTok{)}
                   \OtherTok{→}\NormalTok{ transp }\OtherTok{(}\NormalTok{snd }\OtherTok{(}\NormalTok{Rezk p}\OtherTok{))} \OtherTok{(}\NormalTok{pairEq e e\textquotesingle{}}\OtherTok{)}\NormalTok{ z}
\NormalTok{                     ≡ transp }\OtherTok{(λ}\NormalTok{ X }\OtherTok{→}\NormalTok{ X}\OtherTok{)}\NormalTok{ e z}
\NormalTok{          lemma1 }\OtherTok{\{}\NormalTok{e }\OtherTok{=}\NormalTok{ refl}\OtherTok{\}}\NormalTok{ refl z }\OtherTok{=}\NormalTok{ refl}
\end{Highlighting}
\end{Shaded}\dnote{Deal with long line.}

\chapter{\texorpdfstring{\(\Pi\)-Types, Jump Monads \& Distributive
Laws}{\textbackslash Pi-Types, Jump Monads \& Distributive Laws}}\label{pi-types-jump-monads-distributive-laws}

We have so far considered how polynomial universes may be equipped with
structure to interpret the unit type and dependent pair types. We have
not yet, however, said much in the way of \emph{dependent function
types.} In order to rectify this omission, it will first be prudent to
consider some additional structure on the category of polynomial
functors -- specifically a new functor
\(\upuparrows : \mathbf{Poly^{Cart}} \times \mathbf{Poly} \to \mathbf{Poly}\)
that plays a similar role for \texttt{Π} types as the composition
\(\triangleleft : \mathbf{Poly} \times \mathbf{Poly} \to \mathbf{Poly}\)
played for \texttt{Σ} types, and which in turn bears a close connection
to a class of structured morphisms in \(\mathbf{Poly}\), which we refer
to as \emph{jump morphisms.}

\section{\texorpdfstring{The \(\upuparrows\) Functor \& Jump
Morphisms}{The \textbackslash upuparrows Functor \& Jump Morphisms}}\label{the-upuparrows-functor-jump-morphisms}

The \(\upuparrows\) functor can be loosely defined as the solution to
the following problem: given a polynomial universe \texttt{𝔲}, find
\texttt{𝔲\ ⇈\ 𝔲} such that \texttt{𝔲} classifies \texttt{𝔲\ ⇈\ 𝔲} if and
only if \texttt{𝔲} has the structure to interpret \texttt{Π} types (in
the same way that \texttt{𝔲} classifies \texttt{𝔲\ ◃\ 𝔲} if and only if
\texttt{𝔲} has the structure to interpret \texttt{Σ} types).
Generalizing this to arbitrary pairs of polynomials
\(p = (A , B), ~ q = (C , D)\) then yields the following formula for
\(p \upuparrows q\): \[
p \upuparrows q = \sum_{(a , f) : \sum_{a : A} C^{B(a)}} y^{\prod_{b : B(a)} D(f(b))}
\]

\begin{Shaded}
\begin{Highlighting}[]
\OtherTok{\_}\NormalTok{⇈}\OtherTok{\_} \OtherTok{:} \OtherTok{∀} \OtherTok{\{}\NormalTok{ℓ ℓ\textquotesingle{} κ κ\textquotesingle{}}\OtherTok{\}} \OtherTok{→}\NormalTok{ Poly ℓ κ }\OtherTok{→}\NormalTok{ Poly ℓ\textquotesingle{} κ\textquotesingle{} }\OtherTok{→}\NormalTok{ Poly }\OtherTok{(}\NormalTok{ℓ ⊔ κ ⊔ ℓ\textquotesingle{}}\OtherTok{)} \OtherTok{(}\NormalTok{κ ⊔ κ\textquotesingle{}}\OtherTok{)}
\OtherTok{(}\NormalTok{A , B}\OtherTok{)}\NormalTok{ ⇈ }\OtherTok{(}\NormalTok{C , D}\OtherTok{)} \OtherTok{=} 
    \OtherTok{(}\NormalTok{ Σ A }\OtherTok{(λ}\NormalTok{ a }\OtherTok{→}\NormalTok{ B a }\OtherTok{→}\NormalTok{ C}\OtherTok{)} 
\NormalTok{    , }\OtherTok{(λ} \OtherTok{(}\NormalTok{a , f}\OtherTok{)} \OtherTok{→} \OtherTok{(}\NormalTok{b }\OtherTok{:}\NormalTok{ B a}\OtherTok{)} \OtherTok{→}\NormalTok{ D }\OtherTok{(}\NormalTok{f b}\OtherTok{)))}
\end{Highlighting}
\end{Shaded}

Note that this construction is straightforwardly functorial with respect
to arbitrary lenses in its 2nd argument, but is only functorial with
respect to Cartesian lenses in its first argument:

\begin{Shaded}
\begin{Highlighting}[]
\NormalTok{⇈Lens }\OtherTok{:} \OtherTok{∀} \OtherTok{\{}\NormalTok{ℓ ℓ\textquotesingle{} ℓ\textquotesingle{}\textquotesingle{} ℓ\textquotesingle{}\textquotesingle{}\textquotesingle{} κ κ\textquotesingle{} κ\textquotesingle{}\textquotesingle{} κ\textquotesingle{}\textquotesingle{}\textquotesingle{}}\OtherTok{\}}
        \OtherTok{→} \OtherTok{(}\NormalTok{p }\OtherTok{:}\NormalTok{ Poly ℓ κ}\OtherTok{)} \OtherTok{(}\NormalTok{q }\OtherTok{:}\NormalTok{ Poly ℓ\textquotesingle{} κ\textquotesingle{}}\OtherTok{)} 
        \OtherTok{→} \OtherTok{(}\NormalTok{r }\OtherTok{:}\NormalTok{ Poly ℓ\textquotesingle{}\textquotesingle{} κ\textquotesingle{}\textquotesingle{}}\OtherTok{)} \OtherTok{(}\NormalTok{s }\OtherTok{:}\NormalTok{ Poly ℓ\textquotesingle{}\textquotesingle{}\textquotesingle{} κ\textquotesingle{}\textquotesingle{}\textquotesingle{}}\OtherTok{)}
        \OtherTok{→} \OtherTok{(}\NormalTok{f }\OtherTok{:}\NormalTok{ p ⇆ r}\OtherTok{)} \OtherTok{→}\NormalTok{ isCartesian p r f}
        \OtherTok{→} \OtherTok{(}\NormalTok{g }\OtherTok{:}\NormalTok{ q ⇆ s}\OtherTok{)} \OtherTok{→} \OtherTok{(}\NormalTok{p ⇈ q}\OtherTok{)}\NormalTok{ ⇆ }\OtherTok{(}\NormalTok{r ⇈ s}\OtherTok{)}
\NormalTok{⇈Lens p q r s }\OtherTok{(}\NormalTok{f , f♯}\OtherTok{)}\NormalTok{ cf }\OtherTok{(}\NormalTok{g , g♯}\OtherTok{)} \OtherTok{=} 
    \OtherTok{(} \OtherTok{(λ} \OtherTok{(}\NormalTok{a , h}\OtherTok{)} \OtherTok{→} \OtherTok{(}\NormalTok{f a}\OtherTok{)}\NormalTok{ , }\OtherTok{(λ}\NormalTok{ b }\OtherTok{→}\NormalTok{ g }\OtherTok{(}\NormalTok{h }\OtherTok{(}\NormalTok{f♯ a b}\OtherTok{))))} 
\NormalTok{    , }\OtherTok{λ} \OtherTok{(}\NormalTok{a , h}\OtherTok{)}\NormalTok{ k b}
      \OtherTok{→}\NormalTok{ g♯ }\OtherTok{(}\NormalTok{h b}\OtherTok{)} \OtherTok{(}\NormalTok{transp }\OtherTok{(}\NormalTok{snd s}\OtherTok{)} 
                   \OtherTok{(}\NormalTok{ap g }\OtherTok{(}\NormalTok{ap h }\OtherTok{(}\NormalTok{snd }\OtherTok{(}\NormalTok{snd }\OtherTok{(}\NormalTok{cf a}\OtherTok{))}\NormalTok{ b}\OtherTok{)))} 
                   \OtherTok{(}\NormalTok{k }\OtherTok{(}\NormalTok{inv }\OtherTok{(}\NormalTok{cf a}\OtherTok{)}\NormalTok{ b}\OtherTok{))))}
\end{Highlighting}
\end{Shaded}

By construction, the existence of a Cartesian lens
\texttt{(π\ ,\ π♯)\ :\ 𝔲\ ◃\ 𝔲\ ⇆\ 𝔲} effectively shows that \texttt{𝔲}
is closed under \texttt{Π}-types, since:

\begin{itemize}
\tightlist
\item
  \texttt{π} maps a pair \texttt{(A\ ,\ B)} consisting of
  \texttt{A\ :\ 𝓤} and \texttt{B\ :\ u(A)\ →\ 𝓤} to a term
  \texttt{π(A,B)} representing the corresponding \texttt{Π} type. This
  corresponds to the type formation rule
  \[ \inferrule{\Gamma \vdash A : \mathsf{Type}\\ \Gamma, x : A \vdash B[x] ~ \mathsf{Type}}{\Gamma \vdash \Pi x : A . B[x] ~ \mathsf{Type}} \]
\item
  The ``elimination rule'' \texttt{π♯\ (A\ ,\ B)}, for any pair
  \texttt{(A\ ,\ B)} as above, maps an element \texttt{f\ :\ π(A,B)} to
  a function \texttt{π♯\ (A\ ,\ B)\ f\ :\ (a\ :\ u(A))\ →\ u\ (B\ x)}
  which takes an element \texttt{x} of \texttt{A} and yields an element
  of \texttt{B\ x}. This corresponds to the rule for function
  application: \[
  \inferrule{\Gamma \vdash f : \Pi x : A . B[x]\\ \Gamma \vdash a : A}{\Gamma \vdash f ~ a : B[a]}
  \]
\item
  Since \texttt{π♯\ (A\ ,\ B)} is an equivalence, it follows that there
  is an inverse
  \texttt{π♯⁻¹\ (A\ ,\ B)\ :\ ((x\ :\ u(A))\ →\ u(B(x))\ →\ u(π(A,B))},
  which corresponds to \(\lambda\)-abstraction: \[
  \inferrule{\Gamma, x : A \vdash f[x] : B[x]}{\Gamma \vdash \lambda x . f[x] : \Pi x : A . B[x]}
  \]
\item
  The fact that \texttt{π♯⁻¹\ (A\ ,\ B)} is both a left and a right
  inverse to \texttt{π♯} then corresponds to the \(\beta\) and \(\eta\)
  laws for \texttt{Π} types. \[
  (\lambda x . f[x]) ~ a = f[a] \qquad f = \lambda x . f ~ x
  \]
\end{itemize}

\section{From Jump Morphisms to Distributive
Laws}\label{from-jump-morphisms-to-distributive-laws}

\chapter{Other Type Formers in Polynomial
Universes}\label{other-type-formers-in-polynomial-universes}

\section{Identity Types}\label{identity-types}

\section{Positive Types}\label{positive-types}

\chapter{Conclusion}\label{conclusion}

\makeatletter
\@ifclassloaded{memoir}{\ifartopt\else\backmatter\fi}{\backmatter}
\makeatother
\end{document}
